\documentclass[12pt]{article}
\usepackage{config}
\usepackage{subfiles}
\pgfplotsset{compat=1.18}

\begin{document}

\begin{flushright}
    Конспект Шорохова Сергея

    Если нашли опечатку/ошибку - пишите @le9endwp 
\end{flushright}

\tableofcontents
\newpage

\section{Глава 9. Теория меры}

\subsection{§1. Системы множеств}

\begin{defin}{Объемлющее множество}
    $X$ -- объемлющее множество. Будем рассматривать $A \subset X$
\end{defin}

\begin{declar}{Обозначения}
    $A \sqcup B$ -- объединение множеств $A$ и $B$ и множества $A$ и $B$ не пересекаются

    $\bigsqcup\limits_{k = 1}^n A_k$ -- объединение и $A_i \cap A_j = \varnothing$

    Дизъюнктные множества = непересекающиеся множества
\end{declar}

\begin{defin}{Разбиение множества}
    Множества $E_\alpha,\ \alpha \in I$ -- разбиение множества $E$, если $E = \bigsqcup\limits_{a \in I} E_\alpha$
\end{defin}

\begin{defin}{Система подмножеств и ее свойства}
    $\A$ -- система подмножеств $X$ (т.е. $\A \subset 2^X$)

    \begin{enumerate}
        \item $\A$ имеет свойство $\sigma_0$, если $\forall A, B \in \A \Rightarrow A \cup B \in \A$
        \item $\A$ имеет свойство $\delta_0$, если $\forall A, B \in \A \Rightarrow A \cap B \in \A$
        \item $\A$ имеет свойство $\sigma$, если $\forall A_1, A_2 \ldots \in \A \Rightarrow \bigcup\limits_{n = 1}^\infty A_n \in \A$
        \item $\A$ имеет свойство $\delta$, если $\forall A_1, A_2 \ldots \in \A \Rightarrow \bigcap\limits_{n = 1}^\infty A_n \in \A$
        \item $\A$ -- симметричная система, если $\forall A \in \A \Rightarrow X \setminus A \in \A$
    \end{enumerate}
\end{defin}

\begin{Reminder}{}
    $X \setminus \bigcup\limits_{\alpha \in I} A_\alpha = \bigcap\limits_{\alpha \in I} X \setminus A_\alpha$

    $X \setminus \bigcap\limits_{\alpha \in I} A_\alpha = \bigcup\limits_{\alpha \in I} X \setminus A_\alpha$
\end{Reminder}

\begin{propos}{}
    Если $\A$ симметричная система, то $\begin{gathered}
        (\sigma_0) \Leftrightarrow (\delta_0) \\
        (\sigma) \Leftrightarrow (\delta)
    \end{gathered}$
\end{propos}

\begin{defin}{Алгебра}
    $\A$ -- алгебра, если

    \begin{enumerate}
        \item $\varnothing \in \A$
        \item $\A$ -- симметричная система 
        \item Есть свойства $(\sigma_0)$ и $(\delta_0)$
    \end{enumerate}
\end{defin}

\begin{defin}{$\sigma$-алгебра}
    $\A$ -- $\sigma$-алгебра, если 

    \begin{enumerate}
        \item $\varnothing \in \A$
        \item $\A$ -- симметричная система
        \item Есть свойства $(\sigma)$ и $(\delta)$
    \end{enumerate}
\end{defin}

\begin{theo}{Свойства}
    \begin{enumerate}
        \item Если $\A$ -- алгебра и $A_1 \ldots A_n \in \A$, то $\bigcup\limits_{k = 1}^n A_k$ и $\bigcap\limits_{k = 1}^n A_k \in \A$
        \item Если $\A$ -- $\sigma$-алгебра, то $\A$ -- алгебра
        \item Если $\A$ -- алгебра и $A, B \in \A$, то $\underbrace{A \setminus B}_{A \cap (X \setminus B)} \in \A$ 
    \end{enumerate}
\end{theo}

\begin{Example}{}
    \begin{enumerate}
        \item $X = \R^n$
        
        $\A$ -- все ограниченные множества и их дополнения. Это алгебра, но не $\sigma$-алгебра

        \item $2^X$ -- $\sigma$-алгебра
        \item Индуцированная ($\sigma$-)алгебра 
        
        $Y \subset X,\ \A$ -- ($\sigma$-)алгебра подмножеств $X$

        $\B := \{A \cap Y : A \in \A\}$ -- ($\sigma$-)алгебра подмножеств $Y$

        \item $X \supset A, B$
        
        $\A$ -- алгебра подмножеств $X$

        $\varnothing, X, A, B, A \cup B, A \cap B, A \setminus B, B \setminus A, X \setminus A, X \setminus B, A \triangle B, X \setminus (A \cap B), X \setminus (A \cup B), \\ X \setminus (A \triangle B), X \setminus (A \setminus B), X \setminus (B \setminus A)$

        \item $A_\alpha$ -- ($\sigma$-)алгебра подмножеств $X$
        
        Тогда $\B = \bigcap\limits_{\alpha \in I} \A_\alpha$ -- ($\sigma$-)алгебра подмножеств $X$

        \textit{Доказательство:}

        \begin{enumerate}
            \item $\varnothing \in \A_\alpha \Rightarrow \varnothing \in \B$
            \item $A \in \B \Rightarrow A \in \A_\alpha \forall \alpha \Rightarrow X \setminus A \in \A_\alpha \forall \alpha \Rightarrow X \setminus A \in \B$
        \end{enumerate}
    \end{enumerate}
\end{Example}

\begin{theo}{}
    Пусть $\E$ -- система подмножеств $X$

    Тогда существует наименьшая по включению ($\sigma$-)алгебра $\A$, содержащая $\E$
\end{theo}

\textit{Доказательство:}

Пусть $\A_\alpha$ -- всевозможные алгебры, содержащие $\E$ ($2^X$ подходит)

$\A := \bigcap\limits_{\alpha \in I} \A_\alpha$ -- алгебра и $\A \subset \A_\alpha \forall \alpha$

\newpage

\begin{defin}{Борелевская оболочка}
    $\E$ -- система подмножеств $X$

    Борелевская оболочка системы $\E$ -- наименьшая по включению $\sigma$-алгебра, содержащая $\E$
\end{defin}

\begin{declar}{Обозначение}
    $\B(\E)$
\end{declar}

\begin{defin}{Борелевская $\sigma$-алгебра}
    Борелевская $\sigma$-алгебра -- это $\B(\E)$, где $\E$ -- всевозможные открытые множества в $\R^n$
\end{defin}

\begin{declar}{Обозначение}
    $\B^n$
\end{declar}

\begin{Remark}{}
    $\B^n \neq 2^{\R^n}$
\end{Remark}

\begin{defin}{Кольцо}
    $\A$ -- семейство подмножеств $X$

    $\A$ -- кольцо, если 
    
    \begin{enumerate}
        \item $\varnothing \in \A$
        \item $A, B \in \A \Rightarrow A \cap B \in \A,\ A \cup B \in \A$
        \item $A, B \in \A \Rightarrow A \setminus B \in \A$
    \end{enumerate}
\end{defin}

\begin{Remark}{}
    $\A$ -- алгебра $\Leftrightarrow \A$ -- кольцо и $X \in \A$
\end{Remark}

\begin{defin}{}
    $\P$ -- семейство подмножеств $X$

    $\P$ -- полукольцо, если 

    \begin{enumerate}
        \item $\varnothing \in \P$
        \item $\forall A, B \in \P \Rightarrow A \cap B \in \P$
        \item $\forall A, B \in \P\ \exists Q_1 \ldots Q_m \in \P$, т.ч. $A \setminus B = \bigsqcup\limits_{k = 1}^m Q_k$
    \end{enumerate}
\end{defin}

\begin{Example}{}
    \begin{enumerate}
        \item $X = \R;\ \P := \{(a, b] : a, b \in \R\}$ -- полукольцо
        \item $X = \R;\ \P := \{(a, b] : a, b \in \Q\}$ -- полукольцо
    \end{enumerate}
\end{Example}

\begin{lem}{}
    $\bigcup\limits_{k = 1}^n A_k = \bigsqcup\limits_{k = 1}^n \underbrace{(A_k \setminus \bigcup\limits_{j = 1}^{k - 1} A_j)}_{B_k}$ (для $\infty$ вместо $n$ тоже верно)
\end{lem}

\textit{Доказательство:}

\begin{itemize}
    \item $B_k \subset A_k \Rightarrow\ \supset$ верно
    \item $\subset$ возьмем $x \in \bigcup\limits_{k = 1}^n A_k \Rightarrow$ найдется наименьший индекс $m$, т.ч. $x \in A_m$ и $x \notin A_{m - 1} \ldots A_1 \Rightarrow \\ \Rightarrow x \in B_m$
    \item Дизъюнктность $k < m \Rightarrow B_k \cap B_m = \varnothing$

    $B_m = A_m \setminus \bigcup\limits_{j = 1}^{m - 1} A_j \subset A_m \setminus A_k \subset A_m \setminus B_k$
    
    $B_k \subset A_k$
\end{itemize}

\begin{theo}{}
    $\P$ -- полукольцо. Тогда

    \begin{enumerate}
        \item $P, P_1 \ldots P_n \in \P \Rightarrow \exists Q_1 \ldots Q_m \in \P$, т.ч. $P \setminus \bigcup\limits_{k = 1}^n P_k = \bigsqcup\limits_{j = 1}^m Q_j$
        \item $P_1, P_2 \ldots \in \P \Rightarrow \exists Q_{ij} \in \P$, т.ч. $\bigcup\limits_{k = 1}^n P_k = \bigsqcup\limits_{k = 1}^n\bigsqcup\limits_{j = 1}^{m_k} Q_{kj}$, где $Q_{kj} \subset P_k \forall\ k, j$
        \item В п. 2 можно вместо $n$ написать $\infty$
    \end{enumerate}
\end{theo}

\textit{Доказательство:}

\begin{enumerate}
    \item Индукция. База $n = 1$ -- определение полукольца
    
    Переход $n \to n + 1$

    $P \setminus \bigcup\limits_{k = 1}^{n + 1}P_k = \underbrace{(P \setminus \bigcup\limits_{k = 1}^n P_k)}_\text{инд. предполож.} \setminus P_{n + 1} = \underbrace{(\bigsqcup\limits_{j = 1}^m Q_j)}_\text{где $Q_j \in \P$} \setminus P_{n + 1} = \bigsqcup\limits_{j = 1}^m Q_j \setminus P_{n + 1} = \bigsqcup\limits_{j = 1}^m\bigsqcup\limits_{i = 1}^{m_j} Q_{ji}$

    \item $\bigcup\limits_{k = 1}^n P_k = \bigsqcup\limits_{k = 1}^n \underbrace{(P_k \setminus \bigcup\limits_{j = 1}^{k - 1} P_j)}_\text{п. 1}$
\end{enumerate}

\begin{defin}{}
    $\P$ -- полукольцо подмножеств $X$

    $\QQ$ -- полукольцо подмножеств $Y$

    $\P \times \QQ := \{A \times B : A \in \P \text{ и } B \in \QQ\}$ -- декартово произведение полуколец $\P$ и $\QQ$
\end{defin}

\newpage

\begin{theo}{}
    Декартово произведение полуколец -- полукольцо
\end{theo}

\textit{Доказательство:}

\begin{enumerate}
    \item Пустые очев
    \item $C \times D$ и $A \times B \in \P \times \QQ \Rightarrow (A \times B) \cap (C \times D) = \underbrace{(A \cap C)}_{\in \P} \times \underbrace{(B \cap D)}_{\in \QQ}$
    \item $A \times B, C \times D \in \P \times \QQ \xRightarrow[]{?} (A \times B) \setminus (C \times D) = \bigsqcup\limits_{k = 1}^m \underbrace{P_k}_{\in \P} \times \underbrace{Q_k}_{\in \QQ}$
    
    $(A \times B) \setminus (C \times D) = \underbrace{(A \setminus C)}_{\bigsqcup\limits_{j = 1}^m P_j} \times \underbrace{B}_{\in \QQ} \sqcup \underbrace{(A \cap C)}_{\in \P} \times \underbrace{(B \setminus D)}_{\bigsqcup\limits_{i = 1}^n Q_i}$
\end{enumerate}

\begin{defin}{Замкнутый и открытый параллелепипеды}
    $a, b \in \R^n$

    Замкнутый параллелепипед $[a, b] := [a_1, b_1] \times \ldots \times [a_n, b_n]$

    Открытый параллелепипед $(a, b) := (a_1, b_1) \times \ldots \times (a_n, b_n)$
\end{defin}

\begin{defin}{Ячейка}
    $a, b \in \R^n$

    Ячейка $(a, b] := (a_1, b_1] \times \ldots \times (a_n, b_n]$
\end{defin}

\begin{Remark}{}
    $(a, b) \subset (a, b] \subset [a, b]$
\end{Remark}

\begin{propos}{}
    \begin{enumerate}
        \item Непустая ячейка -- объединение возрастающей (по включению) последовательности замкнутых параллелепипедов
        \item Непустая ячейка -- пересечение убывающей (по включению) последовательности открытых параллелепипедов
    \end{enumerate}
\end{propos}

\textit{Доказательство:}

\begin{enumerate}
    \item $A_k := [a_1 - \frac{1}{k}, b_1] \times [a_2 - \frac{1}{k}, b_2] \times \ldots \times [a_n - \frac{1}{k}, b_n]$
    
    $A_{k + 1} \supset A_k$ и $\bigcup\limits_{k = 1}^\infty A_k = (a, b]$

    \item $B_k := (a_1, b_1 + \frac{1}{k}) \times (a_2, b_2 + \frac{1}{k}) \times \ldots \times (a_n, b_n + \frac{1}{k})$
    
    $B_{k + 1} \subset B_k$ и $\bigcap\limits_{k = 1}^\infty B_k = (a, b]$
\end{enumerate}

\begin{declar}{Обозначения}
    $\P^n := \{(a, b] : a, b \in \R^n\}$

    $\P_\Q^n := \{(a, b] : a, b \in \Q^n\}$
\end{declar}

\begin{propos}{}
    $\P^n$ и $\P^n_\Q$ -- полукольца
\end{propos}

\textit{Доказательство:}

$\P^n = \underbrace{\P^1 \times \P^1 \times \ldots \times \P^1}_\text{полукольца}$

\begin{theo}{}
    $G$ -- непустое открытое множество в $\R^m$

    Тогда $G$ представимо в виде счетного дизъюнктного объединения ячеек с рациональными координатами вершин
\end{theo}

\textit{Доказательство:}

У АИ тут рисуночки, посмотрите запись!

Для $x \in G$ построим ячейку $P_x$ с рациональными координатами вершин, т.ч. $P_x \in G$ и $x \in P_x$

$\bigcup\limits_{x \in G} P_x = G$

Ячеек с рациональными координатами вершин счетное число. Значит если выкинуть повторы из объединения выше, то останется счетное объединение

$G = \bigcup\limits_{n = 1}^\infty P_{x_n} = \bigsqcup\limits_{n = 1}^\infty \bigsqcup\limits_{j = 1}^{m_n} Q_{nj}$ -- ячейки с рациональными координатами вершин

\begin{theo}{Следствие}
    $\B^m = \B(\P^m) = \B(\P^m_\Q)$
\end{theo}

\textit{Доказательство:}

\begin{enumerate}
    \item $\B^m \supset \B(\P^m)$. Достаточно доказать, что $\B^m \supset \P^m$
    
    $(a, b]$ -- счетное пересечение открытых параллелепипедов (т.к. открытых множеств) $\Rightarrow (a, b]$ лежит в $\sigma$-алгебре, содержащей все открытые множества

    \item $\B(\P^m) \supset \B(\P^m_\Q)$. Достаточно доказать, что $\B(\P^m) \supset \P^m_\Q$, но $\B(\P^m) \supset \P^m \supset \P^m_\Q$
    
    \item $\B(\P^m_\Q) \supset \B^m$. Достаточно доказать, что $\B(\P^m_\Q)$ содержит все открытые множества. Это следует из теоремы 1.5.
\end{enumerate}

\newpage

\subsection{§2. Объем и мера}

\begin{defin}{Объем}
    $\P$ -- полукольцо. $\mu : \P \to [0, + \infty]$

    $\mu$ -- объем, если 

    \begin{enumerate}
        \item $\mu \varnothing = 0$
        \item Если $A_1, \ldots A_n$ и $\bigsqcup\limits_{k = 1}^n A_k \in \P$, то $\mu(\bigsqcup\limits_{k = 1}^n A_k) = \sum\limits_{k = 1}^n \mu A_k$
    \end{enumerate}
\end{defin}

\begin{defin}{Мера}
    $\P$ -- полукольцо. $\mu : \P \to [0, + \infty]$

    $\mu$ -- мера, если 

    \begin{enumerate}
        \item $\mu \varnothing = 0$
        \item Если $A_1, A_2 \ldots$ и $\bigsqcup\limits_{k = 1}^\infty A_k$, то $\mu(\bigsqcup\limits_{k = 1}^\infty A_k) = \sum\limits_{k = 1}^\infty \mu A_k$
    \end{enumerate}
\end{defin}

\begin{Exercise}{}
    Если $\mu \varnothing \neq +\infty$, то $\mu \varnothing = 0$ из свойства 2
\end{Exercise}

\begin{Example}{Примеры объемов}
    \begin{enumerate}
        \item $X = \R,\ \P^1$. Длина -- объем. $\mu(a, b] = b - a$
        \item $X = \R,\ \P^1$. $g : \R \to \R$ -- нестрого возрастающая функция
        
        $\nu_g(a, b] := g(b) - g(a)$

        \item Классический объем на $\P^m$
        
        $\lambda_m(a, b] = (b_1 - a_1)(b_2 - a_2) \ldots (b_m - a_m)$ -- объем и даже мера (докажем позже)

        \item $x_0 \in X;\ \mu A = \begin{cases}
            0 & x_0 \notin A \\
            1 & x_0 \in A
        \end{cases}$

        \item $X = \R^2;\ \P$ -- ограниченные множества и их дополнения
        
        $\mu A = \begin{cases}
            0 & A \text{ -- ограничена} \\
            1 & A \text{ дополнение ограничено}
        \end{cases}$ -- объем, но не мера
    \end{enumerate}
\end{Example}

\begin{theo}{Свойства объема}
    $\P$ -- полукольцо, $\mu$ -- объем на $\P$. Тогда 

    \begin{enumerate}
        \item Монотонность
        
        $P, \tilde{P} \in \P$ и $P \subset \tilde{P} \Rightarrow \mu P \leq \mu \tilde{P}$

        \item Усиленная монотонность 
        
        $P_1, P_2 \ldots P_n, \tilde{P} \in \P$ и $\bigsqcup\limits_{k = 1}^n P_k \subset \tilde{P} \Rightarrow \sum\limits_{k = 1}^n \mu P_k \leq \mu \tilde{P}$

        \item[2'.] $P_1, P_2 \ldots, \tilde{P} \in \P$ и $\bigsqcup\limits_{k = 1}^\infty P \subset \tilde{P} \Rightarrow \sum\limits_{k = 1}^\infty \mu P_k \leq \mu \tilde{P}$
        
        \item[3.] Конечная полуаддитивность
        
        $P_1 \ldots P_n, P \in \P$ и $P \subset \bigcup\limits_{k = 1}^n P_k \Rightarrow \mu P \leq \sum\limits_{k = 1}^n \mu P_k$
    \end{enumerate}
\end{theo}

\textit{Доказательство:}

\begin{enumerate}
    \item[2.] $\tilde{P} \setminus \bigsqcup\limits_{k = 1}^n P_k = \bigsqcup\limits_{j = 1}^m Q_j$, где $Q_j \in \P$
    
    $\tilde{P} = \bigsqcup\limits_{k = 1}^n P_k \sqcup \bigsqcup\limits_{j = 1}^m Q_j \Rightarrow \mu \tilde{P} = \sum\limits_{k = 1}^n \mu P_k + \underbrace{\sum\limits_{j = 1}^m}_{\geq 0} \mu Q_j \geq \sum\limits_{k = 1}^n \mu P_k$

    \item[2'.] Предельный переход в неравенстве
    
    \item[3.] $P'_k := P_k \cap P \in \P \Rightarrow P = \bigcup\limits_{k = 1}^n P_k' \underbrace{=}_\text{th.} \bigsqcup\limits_{k = 1}^n \bigsqcup\limits_{j = 1}^{m_k} Q_{kj}$ (они из $\P$) $\Rightarrow \mu P = \sum\limits_{k = 1}^n \sum\limits_{j = 1}^{m_k} \mu Q_{kj}$
    
    $P_k \supset P_k' \supset \bigsqcup\limits_{j = 1}^{m_k} Q_{kj} \Rightarrow \mu P_k \geq \sum\limits_{j = 1}^{m_k} \mu Q_{kj}$
\end{enumerate}

\end{document}