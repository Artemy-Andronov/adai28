\documentclass[12pt]{article}
\usepackage{config}
\usepackage{subfiles}
\pgfplotsset{compat=1.18}

\begin{document}

\begin{flushright}
    Конспект Шорохова Сергея

    Если нашли опечатку/ошибку - пишите @le9endwp 
\end{flushright}

\tableofcontents
\newpage

\section{\S 1. Производящие функции}

\begin{defin}{Производящая функция}
    Пусть $(a_n)_{n = 0}^\infty$ -- последовательность. Ее производящая функция -- это формальный степенной ряд $A(t) = \sum\limits_{n = 0}^\infty a_n t^n = a_0 + a_1t + a_2t^2 + \ldots$
\end{defin}

\begin{nota}{Элементарные операции}
    \begin{enumerate}
        \item $A(t) \pm B(t) = \sum\limits_{n = 0}^\infty (a_n \pm b_n) t^n$
        \item $c \in \C \Rightarrow c \cdot A(t) = \sum\limits_{n = 0}^\infty (ca_n)t^n$
        \item $A(t)B(t) = a_0b_0 + (a_0b_1 + a_1b_0)t + \ldots + (a_0b_n + a_1b_{n - 1} + a_2b_{n - 2} + \ldots + a_nb_0)t^n + \ldots$
    \end{enumerate}
\end{nota}

\begin{defin}{Свертка}
    Последовательность $(c_n)_{n = 0}^\infty$, где $c_n = a_0b_n + a_1b_{n - 1} + \ldots + a_{n - 1}b_1 + a_nb_0$ называется сверткой последовательностей $(a_n)_{n = 0}^\infty$ и $(b_n)_{n = 0}^\infty$
\end{defin}

\begin{Remark}{}
    Множество производящих функций образует коммутативное кольцо с единицей; векторное пространство над полем $\C$

    Вообще это называется коммутативная алгебра с единицей
\end{Remark}

\begin{defin}{Композиция производящих функций}
    Пусть $b_0 = 0$

    $A(B(t)) = a_0 + a_1B(t) + a_2B(t)^2 + \ldots = a_0 + a_1(b_1t + b_2t^2 + b_3t^3 + \ldots) + \\
    + a_2(b_1^2t^2 + 2b_1b_2t^3 + \ldots) + a_3(b_1^3t^3 + \ldots) = a_0 + a_1b_1t + (a_1b_2 + a_2b_1)t^2 + (a_1b_3 + 2a_2b_1b_2 + a_3b_1^3)t^3$

    \begin{Example}{}
        $A(-t) = a_0 - a_1t + a_2t^2 - a_3t^3 + \ldots$
    \end{Example}
\end{defin}

\newpage

\begin{theo}{}
    Пусть $a_0 \neq 0$. Тогда $\exists! B(t)$, т.ч. $A(t) B(t) = 1$
\end{theo}

\textit{Доказательство:}

Ищем $B(t) = b_0 + b_1t + b_2t^2 + \ldots$

$A(t)B(t) = a_0b_0 + (a_0b_1 + a_1b_0)t + (a_2b_0 + a_1b_1 + a_0b_2)t^2 + \ldots = 1$

$a_0b_0 = 1 \Rightarrow$ находим $b_0$

$\underbrace{a_1b_0}_\text{знаем} + a_0b_1 = 0 \Rightarrow$ находим $b_1$

$\underbrace{a_2b_0 + a_1b_1}_\text{знаем} + a_0b_1 = 0 \Rightarrow$ находим $b_2$

И так далее \dots

\begin{theo}{}
    $b_0 = 0, b_1 \neq 0$. Тогда $\exists! A(t)$ и $C(t)$, т.ч. $a_0 = c_0 = 0$ и $A(B(t)) = B(C(t)) = t$
\end{theo}

\begin{Exercise}{}
    Доказать теорему 1.2.
\end{Exercise}

\begin{defin}{Производная}
    $A'(t) = a_1 + 2a_2t + 3a_3t^2 + \ldots = \sum\limits_{n = 1}^\infty na_nt^{n - 1}$

    $t \cdot A'(t) = \sum\limits_{n = 0}^\infty na_nt^n$
\end{defin}

\begin{defin}{Первообразная}
    $\int A(t)dt = a_0t + \frac{a_1}{2}t^2 + \frac{a_2}{3}t^3 + \ldots$
\end{defin}

\begin{Remark}{}
    $(\int A(t)dt)' = A(t);\ \int A'(t)dt = A(t) - a_0$
\end{Remark}

\begin{Example}{}
    \begin{enumerate}
        \item $a_n \equiv 1 \Rightarrow A(t) = \sum\limits_{n = 0}^\infty t^n = \frac{1}{1 - t}$
        
        Пусть $(b_n)_{n = 0}^\infty$ -- произвольная последовательность

        $c_n = \underbrace{b_0 + b_1 + \ldots + b_n}_\text{свертка послед. выше}$; $C(t) = \frac{B(t)}{1 - t}$

        \item $e^t = \sum\limits_{k = 0}^\infty \frac{t^k}{k!}$
        \item $\sum\limits_{n = 0}^\infty t^n = \frac{1}{1 - t};\ \sum\limits_{n = 0}^\infty nt^n = t(\frac{1}{1 - t})' = \frac{t}{(1 - t)^2}$
    \end{enumerate}
\end{Example}

\begin{Example}{Числа Фиббоначи}
    $F_0 = 0,\ F_1 = 1,\ F_{n + 2} = F_{n + 1} + F_n \Rightarrow F_{n + 2}t^{n + 2} = F_{n + 1}t^{n + 2} + F_n t^{n + 2}$

    $F(t) = \sum\limits_{n = 0}^\infty F_n t^n;\ \underbrace{\sum\limits_{n = 0}^\infty F_{n + 2}t^{n + 2}}_{F(t) - t} = \underbrace{\sum\limits_{n = 0}^\infty F_{n + 1}t^{n + 2}}_{t (\sum\limits_{n = 0}^\infty F_{n + 1}t^{n + 1}) = tF(t)} + \underbrace{\sum\limits_{n = 0}^\infty F_nt^{n + 2}}_{t^2F(t)}$

    $F(t) - t = tF(t) + t^2F(t)$

    $F(t) = \frac{t}{1 - t - t^2}$ -- производящая функция для чисел Фиббоначи

    Корни знаменателя $(t_2 + t - 1 = 0 \Leftrightarrow t = \frac{-1 \pm \sqrt{5}}{2});\ \varphi = \frac{1 + \sqrt{5}}{2};\ \psi = \frac{1 - \sqrt{5}}{2}$

    $1 - t - t^2 = (1 - \varphi t)(1 - \psi t)$

    Ищем разложение на простейшие $\frac{t}{1 - t - t^2} = \frac{A}{1 - \varphi t} + \frac{B}{1 - \psi t} \Leftrightarrow \\ 
    \Leftrightarrow t = A(1 - \psi t) + B(1 - \varphi t) \Leftrightarrow \begin{cases}
        A + B = 0 \\
        A \psi + B \varphi = -1
    \end{cases} \Leftrightarrow  \\ \Leftrightarrow \begin{cases}
        B = -A \\
        A \psi - A \varphi = - 1 \Rightarrow  \begin{cases}
            A = \frac{1}{\varphi - \psi} = \frac{1}{\sqrt{5}} \\
            B = - \frac{1}{\sqrt{5}}
        \end{cases}
    \end{cases}$

    Итого: $F(t) = \frac{1}{\sqrt{5}}(\frac{1}{1 - \varphi t} - \frac{1}{1 - \psi t}) = \frac{1}{\sqrt{5}}(\sum\limits_{n = 0}^\infty \varphi^nt^n - \sum\limits_{n = 0}^\infty \psi^nt^n)$

    $F_n = \frac{1}{\sqrt{5}}((\frac{1 + \sqrt{5}}{2})^n - (\frac{1 - \sqrt{5}}{2})^n) \approx \frac{1}{\sqrt{5}}(\frac{1 + \sqrt{5}}{2})^n$
\end{Example}

\begin{nota}{Как решать линейные рекуррентные соотношения?}
    $a_{n + k} = c_1a_{n + k - 1} + c_2a_{n + k - 2} + \ldots + c_ka_n$; знаем $a_0, a_1 \ldots a_{k - 1}$

    $A(t) = \sum\limits_{n = 0}^\infty a_nt^n$

    $a_{n + k}t^{n + k} = c_1ta_{n + k - 1}t^{n + k - 1} + c_2t^2a_{n + k - 2}t^{n + k - 2} + \ldots + c_kt^ka_nt^n$

    Суммируем по $n = 0 : \underbrace{\sum\limits_{n = 0}^\infty a_{n + k}t^{n + k}}_{A(t) - a_0 - a_1t - \ldots - a_{k - 1}t^{k - 1}} = c_1t\underbrace{\sum\limits_{n = 0}^\infty a_{n + k - 1}t^{n + k - 1}}_{A(t) - a_0 - a_1t - \ldots - a_{k - 2}t^{k - 2}} + \ldots + c_kt^k\underbrace{\sum\limits_{n = 0}^\infty a_nt^n}_{A(t)}$

    Получаем уравнение: $(1 - c_1t - c_2t^2 - \ldots - c_kt^k)A(t) = \underbrace{P(t)}_\text{многочлен, знаем}$

    $A(t) = \frac{P(t)}{Q(t)}$ -- рациональная функция 

    $Q(t) = (1 - \alpha_1t)^{r_1} (1 - \alpha_2t)^{r_2} \ldots (1 - \alpha_et)^{r_e}$

    Раскладываем на простейшие вида $\frac{1}{(1 - \alpha_s t)^{m}}$

    $\frac{1}{1 - \alpha_s t} = \sum\limits_{n = 0}^\infty \alpha_s^nt^n$

    $\frac{1}{(1 - \alpha_st)^2} = \sum\limits_{n = 0}^\infty (n + 1)\alpha_s^nt^n$
\end{nota}

\begin{Remark}{Вопрос}
    Когда производящая функция -- рациональная?
\end{Remark}

\begin{defin}{Квазимногочлен}
    Последовательность $(a_n)_{n = 0}^\infty$ -- квазимногочлен, если $a_n = c_1(n)q_1^n + c_2(n)q_2 + \ldots + c_k(n)q_k^n$, где $q_1 \ldots q_k \in \C;\ c_1(n) \ldots c_k(n)$ -- многочлены с комплексными коэффициентами 
\end{defin}

\begin{theo}{}
    $A(t) = \sum\limits_{n = 0}^\infty a_nt^n;\ A(t)$ -- рациональна $\Leftrightarrow (a_n)_{n = 0}^\infty$ -- квазимногочлен, начиная с некоторого места
\end{theo}

\textit{Доказательство:}

\begin{itemize}
    \item["$\Rightarrow$"\ ] $A(t)$ -- рациональная $\Rightarrow$ раскладываем на простейшие вида $(1 - qt)^{-m} + \underbrace{\text{ некоторый многочлен}}_\text{влияет на первые неск. эл. посл-ти}$
    
    $(1 - qt)^{-m} = \sum\limits_{n = 0}^\infty {m + n - 1 \choose n} q^nt^n = \sum\limits_{n = 0}^\infty \underbrace{\frac{(n + 1)(n + 2)\ldots(n + m - 1)}{(m - 1)!}}_\text{многочлен от $n$}q^nt^n$

    \item["$\Leftarrow$"\ ] Надо доказать, что $(c(n)q^n)_{n = 0}^\infty$ имеет рациональную производящую функцию
    
    $c(n) = \sum\limits_{m \geq 0} \alpha_m n(n + 1)\ldots(n + m) = \alpha_0 + \alpha_1(n + 1) + \alpha_2(n + 1)(n + 2) + \ldots$

    $\sum\limits_{n = 0}^\infty c(n)q^nt^n = \sum\limits_{n = 0}^\infty \sum\limits_{m \geq 0} \alpha_m(n + 1)(n + 2) \ldots (n + m) \cdot (qt)^n \stackrel{x = qt}{=} \sum\limits_{m \geq 0} \alpha_m \sum\limits_{n = 0}^\infty \underbrace{(n + 1)(n + 2) \ldots (n + m) x^n}_{(x^{n + m})^{(m)}} = \\
     = \sum\limits_{m \geq 0} \alpha_m \cdot (\sum\limits_{n = 0}^\infty x^{n + m})^{(m)} = \sum\limits_{m \geq 0} \alpha_m (\sum\limits_{n = 0}^\infty x^n)^{(m)} = \sum\limits_{m \geq 0} \alpha_m (\frac{1}{1 - x})^{(m)}$

     Получаем рациональную функцию 
\end{itemize}

\begin{defin}{Произведение Адамара}
    $A(t) = \sum\limits_{n = 0}^\infty a_nt^n;\ B(t) = \sum\limits_{n = 0}^\infty b_nt^n$

    Произведение Адамара $A(t) \odot B(t) = \sum\limits_{n = 0}^\infty (a_nb_n)t^n$
\end{defin}

\begin{theo}{Следствие}
    Произведение Адамара рациональных функций -- рациональная функция (очевидно из теоремы)
\end{theo}

\begin{Example}{}
    $F_1 + \ldots F_n = S_n =\ ?$

    $\F(t) = \sum\limits_{n = 0}^\infty F_nt^n = \underbrace{\frac{t}{1 - t - t^2}}_{\frac{1}{\sqrt{5}}(\frac{1}{1 - \varphi 1} - \frac{1}{1 - \psi t})} = \sum\limits_{n = 0}^\infty S_nt^n = \frac{\F(t)}{1 - t}$

    $S(t) = \frac{1}{\sqrt{5}}(\frac{1}{1 - \varphi t} - \frac{1}{1 - \psi t}) \frac{1}{1 - t}$

    Разложим $\frac{1}{1 - \varphi t} \cdot \frac{1}{1 - t} = \frac{A}{1 - \varphi t} + \frac{B}{1 - t} \Leftrightarrow 1 = A(1 - t) + B(1 - \varphi t) \Leftrightarrow \begin{cases}
        B = \frac{1}{1 - \varphi} = - \varphi \\
        A = 1 + \varphi
    \end{cases}$

    Аналогично $\frac{1}{1 - \psi t} \cdot \frac{1}{1 - t} = \frac{1 + \psi}{1 - \psi t} - \frac{\psi}{1 - t}$

    Итого, $S(t) = \frac{1}{\sqrt{5}}(\underbrace{\frac{1 + \varphi}{1 - \varphi t} - \frac{1 + \psi}{1 - \psi t}}_{\frac{1 + \varphi - \psi t - \varphi \psi t - 1 - \psi + \varphi t + \varphi \psi t}{1 - t - t^2}} - \underbrace{\frac{\varphi - \psi}{1 - t}}_{\frac{\sqrt{5}}{1 - t}})$

    $S(t) = \frac{1 + t}{1 - t - t^2} - \frac{1}{1 - t}$

    $\frac{t}{1 - t - t^2} = \sum\limits_{n = 0}^\infty F_nt^n = \F(t)$

    $\frac{1}{1 - t - t^2} = \frac{\F(t)}{t} = \sum\limits_{n = 1}^\infty F_nt^{n - 1} = \sum\limits_{ n = 0}^\infty F_{n + 1}t^n$

    Ответ: $F_{n + 2} - 1$
\end{Example}

\begin{Example}{Еще один пример}
    \textbf{Осторожно! На записи рисуночки}

    Взаимно рекуррентные последовательности

    \textbf{Задача:} сколько способов разбить прямоугольник $3 \times n$ на доминошки $1 \times 2$?

    $v_n$ -- кол-во способов разбить прямоугольник $3 \times n$ без левой нижней клетки

    $u_n$ -- кол-во способов разбить прямоугольник $3 \times n$

    Методом нехитрого посмотреть запись и увидеть красивые рисунки становится очевидно, что 

    $\begin{cases}
        u_n = 2v_{n - 1} + u_{n - 2} \\
        v_n = u_{n - 1} + v_{n - 2}
    \end{cases}$ при $u_1 = 0, u_2 = F_4 = 3;\ v_1 = 1, v_2 = 0$. Пусть $u_0 = 1;\ v_0 = 0$

    $U(t) = \sum\limits_{n = 0}^\infty u_nt^n;\ V(t) = \sum\limits_{n = 0}^\infty v_nt^n$

    $\begin{cases}
        u_{n + 2}t^{n + 2} = 2v_{n + 1}t^{n + 2} + u_nt^{n + 2} \\
        v_{n + 2}t^{n + 2} = u_{n + 1}t^{n + 2} + v_nt^{n + 2}
    \end{cases} \Rightarrow \begin{cases}
        U(t) - 1 = 2tV(t) + t^2U(t) \\
        V(t) - t = t^2V(t) + t(U(t) - 1)
    \end{cases}$

    $V(t) = \frac{t}{1 - t^2}U(t)$. Подставляем во 2 уравнение

    $U(t) - 1 = \frac{2t^2}{1 - t^2}U(t) + t^2U(t)$

    $U(t) = \frac{1 - t^2}{1 - 4t^2 + t^4}$

    Пусть $t^2 = s$, тогда $W(s) = \frac{1 - s}{1 - 4s + s^2} = \underbrace{\frac{A}{1 - \varphi s}}_{A \cdot \sum\limits_{n = 0}^\infty \varphi^ns^n} + \underbrace{\frac{B}{1 - \psi s}}_{B \cdot \sum\limits_{n = 0}^\infty \psi^ns^n} \Rightarrow \begin{cases}
        A = \frac{1 + \sqrt{3}}{2\sqrt{3}} \\
        B = \frac{\sqrt{3} - 1}{2\sqrt{3}}
    \end{cases}$

    $u_{2n} = A \varphi^n + B \psi^n = \frac{1 + \sqrt{3}}{2\sqrt{3}}(2 + \sqrt{3})^n + \frac{\sqrt{3} - 1}{2\sqrt{3}}(2 - \sqrt{3})^n \approx \frac{1 + \sqrt{3}}{2\sqrt{3}}(2 + \sqrt{3})^n$
\end{Example}

\end{document}