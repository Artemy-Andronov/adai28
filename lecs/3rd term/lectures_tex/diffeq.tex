\documentclass[12pt]{article}
\usepackage{config}
\usepackage{subfiles}
\pgfplotsset{compat=1.18}

\begin{document}

\begin{flushright}
    Конспект Шорохова Сергея

    Если нашли опечатку/ошибку - пишите @le9endwp 
\end{flushright}

\tableofcontents
\newpage

\section{Оргинфа}

Ведет Крыжевич Сергей Геннадьевич

+79219181076 и +48572768176

kryzhevicz@gmail.com и serkryzh@pg.edu.pl

\section{Дифференциальные уравнения первого порядка}

\begin{defin}{Дифференциальные уравнения первого порядка}
    $D \subset \R^2$ -- область, $f : D \to \R$ -- непрерывная функция

    Дифференциальные уравнения первого порядка -- это уравнения вида $y' = f(x, y)$
\end{defin}

\begin{Example}{}
    $y' = xy$
\end{Example}

\begin{defin}{Решение дифференциального уравнения}
    $\q{a, b}$ -- интервал

    Функция $\varphi(x)$ -- решение дифференциального уравнения на $\q{a, b}$, если 

    \begin{enumerate}
        \item $\varphi, \varphi'$ -- непрерывны на $\q{a, b}$
        \item $(x, \varphi(x)) \in D\ \forall x \in \q{a, b}$
        \item $\varphi'(x) = f(x, \varphi(x))$
    \end{enumerate}
\end{defin}

\begin{Example}{}
    $y' = xy$

    Решениями будут:

    \begin{enumerate}
        \item $y = 0$
        \item $y = e^{\frac{x^2}{2}}$
        
        $y' = xe^{\frac{x^2}{2}} = xy$
    \end{enumerate}

    На самом деле решением будет любая функция вида $y = Ce^{\frac{x^2}{2}}$
\end{Example}

\begin{nota}{Начальные данные для дифференциального уравнения}
    $\begin{cases}
        y' = f(x, y) \\
        y(x_0) = y_0
    \end{cases}$
\end{nota}

\begin{defin}{Задача Коши}
    Задача Коши -- дифференциальное уравнение с начальными данными
\end{defin}

\begin{Example}{}
    $\begin{cases}
        y' = xy \\
        y(0) = 5
    \end{cases}$

    $y = Ce^{\frac{x^2}{2}}$

    $5 = Ce^0 = C$

    Получаем ответ $y = 5e^{\frac{x^2}{2}}$
\end{Example}

\begin{defin}{Общее решение дифференциального уравнения}
    Общее решение дифференциального уравнения -- совокупность всех его решений (= решение с параметром)
\end{defin}

\begin{defin}{Интегральная кривая}
    Интегральная кривая -- график решения дифференциального уравнения, т.е. график $\{x, \varphi(x)\}$
\end{defin}

\begin{Remark}{}
    $y' = \sqrt{y};\ y \geq 0$

    Здесь множество не является открытым, но считается, что $y = 0$ является решением (хотя формально им не является)

    Если в каких-то задачах такое будет, в рамках курса не считаем это ошибкой
\end{Remark}

\begin{Remark}{Единственность решений задачи Коши}
    Почти всегда задача Коши имеет единственное решение. Но есть исключения, например

    $\begin{cases}
        y' = 3y^{\frac{2}{3}} \\
        y(0) = 0
    \end{cases}$

    Очевидное решение $y = 0$, но также $y = x^3$. Более того, решением будет любая функция вида $y = (x + C)^3$. График есть на записи

    Более того, можно собрать решение покусочно (ветка параболки вниз + прямая $y = 0$ + ветка параболы вверх)
\end{Remark}

\begin{defin}{Точка единственности/ветвления}
    $\begin{cases}
        y' = f(x, y) \\
        y(x_0) = y_0
    \end{cases}$

    Точка $(x_0, y_0)$ -- точка единственности, если решение задачи Коши единственно. В противном случае это точка ветвления 
\end{defin}

\begin{defin}{Особое решение}
    Решение называется особым, если любая его точка -- точка ветвления
\end{defin}

\begin{theo}{}
    Если в уравнении $y' = f(x, y)$ функция $f$ непрерывна и имеет непрерывную производную по переменной $y$ в области $D$, то для любой точки $(x_0, y_0)$ из $D$ решение задачи Коши с начальными данными $y(x_0) = y_0$ существует и единственно
\end{theo}

\begin{Remark}{}
    По $x$ нужна только непрерывность, производной существовать не обязательно
\end{Remark}

\begin{defin}{Дифференциальные уравнения в симметричной форме}
    $P(x, y)dx + Q(x, y)dy = 0$
\end{defin}

\begin{Example}{}
    $ydx - xdy = 0 \mapsto y' = \frac{y}{x}$ или $x' = \frac{x}{y}$
\end{Example}

\begin{Remark}{}
    Предполагаем, что $P$ и $Q$ -- функции, непрерывные в некоторой области $D$ на плоскости и они не обращаются в ноль одновременно ни в одной точке $D$
\end{Remark}

\begin{defin}{Решение уравнения в симметричной форме}
    \begin{enumerate}
        \item $y' = -\frac{P(x, y)}{Q(x, y)}$, решением будет $y = \varphi(x) : P(x, \varphi(x)) + Q(x, \varphi(x))\varphi'(x) = 0$
        \item $x' = -\frac{Q(x, y)}{P(x, y)}$, решением будет $x = \psi(y) : P(\psi(y), y)\psi'(y) + Q(\psi(y), y) = 0$
        \item $y = \varphi(t), x = \psi(t)$, хотим $P(\psi(t), \varphi(t))\psi'(t) + Q(\psi(t), \varphi(t))\varphi'(t) = 0$
    \end{enumerate}
\end{defin}

\end{document}

