\documentclass[12pt]{article}
\usepackage{config}
\usepackage{subfiles}
\pgfplotsset{compat=1.18}

\begin{document}

\tableofcontents
\newpage

\begin{flushright}
    Конспект Шорохова Сергея

    Если нашли опечатку/ошибку - пишите @le9endwp
\end{flushright}

\section{План на 3 модуль (или 2 сем...)}

\begin{enumerate}
    \item Множества
    \item ЧУМ
    \item Исчисление высказываний
    \item Исчисление предикатов
    \item Теория кодирования
\end{enumerate}

Почитать можно А. Х. Шеня

\section{Множества}

\begin{enumerate}
    \item $x \in A;\ y \not \in A$
    \item Арифметика множеств: $\bigcup, \bigcap, \backslash, \triangle$
    \item $\varnothing$
    \item $A = \{ a, b, c \};\ B = \{d\} \bigcup A$
    \item $A \subset B \Leftrightarrow \forall x \in A \Rightarrow x \in B$
\end{enumerate}

\begin{Remark}{}
    Чисто синтаксически вот такой бред: $\{ \varnothing, \{ \varnothing \}, \{ \varnothing, \{ \varnothing \} \} \}$ имеет смысл
\end{Remark}

$X$ -- множество: $X \neq \varnothing$. Рассмотрим $x \in X$

$Term(x)$ -- проблема, потому что мы не знаем, к каким характеристикам обращаемся и вообще не понятно, что мы выбрали

Спасают аксиомы ZFC

\begin{defin}{Равномощность}
    $A, B$ -- равномощны $\Leftrightarrow \exists f : A \rightarrow B$ -- биекция

    А что с бесконечностями? Давайте возьмем функцию $f : N \rightarrow 2N$

    Хотя множество четных чисел -- подмножество всех, но они равномощны, т.к. $f$ -- биекция
\end{defin}

\begin{defin}{Характеристическая функция}
    $X$ -- множество. Есть $\chi : X \rightarrow \{0, 1\}$, т.е. $\chi(x) = \begin{cases}
        1,\ x \in X \\
        0,\ x \not\in X
    \end{cases}$ -- характеристическая функция
\end{defin}

А пусть $X \subset Y$

\begin{itemize}
    \item произведение характеристических функций $X$ и $Y$ -- это характеристическая функция $X \bigcap Y$
    \item $1 - \chi(x)$ -- характеристическая функция дополнения $X$
    \item $max(\chi_X(x), \chi_Y(x))$ -- характеристическая функция $X \bigcup Y$
    \item $|X| = \sum\limits_{x \in Y} \chi_X(x)$
\end{itemize}

\begin{Example}{}
    Возьмем $2^N;\ B = \{ 0, 1 \}$ и $B^\infty$

    Равномощны ли они? Берем $x \in 2^N$, теперь $b_i = \begin{cases}
        1,\ i \in x \\
        0,\ i \not\in x
    \end{cases}$
\end{Example}

\begin{defin}{Счетное множество}
    $X$ -- счетное, если $X$ равномощно $N$
\end{defin}

\begin{Example}{}
    Например, множество целых чисел счетно, т.к. $x \in Z \Rightarrow \begin{cases}
        2x,\ x \geq 0 \\
        -2x + 1,\ x < 0
    \end{cases}$
\end{Example}

\begin{propos}{}
    \begin{enumerate}
        \item $X$ -- счетно и $Y \subset X \Rightarrow Y$ или счетно, или конечно
        \item $X$ -- бесконечно. Тогда $\exists Y$ -- счетное: $Y \subset X$
        \item $X_1, \ldots X_n \ldots$ -- конечные или счетные. Тогда $\bigcup X_i$ -- конечное или счетное
    \end{enumerate}
\end{propos}

\textit{Доказательство:}

\begin{enumerate}
    \item $X$ -- счетно, т.е. соответствует последовательности $\{x_1, \ldots x_n \ldots\} = \xi$
    
    Возьмем $\xi \cdot \chi(Y)$. Т.е. что-то типа $\{ 0, 0 \ldots x_{i_1}, 0 \ldots x_{i_2}, 0 \ldots \}$ который равносилен $y_1, y_2, \ldots y_n \ldots = Y$

    В свою очередь эта штука либо конечна, либо счетна, т.к. счетен $X$

    \item Просто выбираем по 1 элементу из $X$. Если они кончатся на каком-то шаге -- $X$ не бесконечно
    
    \item Рисуем табличку. Берем элемент (1, 1), потом (1, 2), потом (2, 1), потом (1, 3) и так далее. То есть по диагоналям. Так переберем вообще все элементы (если не понятно, погуглите метод Кантора)
\end{enumerate}

\begin{Exercise}{}
    В качестве следствия попробуйте построить явную биекцию между множеством рациональных чисел и натуральных
\end{Exercise}

\begin{theo}{}
    $A$ -- бесконечно, $B$ -- нбчс, т.е. $B$ -- конечно или счетно

    $A \bigcup B$ равномощно $A$
\end{theo}

\textit{Доказательство:}

$\exists Y \subset A$ -- счетное

$Y$ и $Y \bigcup B$ -- равномощны

$A \bigcup B = (A \backslash Y) \bigcup (Y \bigcup B)$

$A = Y \bigcup (A \backslash Y)$

Биекция между $Y$ и $Y \bigcup B$ сущесвтует, значит $A$ и $A \bigcup B$ равномощны

\begin{Example}{}
    $[0; 1]$ и $B^\infty$. Равномощны ли? Да. Последовательность единиц и нулей -- это бинпоиск числа

    Проблема: $0,(9) = 1,(0)$

    $b_1 \ldots b_k, 1, 1, 1, 1, (1)$

    $(b_1 \ldots b_k) + 1$

    $R \bigcup [0, 1] \sim B^{\infty}$ и $R \bigcup [0, 1] \sim [0, 1] \Rightarrow [0, 1] \sim B^{\infty}$
\end{Example}

\begin{Example}{}
    $[0, 1] \sim [0, 1] \times [0, 1]$

    $0, a_1a_2 \ldots a_k \ldots$

    $0, a_1a_3a_5 \ldots$ и $0, a_2a_4a_6 \ldots$

    \begin{Exercise}{}
        Проблема та же, что и в прошлом примере, но число уязвимых моментов кратно больше. Почините
    \end{Exercise}
\end{Example}

\begin{theo}{Кантор-Бернштейн}
    $A, B;\ A_1 \subset A;\ B_1 \subset B$

    $A_1 \sim B, B_1 \sim A \Rightarrow A \sim B$
\end{theo}

\textit{Доказательство:}

$A$ имеет мощность не больше $B$. Существует какое-то отображение. Нужна его биективность. А где-то по пути может докажем еще и полный порядок

$f: A \rightarrow B_1$ -- биекция

$g : B \rightarrow A_1$ -- еще одна биекция

Заметим, что $g(f(A)) = A_2$ -- биекция, более того этот процесс можно продолжить до бесконечности 

То есть имеем $A \supset A_1 \supset A_2 \ldots$ и $A \sim A_2 \sim A_4 \ldots$ и $A_1 \sim A_3 \sim A_5 \sim \ldots$

Возьмем просто много вложенных $C$-шек таких, что $C \rightarrow C_2 \rightarrow C_4 \ldots$ и $C_1 \rightarrow C_3 \ldots$ при какой-то биекции $h$

Как построить биекцию из $C_6$ в $C_7$? Положим $D_i = C_i \setminus C_{i + 1}$. Тогда $C_0 = D_0 \bigcup D_1 \bigcup D_2 \ldots$

При этом $C_1 = D_1 \bigcup D_2 \bigcup D_3 \ldots$

$D_2 = C_2 \setminus C_3;\ D_0 = C_0 \setminus C_1$. Ну тогда $C_2 = D_2 \bigcup C_3$ и $C_0 = D_0 \bigcup C_1$

При этом биекция $h$ все еще существует. Можем сопоставить $D_{2k} \rightarrow D_{2(k + 1)}$, а $D_{2k + 1} \rightarrow D_{2k + 1}$, т.е. построить биекцию между $C_0$ и $C_1$. Победа

Явная биекция: $q(x) = \begin{cases}
    x, x \in D_{2i + 1} \\
    h(x), x \in D_{2i}
\end{cases}$

\begin{theo}{Теорема Кантора}
    $B^{\inf}$ -- не счетно
\end{theo}

\textit{Доказательство:}

Построили последовательность типа 

\begin{enumerate}
    \item $a_1, a_2 \ldots$
    \item $b_1, b_2 \ldots$
    \item $c_1, c_2 \ldots$
\end{enumerate}

Ну возьмем еще одну последовательность $a_1, b_2, c_3 \ldots$ -- она будет отличаться от всех предыдущих как минимум в одном элементе. Значит $B^{\inf}$ не счетно

\begin{theo}{Обобщенная теорема Кантора}
    $\forall X,\ X \not\sim 2^X$
\end{theo}

\textit{Доказательство:}

Пусть $\exists \varphi : X \rightarrow 2^X$ -- биекция

$Z = \{ x | x \not\in \varphi(x) \}$

$Z \subset X$

$\not\exists z : \varphi(z) = Z \Rightarrow z \not\in Z \Rightarrow z \in Z$

\begin{theo}{Следствие}
    $|2^X| > |X|$

    $\N, 2^{\N}, 2^{2^{\N}}, \ldots$

    $\aleph_0, \aleph_1, \ldots$
\end{theo}

\begin{Remark}{}
    Почему не существует множества всех множеств?

    Пусть существует и называется $U$

    Посмотрим на $U$ и $2^U$

    По Кантору-Бернштейну $U \sim 2^U$, но по теореме Кантора $|U| < |2^U|$ ???? 
\end{Remark}

\begin{theo}{}
    $A$ и $B$ -- множества

    $|A \cup B| = |A| + |B| - |A \cap B|$ при $|A|, |B| < + \infty$

    Если же $|A| = + \infty$, а $|B| < + \infty$, то $|A \cup B| = |A|$

    Что если $|A| = + \infty$ и $|B| = + \infty$? Скажем, НУО $|A| \leq |B|$, тогда $|A \cup B| = |B|$

\begin{Remark}{}
    Вообще мы умеем еще и $|A \times B|$, но там разница будет только в конечных множествах
\end{Remark}

    Есть так же и возведение в степень. С нбчс работа очевидна, а вот с не нбчс уже не так просто

    Что такое $|A|^{|B|}$? Такое описать нормально не получится

    \begin{defin}{}
        Нечто абстрактное и <<умозрительное>> -- $\aleph$

        Так, например, $\aleph + n = \aleph$ и $\aleph \cdot n = \aleph$
    \end{defin}
\end{theo}

\begin{defin}{$\geq$}
    $X$ -- множество
    
    <<$\geq$>> $\subset X \times X$

    \begin{enumerate}
        \item $\forall x \in X \Rightarrow x \geq x$
        \item $\forall x, y, z : x \geq y,\ y \geq z \Rightarrow x \geq z$
        \item $\forall x, y : x \neq y,\ x \geq y \Rightarrow \overline{y \geq x}$
        \item[$\tilde{3}$] $\forall x, y \in X : x \geq y,\ y \geq x \Rightarrow x = y$
    \end{enumerate}
\end{defin}

\begin{theo}{Порядок}
    Заведем отношение $\geq$. Если оно существует для всех пар множества, то это порядок, иначе -- частичный порядок

    Заметим, что он нестрогий. Для строгого нужно добавить проверку на равенство
\end{theo}

\begin{defin}{Частично упорядоченное множество}
    $(X, \geq_X)$ -- ЧУМ
\end{defin}

\begin{Example}{}
    Взяли $\N$ и степенной порядок, т.е.

    $a, b \in \N;\ \exists x \in N\ (x > 1)\ : \begin{cases}
        a = x^k \\
        b = x^m
    \end{cases}$

    $a \geq b \Leftrightarrow k \geq m$
\end{Example}

\begin{defin}{Индуцированный порядок}
    Рассмотрим $Y \subset X$. Если пользоваться тем же отношением порядка на $Y \times Y$, то можно смотреть на $\geq_Y = (\geq_X) \cap (Y \times Y)$ -- индуцированный порядок
\end{defin}

\begin{Remark}{}
    Можно и на $X \times Y$ ввести $\geq_{X \times Y} : (x, y) \geq (a, b) \Leftrightarrow \left[ \begin{gathered}
        x \geq a \\
        \begin{cases}
            x < a \\
            y \geq b
        \end{cases}
    \end{gathered} \right.$

    Такой порядок называется лексикографическим (покоординатным), что в целом то же, что и $(X, \geq_X) + (Y, \geq_Y)$
\end{Remark}

\begin{defin}{Наибольший и максимальный элемент}
    $x \in X$ -- наибольший элемент $\Leftrightarrow \forall y \in X : y < x$

    $x \in X$. Если $\not\exists y \in X : y > x$, то $x$ -- максимальный элемент

    \begin{Remark}{}
        Наибольший элемент -- всегда максимальный, но не наоборот
    \end{Remark}
\end{defin}

\begin{defin}{Изоморфизм}
    $(X, \geq_X) \sim (Y, \geq_Y)$ -- изоморфизм, если $\exists f : X \rightarrow Y$ -- биекция, сохраняющая порядок
\end{defin}

Что можно сказать про $(\R, \geq_\R)$? Можно построить биекцию $x \mapsto x + 1$ -- это автоморфизм

А что с $\R_+, \geq_{\R_+}$? Тут уже не получится построить автоморфизм (т.к. из луча $(0, +\infty)$ уйдем в луч $(1, +\infty)$)

\begin{Remark}{}
    Из существования биекции не следует существование автоморфизма

    Берем $X, Y;\ h : X \to Y$ -- биекция

    И $\forall x, y \in X : x \geq y \Rightarrow h(x) \geq h(y)$

    Смотрим на $\Z, \Q$. Пусть $\exists h : \Z \to \Q$ 

    Рассмотрим двойку и тройку

    $\not\exists x \in \Z : 2 < x < 3$

    $h(2) = y_2;\ h(3) = y_3$

    $h^{-1}(\frac{y_2 + y_3}{2}) = x$

    Целого числа между 2 и 3 нет, но по биекции оно есть
\end{Remark}

\begin{defin}{Плотность}
    $x$ -- плотная точка, если

    $\begin{cases}
        \forall y < x\ \exists z : y < z < x \\
        \forall y > x\ \exists z : x < z < y
    \end{cases}$
\end{defin}

\begin{Example}{}
    Возьмем множество $\{0, \frac{1}{2}, \frac{1}{3}, \ldots \frac{1}{n} \ldots\}$. В нем плотная точка -- только 0
\end{Example}

\begin{theo}{}
    $X$ -- всюду плотное (нет соседних элементов), счетное, без наибольшего и наименьшего элемента

    Это значит, что $X \cong \Q$
\end{theo}

\textit{Доказательство:}

Возьмем $n$ точек из $X$ и $n$ точек из $\Q$. Построим между ними изоморфизм

Теперь нам нужен изоморфизм из $n + 1$ отрезков из $X$ в $n + 1$ отрезок множества $\Q$. Далее идем рекурсивно

Получим для точки что-то типа системы стягивающихся отрезков

\begin{Exercise}{}
    Попробуйте придумать явный изоморфизм между $\Q$ и $\Q \cap (0, 1)$
\end{Exercise}

\begin{Remark}{}
    $x \to x + 1$ -- автоморфизм $\Z$

    $h(\N) \neq \N$

    Пусть есть изоморфизм $g(\Z) \to \N$

    Применим прошлую функцию и получим $h(g(\Z)) \to h(\N)$

    Но $g(h(\Z)) = g(\Z) = \N$, а $h(\N) \neq \N$
\end{Remark}

\begin{nota}{}
    $\forall m < n;\ A(m)$ -- истина $\Rightarrow A(n)$ (если $A(0)$)
\end{nota}

\begin{theo}{}
    $X$ -- ЧУМ

    \begin{enumerate}
        \item $\forall Y \subset X;\ \exists \min Y$
        \item $\not\exists x_1, x_2 \ldots x_n \ldots : x_1 > x_2 > x_3 > \ldots > x_n \ldots$
        \item Для $X$ работает принцип индукции
    \end{enumerate}
\end{theo}

\begin{Remark}{}
    Переформулируем 3 пункт: $A$ -- какое-то произвольное свойство, тогда

    $(\forall x(\forall y(y < x) \Rightarrow A(y)) \Rightarrow A(x)) \Rightarrow \forall x\ A(x)$
\end{Remark}

\textit{Доказательство:}

\begin{itemize}
    \item[$2 \Rightarrow 1$] Пусть $\exists Y :$ в $Y$ нет минимального элемента
    
    Рассмотрим $X_1 \in Y \Rightarrow \exists x_2 < x_1 \Rightarrow \exists x_3 < x_2 \ldots$

    \item[$1 \Rightarrow 2$] очев 
    
    \item[$1 \Rightarrow 3$] Пусть $Y \neq \varnothing;\ \forall y \in Y\ \overline{A}(y)$
    
    $X \setminus Y = A(x)$

    $\exists y_0 = \min Y \Rightarrow \forall x < y_0\ A(x)$

    Тут что-то было

    \item[$3 \Rightarrow 1$] Пусть $\exists Y : \not\exists \min Y$
    
    $A(x) \sim x \notin Y$

    Смотрим на какой-то $x \in Y$. $\forall y < x\ A(y)$

    Дословно: если для какого-то $x$ выполнялось бы условие выше, то $x$ был минимальным, а минимального нет, значит $x$ нет
\end{itemize}

\begin{nota}{Необоснованная индукция}
    Это примеры, когда индукцию мы использовали, но что-то не так

    \begin{enumerate}
        \item Графы
        \item Китайская теорема об остатках
    \end{enumerate}

    Что не так? На самом деле, например, в КТО мы опирались не на $\N$, а на $\N \times \N$. Надо доказать, что оно фундированное, тогда использование индукции обосновано
\end{nota}

\textit{Доказательство:}

$\q{a_1, b_1} < \q{a_2, b_2} \Leftrightarrow \left[ \begin{gathered}
    a_1 < a_2 \\
    \begin{cases}
        a_1 = a_2 \\
        b_1 < b_2
    \end{cases}
\end{gathered} \right.$

Рассмотрим $A \subset X$

$A_1 := \{ a | \q{a, b} \in A \}$

$A_1 \subset \N \Rightarrow \exists a_1$ -- наименьший элемент $A_1$

$B_1 := \{b | \q{a_1, b} \in A\}$

$B_1 \subset \N \Rightarrow \exists b_1$ -- наименьший элемент $B_1$

Иными словами, в $A$ есть элемент $\q{a_1, b_1}$. Про него мы знаем, что все остальные элементы из $A$ либо больше его по первой координате, либо равны с ним по первой и больше по второй

Значит $\q{a_1, b_1}$ -- наименьший, а тогда $A$ -- фундированное

\begin{Remark}{}
    Это работает для $\N^k\ \forall k \in \N$
\end{Remark}

\begin{theo}{}
    А что с $\N + \N$?

    Возьмем их как $x_1 \ldots x_n \ldots$ и $y_1 \ldots y_n \ldots$. Далее $\forall x_i\ \forall y_i\ x_i < y_i$

    Давайте докажем, что это множество тоже фундированное
\end{theo}

\textit{Доказательство:}

Пусть $\exists a_1 \ldots a_n \ldots \in \N + \N : a_1 > a_2 > \ldots > a_n > \ldots$

Возьмем $b_1 \ldots b_n \ldots \in \N_1$

$\q{a_1 \ldots a_n \ldots} \setminus \q{b_1 \ldots b_n \ldots}$ -- бесконечное количество 

\begin{Example}{Где используется?}
    Например, если хотим две разных индукции про одно и то же множество
\end{Example}

\begin{defin}{Упорядоченное множество}
    $X$ -- фундированное, $\rho$ -- линейный порядок

    $(X, \rho)$ -- (вполне) упорядоченное множество, а $\rho$ в таком случае -- полный порядок
\end{defin}

\begin{theo}{Свойства}
    \begin{enumerate}
        \item $\forall x \in X\ \exists y \in X : y > x$ и $\not\exists z : y > z > x$
        \item Если $A \subset X$ и $A$ ограниченное, то у него есть наибольший элемент (супремум достижим)
        
        А правда ли это? Смотрите попытку доказать
    \end{enumerate}
\end{theo}

\textit{Доказательство:}

\begin{enumerate}
    \item На уровне очев 
    \item $\forall A \subset X : A$ -- ограничено 
    
    $\exists u = \sup A$

    $\exists a = \min \{ x | x > A \}$

    А потом оказалось, что вообще нет. Возьмем $\N + \N$. Ноль второго подмножество будет больше любого элемента первого, но недостижим в нем
\end{enumerate}

\begin{Exercise}{}
    Построить вполне упорядоченное множество для рациональных чисел (нужен порядок, гарантирующий фундированность)
\end{Exercise}

Начнем обозначать элементы какого-то множества как $0, 1, 2 \ldots$. Получили отображение на натуральный ряд. Если множество конечно, то процесс оборвется, иначе получим в точности натуральный ряд

Более того, можем вполне делать $\omega_1, \omega_1 + 1, \omega_1 + 2 \ldots$. А можем еще и $\omega_2, \omega_2 + 1, \omega_2 + 2 \ldots$

\begin{defin}{Начальный отрезок}
    $A$ -- вполне упорядоченное множество

    Пусть $A = B \sqcup C$. При этом $\forall b \in B\ \forall c \in C\ b < c$

    Тогда $B$ -- начальный отрезок $A$
\end{defin}

\begin{nota}{Свойства:}
    \begin{enumerate}
        \item Начальный отрезок -- вполне упорядоченное множество
        \item Начальный отрезок начального отрезка -- начальный отрезок
        \item Если рассмотрим $\{B | B \text{ -- начальный отрезок } A\}$, то это множество упорядочено по включению
    \end{enumerate}
\end{nota}

\begin{defin}{Трансфинитная индукция}
    У нас есть индукция с шагом $n \to n + 1$, что в целом равносильно рекурсивному доказательству вида $A_{n + 1} \leftarrow A_n$ + замкнуть какой-то базой

    А что если попытаться сделать $A_x \leftarrow A_{[0, x_1]};\ x_1 < x$?

    Индукция, которую получим из такой рекурсии будет трансфинитной 
\end{defin}

\begin{theo}{Теорема Цермело}
    На любом множестве можно ввести такое отношение порядка, что множество станет вполне упорядоченным
\end{theo}

\begin{Remark}{}
    Доказывается она как эквивалентная аксиоме выбора
\end{Remark}

\section{Логика и исчисление высказываний}

\begin{defin}{Высказывание}
    Любому высказыванию можем сопоставить ровно один элемент из множества 
    
    $\{T, F\} =: B$ -- истина или ложь
\end{defin}

\begin{tabular}{cc|c|c}
    $A$ & $B$ & $A \lor B$ & $A \land B$ \\
    \hline
    1 & 1 & 1 & 1 \\
    1 & 0 & 1 & 0 \\
    0 & 1 & 1 & 0 \\
    0 & 0 & 0 & 0
\end{tabular}

\begin{defin}{Пропозициональная формула}
    $A$ -- пропозициональная формула

    \begin{enumerate}
        \item $x_i$ -- пропозициональная переменная $\Rightarrow A$ -- пропозициональная формула
        \item $A$ -- пропозициональная формула $\Rightarrow \overline{A}$ -- пропозициональная формула
        \item $A, B$ -- пропозициональные формулы $\Rightarrow (A \lor B), (A \land B), (A \rightarrow B), (A \Leftrightarrow B)$ -- пропозициональные формулы
    \end{enumerate}
\end{defin}

\begin{defin}{Тавтология}
    Формула, которая истинна при любых значениях переменных
\end{defin}

\begin{Example}{}
    $A \lor \overline{A}$ 
\end{Example}

\begin{Remark}{}
    Все хорошо только если $A$ -- пропозициональная переменная. Если это формула, то нужно аккуратно доказывать тавтологичность

    Попробуем доказать
\end{Remark}

\textit{Доказательство:}

$A = \overline{B}$

Получили $\overline{B} \lor \overline{\overline{B}}$. Попали в рекурсию

\newpage

$x_1 \ldots x_n$ -- переменные на $B$

$f : B^n \to B$

$f(x_1 \ldots x_n)$

\begin{defin}{Полный набор связок}
    Набор связок называется полным, если с его помощью можно выразить любую функцию
\end{defin}

\begin{Example}{}
    Система связок $\{\land, \lor\}$ -- неполная, т.к. нет способа получить $f(0 \ldots 0) \mapsto 1$
\end{Example}

\begin{lem}{Лемма об однообразности разбора}
    $\forall A$ -- пропозициональная формула : $A$ -- не пропозициональная переменная 
    
    $\exists!$ представление в виде $\left[ \begin{gathered}
        B \lor C \\
        B \land C \\
        B \to C \\
        \overline{B}
    \end{gathered} \right.$
\end{lem}

\begin{defin}{Скобочный итог}
    $c_1 \ldots c_k$

    Скобочный итог($i$) = $N_\text{откр}(i) - N_\text{закр}(i)$
\end{defin}

\begin{defin}{Полная система связок}
    Система связок в пропозициональных формулах называется полной, если с ее помощью можно выразить любую пропозициональную формулу
\end{defin}

$x_1 \ldots x_n$ -- переменные

$f(0 \ldots 0) = f_1$

$f(0 \ldots 0 1) = f_2$

$f(1 \ldots 1) = f_{2^n}$

$(\overline{x_1} \land \overline{x_2} \land \ldots \land \overline{x_n})$ -- истинно только если подставили все нули

$(\overline{x_1} \land \overline{x_2} \land \ldots \land x_n)$ -- истинно только если подставили все нули, кроме последнего

Делаем так вплоть до $(x_1 \ldots x_n)$

Теперь скажем, что берем $n$-ую скобку только если $f_n$ -- истина. Объединяем все через $\lor$

\begin{defin}{КНФ}
    Можем $\begin{pmatrix}
        x_1 & \ldots & x_k & f \\
        0 & \ldots & 0 & f_1 \\
        0 & \ldots & 1 & f_2 \\
        \ldots & \ldots & \ldots & \ldots \\
        1 & \ldots & 1 & f_{2^n}
    \end{pmatrix}$
    
    Если $f_i = 1$, то берем $x_1 \land x_2 \land \overline{x_3} \ldots \land x_k$ (если $x_i = 0$, то берем $\overline{x_i}$)
    
    Делаем $\bigvee$ по всем $i$ таким, что $f_i = 1$
    
    Это конъюнктивная нормальная форма
\end{defin}

\begin{defin}{Моном}
    $\bigwedge x_i$ -- моном
\end{defin}

\begin{defin}{Полином Жегалкина}
    Если сделаем моном на $\oplus$, то получим полином Жегалкина
\end{defin}

\begin{theo}{}
    Любую булеву функцию можно представить в виде полинома Жегалкина
\end{theo}

\textit{Доказательство:}

$\overline{A} = A \oplus 1$

$A \lor B = A \oplus B \oplus (A \land B)$

$A \land B = AB$ -- это можно честно перемножить

\begin{theo}{Критерий Поста}
    Система связок полна $\Leftrightarrow$ она не входит ни в один из 5 классов:

    \begin{enumerate}
        \item Монотонные функции
        
        Есть функция $f$, она монотонная, если 
        
        $a_1 \ldots a_k \geq b_1 \ldots b_k \Leftrightarrow f(a_1 \ldots a_k) \geq f(b_1 \ldots b_k)$

        \item Функции, сохраняющие 0
        \item Функции, сохраняющие 1
        \item Линейные функции 
        \item Самодвойственные функции
        
        $f(\overline{x_1} \ldots \overline{x_k}) = \overline{f(x_1 \ldots x_k)}$
    \end{enumerate}
\end{theo}

\textit{Доказательство:}

$f$ -- не сохраняет 0

$f(0 \ldots 0) = 1$

$f(1 \ldots 1) = \left[ \begin{gathered}
    0 \\
    1
\end{gathered} \right.$

$g$ -- не сохраняет 1

$g(0 \ldots 0) = \left[ \begin{gathered}
    0 \\
    1
\end{gathered} \right.$

$g(1 \ldots 1) = 1$

$h(1, 0 \ldots 0) = 0$

$h(0, 0 \ldots 0) = 1$

То есть даже из двух констант умеем получать отрицание

$P(x_1 \ldots x_k)$ -- нелинейная

НУО $P(x_1 \ldots x_k) = x_1x_2A(x_3 \ldots x_k) \oplus x_1B(x_3 \ldots x_k) \oplus x_2C(x_3 \ldots x_k) \oplus D(x_3 \ldots x_k)$

Зафиксируем набор $\alpha = x_3 \ldots x_k : A(\alpha) = 1$

$P(x_1, x_2, \alpha) = \begin{cases}
    x_1x_2 \\
    x_1x_2 \oplus x_1 \\
    x_1x_2 \oplus x_2 \\
    x_1x_2 \oplus x_1 \oplus x_2 \\
    x_1x_2 \oplus 1 \\
    x_1x_1 \oplus x_1 \oplus 1 \\
    x_1x_2 \oplus x_2 \oplus 1 \\
    x_1x_2 \oplus x_1 \oplus x_2 \oplus 1
\end{cases}$

А тут уже есть и $\land$, и $\lor$, значит есть полная система связок, значит доказали

\begin{defin}{}
    $C_b(f)$ -- минимальный размер схемы элементов из $B$ для вычисления $f$
\end{defin}

\begin{Remark}{}
    $\exists \lambda$

    $C_{B_1}(f) \leq \lambda C_{B_2}(f)$

    $C_{B_2}(f) \leq \lambda C_{B_1}(f)$
\end{Remark}

\begin{nota}{}
    $X = x_1 \ldots x_k$

    $Y = y_1 \ldots y_k$ (НУО равны, иначе к меньше в начало докидаем нули)

    Хотим функцию Compare. Какого она может быть размера?
\end{nota}

\end{document}