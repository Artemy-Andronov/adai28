\documentclass[12pt]{article}
\usepackage{config}
\usepackage{subfiles}
\pgfplotsset{compat=1.18}

\begin{document}

\begin{flushright}
    Конспект Шорохова Сергея

    Если нашли опечатку/ошибку - пишите @le9endwp 
\end{flushright}

\tableofcontents
\newpage

\section{Глава 9. Теория меры}

\subsection{§1. Системы множеств}

\begin{defin}{Объемлющее множество}
    $X$ -- объемлющее множество. Будем рассматривать $A \subset X$
\end{defin}

\begin{declar}{Обозначения}
    $A \sqcup B$ -- объединение множеств $A$ и $B$ и множества $A$ и $B$ не пересекаются

    $\bigsqcup\limits_{k = 1}^n A_k$ -- объединение и $A_i \cap A_j = \varnothing$

    Дизъюнктные множества = непересекающиеся множества
\end{declar}

\begin{defin}{Разбиение множества}
    Множества $E_\alpha,\ \alpha \in I$ -- разбиение множества $E$, если $E = \bigsqcup\limits_{a \in I} E_\alpha$
\end{defin}

\begin{defin}{Система подмножеств и ее свойства}
    $\A$ -- система подмножеств $X$ (т.е. $\A \subset 2^X$)

    \begin{enumerate}
        \item $\A$ имеет свойство $\sigma_0$, если $\forall A, B \in \A \Rightarrow A \cup B \in \A$
        \item $\A$ имеет свойство $\delta_0$, если $\forall A, B \in \A \Rightarrow A \cap B \in \A$
        \item $\A$ имеет свойство $\sigma$, если $\forall A_1, A_2 \ldots \in \A \Rightarrow \bigcup\limits_{n = 1}^\infty A_n \in \A$
        \item $\A$ имеет свойство $\delta$, если $\forall A_1, A_2 \ldots \in \A \Rightarrow \bigcap\limits_{n = 1}^\infty A_n \in \A$
        \item $\A$ -- симметричная система, если $\forall A \in \A \Rightarrow X \setminus A \in \A$
    \end{enumerate}
\end{defin}

\begin{Reminder}{}
    $X \setminus \bigcup\limits_{\alpha \in I} A_\alpha = \bigcap\limits_{\alpha \in I} X \setminus A_\alpha$

    $X \setminus \bigcap\limits_{\alpha \in I} A_\alpha = \bigcup\limits_{\alpha \in I} X \setminus A_\alpha$
\end{Reminder}

\begin{propos}{}
    Если $\A$ симметричная система, то $\begin{gathered}
        (\sigma_0) \Leftrightarrow (\delta_0) \\
        (\sigma) \Leftrightarrow (\delta)
    \end{gathered}$
\end{propos}

\begin{defin}{Алгебра}
    $\A$ -- алгебра, если

    \begin{enumerate}
        \item $\varnothing \in \A$
        \item $\A$ -- симметричная система 
        \item Есть свойства $(\sigma_0)$ и $(\delta_0)$
    \end{enumerate}
\end{defin}

\begin{defin}{$\sigma$-алгебра}
    $\A$ -- $\sigma$-алгебра, если 

    \begin{enumerate}
        \item $\varnothing \in \A$
        \item $\A$ -- симметричная система
        \item Есть свойства $(\sigma)$ и $(\delta)$
    \end{enumerate}
\end{defin}

\begin{theo}{Свойства}
    \begin{enumerate}
        \item Если $\A$ -- алгебра и $A_1 \ldots A_n \in \A$, то $\bigcup\limits_{k = 1}^n A_k$ и $\bigcap\limits_{k = 1}^n A_k \in \A$
        \item Если $\A$ -- $\sigma$-алгебра, то $\A$ -- алгебра
        \item Если $\A$ -- алгебра и $A, B \in \A$, то $\underbrace{A \setminus B}_{A \cap (X \setminus B)} \in \A$ 
    \end{enumerate}
\end{theo}

\begin{Example}{}
    \begin{enumerate}
        \item $X = \R^n$
        
        $\A$ -- все ограниченные множества и их дополнения. Это алгебра, но не $\sigma$-алгебра

        \item $2^X$ -- $\sigma$-алгебра
        \item Индуцированная ($\sigma$-)алгебра 
        
        $Y \subset X,\ \A$ -- ($\sigma$-)алгебра подмножеств $X$

        $\B := \{A \cap Y : A \in \A\}$ -- ($\sigma$-)алгебра подмножеств $Y$

        \item $X \supset A, B$
        
        $\A$ -- алгебра подмножеств $X$

        $\varnothing, X, A, B, A \cup B, A \cap B, A \setminus B, B \setminus A, X \setminus A, X \setminus B, A \triangle B, X \setminus (A \cap B), X \setminus (A \cup B), \\ X \setminus (A \triangle B), X \setminus (A \setminus B), X \setminus (B \setminus A)$

        \item $A_\alpha$ -- ($\sigma$-)алгебра подмножеств $X$
        
        Тогда $\B = \bigcap\limits_{\alpha \in I} \A_\alpha$ -- ($\sigma$-)алгебра подмножеств $X$

        \textit{Доказательство:}

        \begin{enumerate}
            \item $\varnothing \in \A_\alpha \Rightarrow \varnothing \in \B$
            \item $A \in \B \Rightarrow A \in \A_\alpha \forall \alpha \Rightarrow X \setminus A \in \A_\alpha \forall \alpha \Rightarrow X \setminus A \in \B$
        \end{enumerate}
    \end{enumerate}
\end{Example}

\begin{theo}{}
    Пусть $\E$ -- система подмножеств $X$

    Тогда существует наименьшая по включению ($\sigma$-)алгебра $\A$, содержащая $\E$
\end{theo}

\textit{Доказательство:}

Пусть $\A_\alpha$ -- всевозможные алгебры, содержащие $\E$ ($2^X$ подходит)

$\A := \bigcap\limits_{\alpha \in I} \A_\alpha$ -- алгебра и $\A \subset \A_\alpha \forall \alpha$

\newpage

\begin{defin}{Борелевская оболочка}
    $\E$ -- система подмножеств $X$

    Борелевская оболочка системы $\E$ -- наименьшая по включению $\sigma$-алгебра, содержащая $\E$
\end{defin}

\begin{declar}{Обозначение}
    $\B(\E)$
\end{declar}

\begin{defin}{Борелевская $\sigma$-алгебра}
    Борелевская $\sigma$-алгебра -- это $\B(\E)$, где $\E$ -- всевозможные открытые множества в $\R^n$
\end{defin}

\begin{declar}{Обозначение}
    $\B^n$
\end{declar}

\begin{Remark}{}
    $\B^n \neq 2^{\R^n}$
\end{Remark}

\begin{defin}{Кольцо}
    $\A$ -- семейство подмножеств $X$

    $\A$ -- кольцо, если 
    
    \begin{enumerate}
        \item $\varnothing \in \A$
        \item $A, B \in \A \Rightarrow A \cap B \in \A,\ A \cup B \in \A$
        \item $A, B \in \A \Rightarrow A \setminus B \in \A$
    \end{enumerate}
\end{defin}

\begin{Remark}{}
    $\A$ -- алгебра $\Leftrightarrow \A$ -- кольцо и $X \in \A$
\end{Remark}

\begin{defin}{}
    $\P$ -- семейство подмножеств $X$

    $\P$ -- полукольцо, если 

    \begin{enumerate}
        \item $\varnothing \in \P$
        \item $\forall A, B \in \P \Rightarrow A \cap B \in \P$
        \item $\forall A, B \in \P\ \exists Q_1 \ldots Q_m \in \P$, т.ч. $A \setminus B = \bigsqcup\limits_{k = 1}^m Q_k$
    \end{enumerate}
\end{defin}

\begin{Example}{}
    \begin{enumerate}
        \item $X = \R;\ \P := \{(a, b] : a, b \in \R\}$ -- полукольцо
        \item $X = \R;\ \P := \{(a, b] : a, b \in \Q\}$ -- полукольцо
    \end{enumerate}
\end{Example}

\begin{lem}{}
    $\bigcup\limits_{k = 1}^n A_k = \bigsqcup\limits_{k = 1}^n \underbrace{(A_k \setminus \bigcup\limits_{j = 1}^{k - 1} A_j)}_{B_k}$ (для $\infty$ вместо $n$ тоже верно)
\end{lem}

\textit{Доказательство:}

\begin{itemize}
    \item $B_k \subset A_k \Rightarrow\ \supset$ верно
    \item $\subset$ возьмем $x \in \bigcup\limits_{k = 1}^n A_k \Rightarrow$ найдется наименьший индекс $m$, т.ч. $x \in A_m$ и $x \notin A_{m - 1} \ldots A_1 \Rightarrow \\ \Rightarrow x \in B_m$
    \item Дизъюнктность $k < m \Rightarrow B_k \cap B_m = \varnothing$

    $B_m = A_m \setminus \bigcup\limits_{j = 1}^{m - 1} A_j \subset A_m \setminus A_k \subset A_m \setminus B_k$
    
    $B_k \subset A_k$
\end{itemize}

\begin{theo}{}
    $\P$ -- полукольцо. Тогда

    \begin{enumerate}
        \item $P, P_1 \ldots P_n \in \P \Rightarrow \exists Q_1 \ldots Q_m \in \P$, т.ч. $P \setminus \bigcup\limits_{k = 1}^n P_k = \bigsqcup\limits_{j = 1}^m Q_j$
        \item $P_1, P_2 \ldots \in \P \Rightarrow \exists Q_{ij} \in \P$, т.ч. $\bigcup\limits_{k = 1}^n P_k = \bigsqcup\limits_{k = 1}^n\bigsqcup\limits_{j = 1}^{m_k} Q_{kj}$, где $Q_{kj} \subset P_k \forall\ k, j$
        \item В п. 2 можно вместо $n$ написать $\infty$
    \end{enumerate}
\end{theo}

\textit{Доказательство:}

\begin{enumerate}
    \item Индукция. База $n = 1$ -- определение полукольца
    
    Переход $n \to n + 1$

    $P \setminus \bigcup\limits_{k = 1}^{n + 1}P_k = \underbrace{(P \setminus \bigcup\limits_{k = 1}^n P_k)}_\text{инд. предполож.} \setminus P_{n + 1} = \underbrace{(\bigsqcup\limits_{j = 1}^m Q_j)}_\text{где $Q_j \in \P$} \setminus P_{n + 1} = \bigsqcup\limits_{j = 1}^m Q_j \setminus P_{n + 1} = \bigsqcup\limits_{j = 1}^m\bigsqcup\limits_{i = 1}^{m_j} Q_{ji}$

    \item $\bigcup\limits_{k = 1}^n P_k = \bigsqcup\limits_{k = 1}^n \underbrace{(P_k \setminus \bigcup\limits_{j = 1}^{k - 1} P_j)}_\text{п. 1}$
\end{enumerate}

\begin{defin}{}
    $\P$ -- полукольцо подмножеств $X$

    $\QQ$ -- полукольцо подмножеств $Y$

    $\P \times \QQ := \{A \times B : A \in \P \text{ и } B \in \QQ\}$ -- декартово произведение полуколец $\P$ и $\QQ$
\end{defin}

\newpage

\begin{theo}{}
    Декартово произведение полуколец -- полукольцо
\end{theo}

\textit{Доказательство:}

\begin{enumerate}
    \item Пустые очев
    \item $C \times D$ и $A \times B \in \P \times \QQ \Rightarrow (A \times B) \cap (C \times D) = \underbrace{(A \cap C)}_{\in \P} \times \underbrace{(B \cap D)}_{\in \QQ}$
    \item $A \times B, C \times D \in \P \times \QQ \xRightarrow[]{?} (A \times B) \setminus (C \times D) = \bigsqcup\limits_{k = 1}^m \underbrace{P_k}_{\in \P} \times \underbrace{Q_k}_{\in \QQ}$
    
    $(A \times B) \setminus (C \times D) = \underbrace{(A \setminus C)}_{\bigsqcup\limits_{j = 1}^m P_j} \times \underbrace{B}_{\in \QQ} \sqcup \underbrace{(A \cap C)}_{\in \P} \times \underbrace{(B \setminus D)}_{\bigsqcup\limits_{i = 1}^n Q_i}$
\end{enumerate}

\begin{defin}{Замкнутый и открытый параллелепипеды}
    $a, b \in \R^n$

    Замкнутый параллелепипед $[a, b] := [a_1, b_1] \times \ldots \times [a_n, b_n]$

    Открытый параллелепипед $(a, b) := (a_1, b_1) \times \ldots \times (a_n, b_n)$
\end{defin}

\begin{defin}{Ячейка}
    $a, b \in \R^n$

    Ячейка $(a, b] := (a_1, b_1] \times \ldots \times (a_n, b_n]$
\end{defin}

\begin{Remark}{}
    $(a, b) \subset (a, b] \subset [a, b]$
\end{Remark}

\begin{propos}{}
    \begin{enumerate}
        \item Непустая ячейка -- объединение возрастающей (по включению) последовательности замкнутых параллелепипедов
        \item Непустая ячейка -- пересечение убывающей (по включению) последовательности открытых параллелепипедов
    \end{enumerate}
\end{propos}

\textit{Доказательство:}

\begin{enumerate}
    \item $A_k := [a_1 - \frac{1}{k}, b_1] \times [a_2 - \frac{1}{k}, b_2] \times \ldots \times [a_n - \frac{1}{k}, b_n]$
    
    $A_{k + 1} \supset A_k$ и $\bigcup\limits_{k = 1}^\infty A_k = (a, b]$

    \item $B_k := (a_1, b_1 + \frac{1}{k}) \times (a_2, b_2 + \frac{1}{k}) \times \ldots \times (a_n, b_n + \frac{1}{k})$
    
    $B_{k + 1} \subset B_k$ и $\bigcap\limits_{k = 1}^\infty B_k = (a, b]$
\end{enumerate}

\begin{declar}{Обозначения}
    $\P^n := \{(a, b] : a, b \in \R^n\}$

    $\P_\Q^n := \{(a, b] : a, b \in \Q^n\}$
\end{declar}

\begin{propos}{}
    $\P^n$ и $\P^n_\Q$ -- полукольца
\end{propos}

\textit{Доказательство:}

$\P^n = \underbrace{\P^1 \times \P^1 \times \ldots \times \P^1}_\text{полукольца}$

\begin{theo}{}
    $G$ -- непустое открытое множество в $\R^m$

    Тогда $G$ представимо в виде счетного дизъюнктного объединения ячеек с рациональными координатами вершин
\end{theo}

\textit{Доказательство:}

У АИ тут рисуночки, посмотрите запись!

Для $x \in G$ построим ячейку $P_x$ с рациональными координатами вершин, т.ч. $P_x \in G$ и $x \in P_x$

$\bigcup\limits_{x \in G} P_x = G$

Ячеек с рациональными координатами вершин счетное число. Значит если выкинуть повторы из объединения выше, то останется счетное объединение

$G = \bigcup\limits_{n = 1}^\infty P_{x_n} = \bigsqcup\limits_{n = 1}^\infty \bigsqcup\limits_{j = 1}^{m_n} Q_{nj}$ -- ячейки с рациональными координатами вершин

\begin{theo}{Следствие}
    $\B^m = \B(\P^m) = \B(\P^m_\Q)$
\end{theo}

\textit{Доказательство:}

\begin{enumerate}
    \item $\B^m \supset \B(\P^m)$. Достаточно доказать, что $\B^m \supset \P^m$
    
    $(a, b]$ -- счетное пересечение открытых параллелепипедов (т.к. открытых множеств) $\Rightarrow (a, b]$ лежит в $\sigma$-алгебре, содержащей все открытые множества

    \item $\B(\P^m) \supset \B(\P^m_\Q)$. Достаточно доказать, что $\B(\P^m) \supset \P^m_\Q$, но $\B(\P^m) \supset \P^m \supset \P^m_\Q$
    
    \item $\B(\P^m_\Q) \supset \B^m$. Достаточно доказать, что $\B(\P^m_\Q)$ содержит все открытые множества. Это следует из теоремы 1.5.
\end{enumerate}

\newpage

\subsection{§2. Объем и мера}

\begin{defin}{Объем}
    $\P$ -- полукольцо. $\mu : \P \to [0, + \infty]$

    $\mu$ -- объем, если 

    \begin{enumerate}
        \item $\mu \varnothing = 0$
        \item Если $A_1, \ldots A_n$ и $\bigsqcup\limits_{k = 1}^n A_k \in \P$, то $\mu(\bigsqcup\limits_{k = 1}^n A_k) = \sum\limits_{k = 1}^n \mu A_k$
    \end{enumerate}
\end{defin}

\begin{defin}{Мера}
    $\P$ -- полукольцо. $\mu : \P \to [0, + \infty]$

    $\mu$ -- мера, если 

    \begin{enumerate}
        \item $\mu \varnothing = 0$
        \item Если $A_1, A_2 \ldots$ и $\bigsqcup\limits_{k = 1}^\infty A_k$, то $\mu(\bigsqcup\limits_{k = 1}^\infty A_k) = \sum\limits_{k = 1}^\infty \mu A_k$
    \end{enumerate}
\end{defin}

\begin{Exercise}{}
    Если $\mu \varnothing \neq +\infty$, то $\mu \varnothing = 0$ из свойства 2
\end{Exercise}

\begin{Example}{Примеры объемов}
    \begin{enumerate}
        \item $X = \R,\ \P^1$. Длина -- объем. $\mu(a, b] = b - a$
        \item $X = \R,\ \P^1$. $g : \R \to \R$ -- нестрого возрастающая функция
        
        $\nu_g(a, b] := g(b) - g(a)$

        \item Классический объем на $\P^m$
        
        $\lambda_m(a, b] = (b_1 - a_1)(b_2 - a_2) \ldots (b_m - a_m)$ -- объем и даже мера (докажем позже)

        \item $x_0 \in X;\ \mu A = \begin{cases}
            0 & x_0 \notin A \\
            1 & x_0 \in A
        \end{cases}$

        \item $X = \R^2;\ \P$ -- ограниченные множества и их дополнения
        
        $\mu A = \begin{cases}
            0 & A \text{ -- ограничена} \\
            1 & A \text{ дополнение ограничено}
        \end{cases}$ -- объем, но не мера
    \end{enumerate}
\end{Example}

\begin{theo}{Свойства объема}
    $\P$ -- полукольцо, $\mu$ -- объем на $\P$. Тогда 

    \begin{enumerate}
        \item Монотонность
        
        $P, \tilde{P} \in \P$ и $P \subset \tilde{P} \Rightarrow \mu P \leq \mu \tilde{P}$

        \item Усиленная монотонность 
        
        $P_1, P_2 \ldots P_n, \tilde{P} \in \P$ и $\bigsqcup\limits_{k = 1}^n P_k \subset \tilde{P} \Rightarrow \sum\limits_{k = 1}^n \mu P_k \leq \mu \tilde{P}$

        \item[2'.] $P_1, P_2 \ldots, \tilde{P} \in \P$ и $\bigsqcup\limits_{k = 1}^\infty P \subset \tilde{P} \Rightarrow \sum\limits_{k = 1}^\infty \mu P_k \leq \mu \tilde{P}$
        
        \item[3.] Конечная полуаддитивность
        
        $P_1 \ldots P_n, P \in \P$ и $P \subset \bigcup\limits_{k = 1}^n P_k \Rightarrow \mu P \leq \sum\limits_{k = 1}^n \mu P_k$
    \end{enumerate}
\end{theo}

\textit{Доказательство:}

\begin{enumerate}
    \item[2.] $\tilde{P} \setminus \bigsqcup\limits_{k = 1}^n P_k = \bigsqcup\limits_{j = 1}^m Q_j$, где $Q_j \in \P$
    
    $\tilde{P} = \bigsqcup\limits_{k = 1}^n P_k \sqcup \bigsqcup\limits_{j = 1}^m Q_j \Rightarrow \mu \tilde{P} = \sum\limits_{k = 1}^n \mu P_k + \underbrace{\sum\limits_{j = 1}^m}_{\geq 0} \mu Q_j \geq \sum\limits_{k = 1}^n \mu P_k$

    \item[2'.] Предельный переход в неравенстве
    
    \item[3.] $P'_k := P_k \cap P \in \P \Rightarrow P = \bigcup\limits_{k = 1}^n P_k' \underbrace{=}_\text{th.} \bigsqcup\limits_{k = 1}^n \bigsqcup\limits_{j = 1}^{m_k} Q_{kj}$ (они из $\P$) $\Rightarrow \mu P = \sum\limits_{k = 1}^n \sum\limits_{j = 1}^{m_k} \mu Q_{kj}$
    
    $P_k \supset P_k' \supset \bigsqcup\limits_{j = 1}^{m_k} Q_{kj} \Rightarrow \mu P_k \geq \sum\limits_{j = 1}^{m_k} \mu Q_{kj}$
\end{enumerate}

\begin{Remark}{}
    \begin{enumerate}
        \item Если $\mu$ -- объем на алгебре $\A$, $A \subset B;\ A, B \in \A$ и $\mu A < + \infty$, то $\mu (B \setminus A) = \mu B - \mu A$
        
        \textit{Доказательство:} Т.к. $B = A \sqcup (B \setminus A)$

        \item Объем на полукольце можно продолжить на кольцо, состоящего из всевозможных объединений элементов полукольца
    \end{enumerate}
\end{Remark}

\begin{theo}{}
    $\P$ и $\QQ$ -- полукольца подмножеств $X$ и $Y$. $\mu$ и $\nu$ -- объемы на $\P$ и $\QQ$

    $\lambda \underbrace{(P \times Q)}_{P \in \P;\ Q \in \QQ} := \mu P \cdot \nu Q$ (считаем, что $0 \cdot + \infty = + \infty \cdot 0 = 0$)

    Тогда $\lambda$ -- объем на $\P \times \QQ$
\end{theo}

\begin{theo}{Следствие}
    Классический объем $\lambda_m$ -- объем
\end{theo}

\textit{Доказательство:}

\begin{itemize}
    \item[Случай 1.] $P = \bigsqcup\limits_{j = 1}^m P_j$ и $Q = \bigsqcup\limits_{k = 1}^n Q_k$
    
    Тогда $P \times Q = \bigsqcup\limits_{j = 1}^m \bigsqcup\limits_{k = 1}^n P_j \times Q_k$

    $\lambda(P \times Q) = \mu P \cdot \nu Q = \sum\limits_{j = 1}^m \mu P_j \cdot \sum\limits_{k = 1}^n \nu Q_k = \sum\limits_{j = 1}^m \sum\limits_{k = 1}^n \mu P_j \cdot \nu Q_k = \sum\limits_{j = 1}^m \sum\limits_{k = 1}^n \lambda(P_j \times Q_k)$

    \item[Случай 2.] $P \times Q = \bigsqcup\limits_{k = 1}^n P_k \times Q_k \xRightarrow[]{?} \lambda(P \times Q) = \sum\limits_{k = 1}^n \lambda(P_k \times Q_k)$
    
    Разбиваем $P$ на кусочки $P = \bigsqcup\limits_{j = 1}^m P_j'$ и каждая $P_k$ -- дизъюнктное объединение \\ каких-то $P_j'$
\end{itemize}

\begin{Example}{Примеры мер}
    \begin{enumerate}
        \item $\lambda_m$ -- мера (потом докажем)
        \item $g : \R \to \R$ -- нестрого возрастающая и непрерывная справа во всех точках
        
        $\nu_g(a, b] := g(b) - g(a)$ -- мера

        \item $x_0 \in X;\ \mu A = \begin{cases}
            1 & x_0 \in A \\
            0 & x_0 \notin A 
        \end{cases}$ -- мера на $2^X$

        \item Считающая мера = количество элементов в множестве
        \item $X;\ \begin{gathered}
            t_1, t_2 \ldots \in X \\
            w_1, w_2 \ldots \geq 0
        \end{gathered};\ \mu A := \sum\limits_{k : t_k \in A} w_k$ -- мера на $2^X$

        Счетная аддитивность: $A = \bigsqcup\limits_{k = 1}^\infty A_k \xRightarrow[]{?} \mu A = \sum\limits_{k = 1}^\infty \mu A_k$

        В множестве $A_k$ гирьки $w_{k_1}, w_{k_2} \ldots$

        $\mu A_k = \sum\limits_{j = 1}^\infty w_{k_j}$ и $\mu A = \sum w_{k_j}$
        
        Надо понять, что $\sum\limits_{k = 1}^\infty \sum\limits_{j = 1}^\infty w_{k_j} = \sum w_{k_j}$

        \begin{itemize}
            \item[$\leq :$] $\underbrace{\sum\limits_{k = 1}^K \sum\limits_{j = 1}^\infty w_{k_j}}_{\sum\limits_{j = 1}^\infty \sum\limits_{k = 1}^K w_{k_j}} \leq R \Rightarrow L \leq R$
            \item[$\geq :$] Берем частичную сумму $S$ для $R$. Надо доказать, что $S \leq L$
            
            $\begin{gathered}
                K = \max k \text{ в этой частичной сумме} \\
                J = \max j \text{ в этой частичной сумме}
            \end{gathered} \Rightarrow S \leq \sum\limits_{k = 1}^K \sum\limits_{j = 1}^J w_{k_j} \leq L$
        \end{itemize}
    \end{enumerate}
\end{Example}

\begin{theo}{}
    $\mu : \P \to [0, + \infty]$ -- объем на полукольце $\P$. Тогда 

    $\mu$ -- мера $\Leftrightarrow$ (счетная полуаддитивность)

    $(P, P_k \in \P)\ \forall P \subset \bigcup\limits_{k = 1}^\infty P_k \Rightarrow \mu P \leq \sum\limits_{k = 1}^\infty \mu P_k$
\end{theo}

\textit{Доказательство:}

\begin{itemize}
    \item[$\Leftarrow :$] $P = \bigsqcup\limits_{k = 1}^\infty P_k \xRightarrow[\text{сч. полуадд.}]{} \mu P \leq \sum\limits_{k = 1}^\infty \mu P_k$
    
    $P = \bigsqcup\limits_{k = 1}^\infty P_k \xRightarrow[\text{усил. монот.}]{} \mu P \geq \sum\limits_{k = 1}^\infty \mu P_k$

    \item[$\Rightarrow :$] $P_k' := P \cap P_k \Rightarrow P = \bigcup\limits_{k = 1}^\infty P_k' = \bigsqcup\limits_{k = 1}^\infty \bigsqcup\limits_{j = 1}^{m_k} Q_{k_j}$, где $Q_{k_j} \subset P_k' \subset P_k \xRightarrow[\mu \text{ -- мера}]{} \mu P = \sum\limits_{k = 1}^\infty \underbrace{\sum\limits_{j = 1}^{m_k} \mu Q_{k_j}}_{\leq \mu P_k}$
    
    $\bigsqcup\limits_{j = 1}^{m_k} Q_{k_j} \subset P_k \xRightarrow[\text{усил. монот.}]{} \mu P_k \geq \sum\limits_{j = 1}^{m_k} \mu Q_{k_j}$
\end{itemize}

\begin{theo}{Следствие}
    $\mu$ -- мера на $\sigma$-алгебре. Тогда счетное объединение множеств нулевой меры -- множество нулевой меры 
\end{theo}

\textit{Доказательство:}

$\mu A = 0;\ A := \bigcup\limits_{k = 1}^\infty A_k \Rightarrow \mu A \leq \sum\limits_{k = 1}^\infty \mu A_k = 0 \Rightarrow \mu A = 0$

\begin{theo}{Непрерывность меры снизу}
    $\mu$ -- объем на $\sigma$-алгебре $\A$. Тогда равносильны

    \begin{enumerate}
        \item $\mu$ -- мера 
        \item $A_1 \subset A_2 \subset A_3 \subset \ldots;\ A_k \in \A$. Тогда $\mu (\bigcup\limits_{k = 1}^\infty A_k) = \lim\limits_{k \to \infty} \mu A_k$
    \end{enumerate}
\end{theo}

\textit{Доказательство:}

\begin{itemize}
    \item[$1 \Rightarrow 2 :$] $A_0 \neq \varnothing$ и $B_k := A_k \setminus A_{k - 1};\ A := \bigcup\limits_{k = 1}^\infty A_k$
    
    Тогда $A = \bigsqcup\limits_{k = 1}^\infty B_k \Rightarrow \mu A = \sum\limits_{k = 1}^\infty \mu B_k = \lim\limits_{n \to \infty} \underbrace{\sum\limits_{k = 1}^n \mu B_k}_{\mu(\bigsqcup\limits_{k = 1}^n B_k)} = \lim\limits_{n \to \infty} \mu A_n$

    \item[$2 \Rightarrow 1 :$] Пусть $A = \bigsqcup\limits_{k = 1}^\infty C_k;\ A_n := \bigsqcup\limits_{k = 1}^n C_k \Rightarrow A_1 \subset A_2 \subset \ldots \Rightarrow \mu A = \lim\limits_{n \to \infty} \mu A_n = \lim\limits_{n \to \infty} \mu(\bigsqcup\limits_{k = 1}^n C_k) = \\ = \lim\limits_{n \to \infty} \sum\limits_{k = 1}^n \mu C_k = \sum\limits_{k = 1}^\infty \mu C_k$
\end{itemize}

\begin{theo}{Непрерывность меры сверху}
    $\mu$ -- объем на $\sigma$-алгебре $\A$ и $\mu X < + \infty$. Следующие условия равносильны 

    \begin{enumerate}
        \item $\mu$ -- мера 
        \item Непрерывность меры сверху
        
        $A_1 \supset A_2 \supset A_3 \supset \ldots;\ A_k \in \A \Rightarrow \mu(\bigcap\limits_{k = 1}^\infty A_k) = \lim\limits_{k \to \infty} \mu A_k$
        \item Непрерывность меры сверху на пустом множестве
        
        $A_1 \supset A_2 \supset A_3 \supset \ldots;\ A_k \in \A$ и $\bigcap\limits_{k = 1}^\infty A_k = 0 \Rightarrow \lim\limits_{k \to \infty} \mu A_k = 0$
    \end{enumerate}
\end{theo}

\textit{Доказательство:}

\begin{itemize}
    \item[$1 \Rightarrow 2 :$] $B_k := A_1 \setminus A_k;\ B_1 \subset B_2 \subset B_3 \subset \ldots$
    
    $\bigcup\limits_{k = 1}^\infty B_k = A_1 \setminus \bigcap\limits_{k = 1}^\infty A_k$. По предыдущей теореме $\underbrace{\mu(\bigcup\limits_{k = 1}^\infty B_k)}_{\mu A_1 - \mu(\bigcap\limits_{k = 1}^\infty A_k)} = \lim\limits_{k \to \infty} \mu B_k = \mu A_1 - \lim\limits_{k \to \infty} \mu A_k$

    \item[$2 \Rightarrow 3 :$] Очев, 3. -- частный случай 2.
    
    \newpage
    
    \item[$3 \Rightarrow 1 :$] $A = \bigsqcup\limits_{k = 1}^\infty C_k; A_n := \bigsqcup\limits_{k = n + 1}^\infty C_k;\ \bigcap\limits_{n = 1}^\infty A_n = \varnothing$ и $A_1 \supset A_2 \supset A_3 \supset \ldots \Rightarrow \lim \mu A_n = 0$
    
    $A = \bigsqcup\limits_{k = 1}^n C_k \sqcup A_n \Rightarrow \mu A = \underbrace{\sum\limits_{k = 1}^n \mu C_k}_{\to \sum\limits_{k = 1}^\infty \mu C_k} + \underbrace{\mu A_n}_{\to 0}$
\end{itemize}

\begin{theo}{Следствие}
    $\mu$ -- мера на $\sigma$-алгебре $\A$ и $A_1 \supset A_2 \supset A_3 \supset \ldots$ и $\mu A_m < + \infty$ для некоторого $m$

    Тогда $\mu(\bigcap\limits_{k = 1}^\infty A_k) = \lim \mu A_k$
\end{theo}

\textit{Доказательство:}

Пишем $A_m \setminus A_k$ вместо $A_1 \setminus A_k$

\begin{Remark}{}
    Условие $\mu X < + \infty$ важно. $A_n := [n, + \infty)$ и $\lambda_1 A_n = + \infty;\ \bigcap\limits_{n = 1}^\infty[n, + \infty) = \varnothing$
\end{Remark}

\begin{Exercise}{}
    Придумать объем, не являющийся мерой, который обладает свойством из следствия
\end{Exercise}

\newpage

\subsection{\S 3. Продолжение меры}

\begin{defin}{Субмера}
    $\nu : 2^X \to [0, + \infty]$ -- субмера, если 

    \begin{enumerate}
        \item $\nu \varnothing = 0$
        \item Монотонность: $A \subset B \Rightarrow \nu A \leq \nu B$
        \item Счетная полуаддитивность: $A \subset \bigcup\limits_{n = 1}^\infty A_n \Rightarrow \nu A \leq \sum\limits_{n = 1}^\infty \nu A_n$
    \end{enumerate}
\end{defin}

\begin{Remark}{}
    2. -- частный случай 3.
\end{Remark}

\begin{defin}{Полная мера}
    $\mu$ -- мера на $\A$. $\mu$ -- полная мера, если 

    $A \in \A$, т.ч. $\mu A = 0 \Rightarrow \forall B \subset A\ B \in \A$ (и тогда $\mu B = 0$)
\end{defin}

\begin{defin}{}
    $\nu$ -- субмера. $E \subset X$

    $E$ -- $\nu$-измеримое множество, если $\forall A \subset X \Rightarrow \nu A = \nu (A \cap E) + \nu (A \setminus E)$
\end{defin}

\begin{Remark}{}
    \begin{enumerate}
        \item Достаточно требовать $\geq$, т.к. $\leq$ из полуаддитивности 
        \item $E_1, E_2 \ldots E_n$ -- $\nu$-измеримые и $E = \bigsqcup\limits_{k = 1}^n E_k \Rightarrow \nu(A \cap E) = \sum\limits_{k = 1}^n \nu(A \cap E_k)$
        
        Индукция по $n$. $n \to n + 1$

        $\nu(A \cap \bigsqcup\limits_{k = 1}^{n + 1} E_k) = \nu \underbrace{((A \cap \bigsqcup\limits_{k = 1}^{n + 1} E_k) \cap E_{n + 1})}_{A \cap E_{n + 1}} + \nu \underbrace{((A \cap \bigsqcup\limits_{k = 1}^{n + 1} E_k) \setminus E_{n + 1})}_{A \cap \bigsqcup\limits_{k = 1}^n E_k}$
    \end{enumerate}
\end{Remark}

\begin{theo}{Теорема Каратеодори}
    $\nu$ -- субмера. Тогда 

    \begin{enumerate}
        \item $\nu$-измеримые множества образуют $\sigma$-алгебру 
        \item Сужение $\nu$ на эту $\sigma$-алгебру -- полная мера 
    \end{enumerate}
\end{theo}

\textit{Доказательство:}

$\A$ -- семейство всех $\nu$-измеримых множеств

\begin{enumerate}
    \item Маленькими шагами :)
    
    \begin{itemize}
        \item[Шаг 1.] Если $\nu E = 0$, то $E$ будет $\nu$-измеримым
        
        $\nu \underbrace{(A \cap E)}_{\subset E} + \nu \underbrace{(A \setminus E)}_{\subset A} \leq \nu E + \nu A = 0 + \nu A = \nu A$
    
        \newpage 
    
        \item[Шаг 2.] $\A$ -- симметричная, т.к. если $E \in \A$, то $X \setminus E \in \A$
        
        $A \cap (X \setminus E) = A \setminus E;\ A \setminus (X \setminus E) = A \cap E$
    
        \item[Шаг 3.] Если $E$ и $F \in \A$, то $E \cup F \in \A$
        
        $\nu A = \nu (A \cap E) + \nu (A \setminus E) = \nu (A \cap E) + \nu ((A \setminus E) \cap F) + \nu \underbrace{((A \setminus E) \setminus F)}_{A \setminus (E \cup F)} \geq \\ \geq \nu (A \cap (E \cup F)) + \nu (A \setminus (E \cup F))$
    
        \item[Шаг 4.] $\A$ -- алгебра
        \item[Шаг 5.] $E = \bigsqcup\limits_{n = 1}^\infty E_n$ и $E_n \in \A \xRightarrow[]{?} E \in \A$
        
        $\nu A = \nu (A \cap \bigsqcup\limits_{k = 1}^n E_k) + \nu (A \setminus \bigsqcup\limits_{k = 1}^n E_k) \geq \nu (A \cap \bigsqcup\limits_{k = 1}^n E_k) + \nu (A \setminus E) = \underbrace{\sum\limits_{k = 1}^n \nu (A \cap E_k)}_{\to \sum\limits_{k = 1}^\infty} + \nu (A \setminus E) \Rightarrow \\ 
        \Rightarrow \nu A \geq \sum\limits_{k = 1}^\infty \nu (A \cap E_k) + \nu (A \setminus E) \geq \nu(\underbrace{\bigcup\limits_{k = 1}^\infty (A \cap E_k)}_{A \cap E}) + \nu (A \setminus E)$
    
        \item[Шаг 6.] $E = \bigcup\limits_{k = 1}^\infty E_k$
        
        Переделаем в дизъюнктное объединение
    
        Т.е. $\A$ -- $\sigma$-алгебра 
    \end{itemize}

    \item $\nu\mid_\A$ -- мера, т.к. это объем и счетная полуаддитивная
    
    $\nu (A \cap \bigsqcup\limits_{k = 1}^n E_k) = \sum\limits_{k = 1}^n \nu (A \cap E_k);\ A = X \Rightarrow$ объем

    $\nu\mid_\A$ -- полная мера. Если $\nu B = 0$ и $A \subset B$, то $\nu A = 0$ и тогда $A \in \A$ по шагу 1
\end{enumerate}

\begin{defin}{Внешняя мера}
    $\mu$ -- мера на полукольце $\P$. Внешняя мера, порожденная $\mu$ называется 
    
    $\mu^* A := \inf\{\sum\limits_{k = 1}^\infty \mu A_k : A \subset \bigcup\limits_{k = 1}^\infty A_k, A_k \in \P\}$

    Если такого покрытия для $A$ нет, то $\mu^* A = + \infty$
\end{defin}

\begin{Remark}{}
    \begin{enumerate}
        \item Можем рассматривать только покрытия дизъюнктными множествами
        
        $\bigcup\limits_{k = 1}^\infty A_k = \bigsqcup\limits_{k = 1}^\infty \bigsqcup\limits_{j = 1}^{m_k} Q_{k_j}$ и $\bigsqcup\limits_{j = 1}^{m_k} Q_{k_j} \subset A_k$

        \item Если $\mu$ -- мера на $\sigma$-алгебре, то $\mu^*A = \inf\{\mu B : B \supset A \text{ и } B \in \A\}$
    \end{enumerate}
\end{Remark}

\begin{theo}{}
    $\mu^*$ -- субмера, совпадающая с $\mu$ на $\P$
\end{theo}

\textit{Доказательство:}

\begin{itemize}
    \item[Шаг 1.] Если $A \in \P$, то $\mu A = \mu^* A$
    
    \begin{itemize}
        \item[$\geq$] Берем покрытие $A, \varnothing, \varnothing, \ldots$. $\mu^* A = \inf \leq \mu A$
        \item[$\leq$] $A \subset \bigcup\limits_{n = 1}^\infty A_n \Rightarrow \mu A \leq \sum\limits_{n = 1}^\infty \mu A_n$ (счетная полуаддитивность меры) $\Rightarrow \mu A \leq \inf = \mu^* A$
    \end{itemize}

    \item[Шаг 2.] $\mu^*$ -- субмера 
    
    Надо проверить, если $A \subset \bigcup\limits_{n = 1}^\infty A_n \Rightarrow \mu^* A \leq \sum\limits_{n = 1}^\infty \mu^* A_n$

    Если справа есть $+ \infty$, то все очев. Считаем, что $\mu^* A_n < + \infty$

    Возьмем покрытие $A_n \subset \bigcup\limits_{k = 1}^\infty C_{nk}$, т.ч. $C_{nk} \in \P$ и $\sum\limits_{k = 1}^\infty \mu C_{nk} < \mu^* A_n + \frac{\varepsilon}{2^n} \Rightarrow A \subset \bigcup\limits_{n = 1}^\infty \bigcup\limits_{k = 1}^\infty C_{nk}$

    $\mu^* A \leq \sum\limits_{n = 1}^\infty \sum\limits_{k = 1}^\infty \mu C_{nk} < \sum\limits_{n = 1}^\infty \mu^* (A_n + \frac{\varepsilon}{2^n}) = \varepsilon + \sum\limits_{n = 1}^\infty \mu^* A_n$
\end{itemize}

\begin{defin}{Стандартное продолжение меры}
    $\mu_0$ -- мера на полукольце $\P$

    $\mu_0^*$ -- внешняя мера, построенная по $\mu_0$ -- субмера 

    $\mu$ -- сужение субмеры $\mu_0^*$ на $\mu_0^*$-измеримые множества 

    $\mu$ называется стандартным продолжением $\mu_0$
\end{defin}

\begin{declar}{}
    Будем писать $\mu$-измеримые, вместо $\mu_0^*$-измеримые 
\end{declar}

\begin{theo}{}
    Это действительно продолжение. Т.е. множества из $\P$ будут $\mu$-измеримы
\end{theo}

\textit{Доказательство:}

\begin{itemize}
    \item[Шаг 1.] Если $E$ и $A \in \P$, то $\mu_0^* A \geq \mu_0^* (A \cap E) + \mu_0^*(A \setminus E)$
    
    $\mu_0^* A = \mu_0 A$ и $\mu_0^* (A \cap E) = \mu_0(A \cap E)$

    $A \setminus E = \bigsqcup\limits_{k = 1}^m Q_k$, где $Q_k \in \P \Rightarrow A = (A \cap E) \sqcup \bigsqcup\limits_{k = 1}^m Q_k \Rightarrow \mu_0 A = \mu_0(A \cap E) + \underbrace{\sum\limits_{k = 1}^m \mu_0^* Q_k}_{\geq \mu_0^* (A \setminus E)} \geq \\
    \geq \mu_0^*(A \cap E) + \mu_0^*(A \setminus E)$

    \item[Шаг 2.] Если $E \in \P$, а $A \notin \P$
    
    Если $\mu_0^* A = + \infty$, то неравенство очевидно. Считаем, что $\mu_0^* A < + \infty$

    Возьмем покрытие $A \subset \bigcup\limits_{n = 1}^\infty P_n$, т.ч. $\sum\limits_{k = 1}^\infty \mu_0 P_k < \mu_0^* A + \varepsilon$ ($P_n \in \P$)

    $\mu_0 P_k \geq \mu_0^* (P_k \cap E) + \mu_0^* (P_k \setminus E)$

    $\varepsilon + \mu_0^* A > \sum\limits_{k = 1}^\infty \mu_0 P_k \geq \sum\limits_{k = 1}^\infty \mu_0^* (P_k \cap E) + \sum\limits_{k = 1}^\infty \mu_0^* (P_k \setminus E) \geq \mu_0^* \underbrace{(\bigcup\limits_{k = 1}^\infty (P_k \cap E))}_{\supset A \cap E} + \mu_0^* \underbrace{(\bigcup\limits_{k = 1}^\infty (P_k \setminus E))}_{\supset A \setminus E} \geq \\
    \geq \mu_0^* (A \cap E) + \mu_0^* (A \setminus E)$
\end{itemize}

\begin{defin}{$\sigma$-конечная мера}
    Мера $\mu$ -- $\sigma$-конечная, если $X = \bigcup\limits_{n = 1}^\infty X_n$, т.ч. $\mu X_n < + \infty$
\end{defin}

\begin{Remark}{}
    \begin{enumerate}
        \item Меру и ее стандартное продолжение будем обозначать одинаково
        \item $\mu$ задана на $\sigma$-алгебре 
        
        $\mu A = \inf \{\sum\limits_{k = 1}^\infty \mu P_k : P_k \in \P,\ \bigcup\limits_{k = 1}^\infty P_k \supset A\}$

        \item Применение стандартного продолжения к стандартному продолжению меры не дает ничего нового
        \item Можно ли продолжить меру на более широкий класс множеств? 
        
        Обычно да, но нет однозначности продолжения 

        \item Можно ли по-другому продолжить меру на $\sigma$-алгебру $\mu$-измеримых множеств?
        
        Если $\mu_0$ -- $\sigma$-конечная мера, то нет!

        \item Обязательно ли полная мера задана на $\sigma$-алгебре $\mu$-измеримых множеств?
        
        Если $\mu_0$ -- $\sigma$-конечная, то да
    \end{enumerate}
\end{Remark}

\begin{Exercise}{}
    Доказать замечание 1.9.3.

    Подсказка: нужно доказать, что $\mu_0^* = \mu^*$
\end{Exercise}

\begin{theo}{}
    $\P$ -- полукольцо, $\mu$ -- стандартное продолжение с полукольца

    $\mu^*$ -- внешняя мера. $A$ -- множество, т.ч. $\mu^*A < + \infty$. Тогда существует $B_{nk} \in \P$, т.ч. $C_n := \bigcup\limits_{k = 1}^\infty B_{nk},\ C := \bigcap\limits_{n = 1}^\infty C_n,\ C \supset A$ и $\mu C = \mu^* A$
\end{theo}

\textit{Доказательство:}

$\mu^* A = \inf \{\sum\limits_{k = 1}^\infty \mu P_k : P_k \in \P \text{ и } \bigcup\limits_{k = 1}^\infty P_k \supset A\}$

Пусть $B_{nk} \in \P$, т.ч. $\sum\limits_{k = 1}^\infty \mu B_{nk} < \mu^* A + \frac{1}{n}$ и $\bigcup\limits_{k = 1}^\infty B_{nk} \supset A$

$A \subset C_n = \bigcup\limits_{k = 1}^\infty B_{nk} \Rightarrow \mu C_n \leq \sum\limits_{k = 1}^\infty \mu B_{nk} < \mu^* A + \frac{1}{n}$

$A \subset C = \bigcap\limits_{n = 1}^\infty C_n \subset C_n;\ \mu C \leq \mu C_n < \mu^* A + \frac{1}{n} \Rightarrow \mu C \leq \mu^* A$

$C \supset A \Rightarrow \mu^* A \leq \mu^* C = \mu C$

\begin{theo}{Следствие}
    $\P$ -- полукольцо, $\mu$ -- стандартное продолжение с $\P$, $A$ -- $\mu$-измеримое множество. \\ $\mu A < + \infty$. Тогда существует $B \in \B(\P)$ и $e$ -- $\mu$-измеримое, т.ч. $A = B \sqcup e$ и $\mu e = 0$
\end{theo}

\textit{Доказательство:}

По теореме существует $C \in \B(\P)$, т.ч. $A \subset C$ и $\mu A = \mu C$

$e_1 := C \setminus A;\ \mu e_1 = \mu C - \mu A = 0$

По теореме найдется $e_2 \in \B(\P)$, т.ч. $e_1 \subset e_2$ и $\mu e_2 = \mu e_1 = 0 \Rightarrow A \supset C \setminus e_2$

$\mu(\underbrace{C \setminus e_2}_{B}) = \mu C = \mu A$

$e := A \setminus B \Rightarrow \mu e = \mu A - \mu B = 0$

\begin{theo}{Единственность продолжения}
    $\P$ -- полукольцо, $\mu$ -- стандартное продолжение с полукольца, $\A$ -- $\sigma$-алгебра, на которой задана $\mu$. $\nu$ -- мера на $\A$, т.ч. $\mu P = \nu P\ \forall P \in \P$

    Если мера $\mu$ -- $\sigma$-конечна, то $\mu A = \nu A\ \forall A \in \A$

    \begin{Reminder}{$\sigma$-конечность}
        $\mu$ -- $\sigma$-конечна, если $X = \bigsqcup\limits_{n = 1}^\infty X_n$, т.ч. $\mu X_n < + \infty$
    \end{Reminder}
\end{theo}

\textit{Доказательство:}

\begin{itemize}
    \item[Шаг 1.] $\mu A \geq \nu A\ \forall A \in \A$
    
    $\mu A = \inf \{\underbrace{\sum\limits_{k = 1}^\infty \mu P_k}_{\geq \nu A} : A \subset \bigcup\limits_{k = 1}^\infty P_k \text{ и } P_k \in \P\}$. По усиленной монотонности меры $\nu \\ \nu A \leq \sum\limits_{k = 1}^\infty \nu P_k = \sum\limits_{k = 1}^\infty \mu P_k \Rightarrow \mu A \geq \nu A$

    \item[Шаг 2.] Если $E \in \A$ и $\mu P < + \infty$, то $\mu(P \cap E) = \nu(P \cap E)\ \forall P \in \P$
    
    $\mu P = \underbrace{\mu(P \cap E)}_{\geq \nu(P \cap E)} + \underbrace{\mu(P \setminus E)}_{\geq \nu(P \setminus E)} \geq \nu(P \cap E) + \nu(P \setminus E) = \nu P \Rightarrow \mu(P \cap E) = \nu(P \cap E)$

    \item[Шаг 3.] $\mu A = \nu A\ \forall A \in \A$
    
    $\mu$ -- $\sigma$-конечная $\Rightarrow X = \bigsqcup\limits_{n = 1}^\infty P_n$, т.ч. $P_n \in \P$ и $\mu P_n < + \infty$

    Тогда $A = \bigsqcup\limits_{n = 1}^\infty(A \cap P_n)$

    $\mu A = \sum\limits_{n = 1}^\infty \mu(A \cap P_n) = \sum\limits_{n = 1}^\infty \nu(A \cap P_n) = \nu A$
\end{itemize}

\newpage 

\subsection{\S 4. Мера Лебега}

\begin{theo}{}
    $\lambda_m$ (классический объем в $\R^m$) -- $\sigma$-конечная мера 
\end{theo}

\textit{Доказательство:} (на записи рисуночки!)

Достаточно проверить счетную полуаддитивность $\lambda_m$, т.е. если $(a, b] \subset \bigcup\limits_{n = 1}^\infty (a_n, b_n]$, то  \\ $\lambda_m (a, b] \leq \sum\limits_{n = 1}^\infty \lambda_m (a_n, b_n]$

Возьмем $a' \in (a, b]$, т.ч. $\lambda_m (a', b] > \lambda_m (a, b] - \varepsilon \Rightarrow [a', b] \subset (a, b]$

Возьмем $b_n'$, т.ч. $(a_n, b_n] \subset (a_n, b_n')$ и $\lambda_m (a_n, b_n'] < \lambda_m (a_n, b_n] + \frac{\varepsilon}{2^n}$

$\underbrace{[a', b]}_\text{замкн. паралл. -- компакт} \subset (a, b] \subset \bigcup\limits_{n = 1}^\infty (a_n, b_n] \subset \bigcup\limits_{n = 1}^\infty \underbrace{(a_n, b_n')}_\text{откр. паралл. -- откр. мн-ва}$

Выберем конечное подпокрытие $(a', b] \subset [a', b] \subset \bigcup\limits_{n = 1}^N (a_n, b_n') \subset \bigcup\limits_{n = 1}^N (a_n, b_n']$

По конечной полуаддитивности объема:

$\lambda_m (a, b] - \varepsilon < \lambda_m (a', b] \leq \sum\limits_{n = 1}^N \lambda_m (a_n, b_n'] < \sum\limits_{n = 1}^N (\lambda_m (a_n, b_n] + \frac{\varepsilon}{2^n}) < \varepsilon + \sum\limits_{n = 1}^\infty \lambda_m (a_n, b_n]$ и устремляем $\varepsilon$ к $0$

\begin{defin}{Мера Лебега}
    Мера Лебега -- стандартное продолжение классического объема
\end{defin}

\begin{declar}{Обозначение}
    $\L^m$ -- $\sigma$-алгебра, на которую продолжили

    Лебеговская $\sigma$-алгебра
\end{declar}

\begin{Remark}{}
    \begin{enumerate}
        \item Если $A \in \L^m$, то $\lambda_m A = \inf \{\sum\limits_{k = 1}^\infty \lambda_m P_k : A \subset \bigcup\limits_{k = 1}^\infty P_k \text{ и } P_k \text{ -- ячейки} \}$
        \item Можно брать ячейки из $\P^m_\Q$
    \end{enumerate}
\end{Remark}

\begin{theo}{Свойства меры Лебега}
    \begin{enumerate}
        \item Открытые множества измеримы и меры непустого открытого $> 0$
        \item Замкнутые множества измеримы 
        \item Мера одноточечного множества равна 0
        \item Мера ограниченного измеримого множества конечна
        \item Всякое измеримое множество -- счетное объединение множеств конечной меры
        
        Картинка! $\R^m = \bigsqcup\limits_{k = 1}^\infty P_k$, $P_k$ -- единичные ячейки. $A = \bigsqcup\limits_{k = 1}^\infty (P_k \cap A)$ и $\lambda_m (P_k \cap A) \leq \\ \leq \lambda_m P_k = 1$

        \item Пусть $E \subset \R^m : \forall \varepsilon > 0$ найдутся $A_\varepsilon, B_\varepsilon \in \L^m$, т.ч. $A_\varepsilon \subset E \subset B_\varepsilon$ и $\lambda_m (B_\varepsilon \setminus A_\varepsilon) < \varepsilon$. Тогда $E \in \L^m$
        
        \begin{Remark}{}
            Это свойство любой полной меры
        \end{Remark}

        \item Пусть $e \subset \R^m$, т.ч. $\forall \varepsilon > 0$ найдется $B_\varepsilon \in \L^m$, т.ч. $e \subset B_\varepsilon$ и $\lambda_m B_\varepsilon < \varepsilon$
        
        Тогда $E \in \L^m$ и $\lambda_m e = 0$

        \item Счетное объединение множеств нулевой меры -- множество нулевой меры 
        \item Счетное множество имеет нулевую меру 
        \item Множество нулевой меры не имеет внутренних точек 
        \item $\lambda_m e = 0$ и $\varepsilon > 0$. Тогда найдутся кубические ячейки $Q_k$, т.ч. $e \subset \bigcup\limits_{k = 1}^\infty Q_k$ и $\sum\limits_{k = 1}^\infty \lambda_m Q_k < \varepsilon$
        \item Пусть $m \geq 2$. $H_k(c) = \{x \in \R^m : x_k = c\}$. Тогда $\lambda_m(H_k(c)) = 0$
        \item Пусть $m \geq 2$. Множество, содержащееся в нбчс объединении гиперплоскостей $H_k(c)$, имеет меру 0
        \item $\lambda_m (a, b] = \lambda_m (a, b) = \lambda_m [a, b]$
    \end{enumerate}
\end{theo}

\textit{Доказательство:}

\begin{enumerate}
    \item[1. ] Открытые множества лежат в $\B(\P^m)$. Картинка на записи! $\lambda_m \delta > \lambda_m(\text{ячейка}) > 0$
    \item[3. ] Картинка! $\lambda_m(\text{точка}) < \lambda_m(\text{ячейка}) = \varepsilon^m$
    \item[4. ] Картинка! $A$ -- ограничено. $\lambda_m A \leq \lambda_m(\text{шар}) \leq \lambda_m(\text{ячейка}) < + \infty$
    \item[6. ] $A_\frac{1}{n} \subset E \subset B_\frac{1}{n};\ \lambda_m (B_\frac{1}{n} \setminus A_\frac{1}{n}) < \frac{1}{n}$

        $A := \bigcup\limits_{n = 1}^\infty A_\frac{1}{n} \in \L^m$ и $B := \bigcap\limits_{n = 1}^\infty B_\frac{1}{n} \in \L^m$

        $B \setminus A \subset B_\frac{1}{n} \setminus A_\frac{1}{n};\ \lambda_m (B \setminus A) \leq \lambda_m (B_\frac{1}{n} \setminus A_\frac{1}{n}) < \frac{1}{n} \Rightarrow \lambda_m (B \setminus A) = 0$

        Тогда т.к. $E \setminus A \subset B \setminus A \Rightarrow E \setminus A \in \L^m$

        Тогда $E = \underbrace{A}_{\in \L^m} \cup \underbrace{(E \setminus A)}_{\in \L^m}$
    \item[7. ] $A_\varepsilon = \varnothing$ в свойстве 6
    \item[10. ] От противного. Если $a$ -- внутренняя точка $A$. Рисунок! $\Rightarrow \lambda_m A \geq \lambda_m(\text{ячейка}) > 0$
    \item[11. ] $0 = \lambda_m e = \inf \{\sum\limits_{k = 1}^\infty \lambda_m P_k : e \subset \bigcup\limits_{k = 1}^\infty P_k \text{ и } P_k \in \P^m_\Q\}$

        Возьмем такие $P_k \in \P^m_\Q$, что $e \subset \bigcup\limits_{k = 1}^\infty P_k$ и $\sum\limits_{k = 1}^\infty \lambda_m P_k < \varepsilon$

        Рассмотрим $P_k$, у нее все стороны имеют рациональную длину. $d = \frac{1}{\text{НОК знаменателей}}$

        $\Rightarrow$ каждая сторона кратна $d \Rightarrow$ нарежем $P_k$ на кубики со стороной $d$

    \item[12. ] $A_n := (-n, n]^m \cap H_k(c) \Rightarrow H_k(c) = \bigcup\limits_{n = 1}^\infty A_n$
    
    Достаточно доказать, что $\lambda_n A_n = 0$. $A_n \subset (-n, n] \times \ldots \times (-n, n] \times (c - \varepsilon, c] \times (-n, n] \times \ldots$

    $\lambda_m(\text{ячейка}) = (2n)^{m - 1} \cdot \varepsilon$
\end{enumerate}

\begin{Remark}{}
    \begin{enumerate}
        \item Существуют несчетные множество нулевой меры
        
        При $m \geq 2$ подойдет $H_1(0)$

        При $m = 1$ подойдет \href{https://ru.wikipedia.org/wiki/%D0%9A%D0%B0%D0%BD%D1%82%D0%BE%D1%80%D0%BE%D0%B2%D0%BE_%D0%BC%D0%BD%D0%BE%D0%B6%D0%B5%D1%81%D1%82%D0%B2%D0%BE}{Канторово множество}: 
        
        $1 = \lambda (0, 1] = \lambda K + \underbrace{\frac{1}{3} + 2 \cdot \frac{1}{9} + 4 \cdot \frac{1}{27} + \ldots + 2^n \cdot \frac{1}{3^{n + 1}}}_{\frac{1}{3} \cdot \frac{1}{1 - \frac{2}{3}} = 1} \Rightarrow \lambda K = 0$

        $(0, 1]$ запишем в троичной системе счисления. Запрещаем запись $\ldots \underbrace{000\ \ \ \ \ \ }_\text{нули}$

        Т.к. $0, 2000\ldots = 0,1222\ldots$

        $\underset{\frac{1}{3}}{(}\ \underset{\frac{2}{3}}{]}$ -- числа, у которых первая цифра после запятой -- 1

        $\underset{\frac{1}{9}}{(}\ \underset{\frac{2}{9}}{]}$ и $\underset{\frac{7}{9}}{(}\ \underset{\frac{8}{9}}{]}$ -- числа, у которых вторая цифра после запятой -- 1

        И так далее 

        $K$ -- числа из $(0, 1]$, у которых в троичной записи нет 1. Биекция между $K$ и $(0, 1]$:
        
        $0 \mapsto 0;\ 2 \mapsto 1$; троичная $\mapsto$ двоичная

        \item Существуют неизмеримые множества (т.е. $\L^m \neq 2^{\R^m})$
    \end{enumerate}
\end{Remark}

\begin{theo}{Регулярность меры Лебега}
    $E \in \L^m$. Тогда существует $G$ -- открытое, $G \supset E$, т.ч. $\lambda_m (G \setminus E) < \varepsilon$
\end{theo}

\textit{Доказательство:}

\begin{itemize}
    \item[$\lambda_m E < + \infty$. ] $\lambda_m E = \inf \{\sum\limits_{k = 1}^\infty \lambda_n P_k : P_k \text{ -- ячейки и } E \subset \bigcup\limits_{k = 1}^\infty P_k\}$
    
    Возьмем такие ячейки, что $\sum\limits_{k = 1}^\infty \lambda_m P_k < \lambda_m E + \varepsilon$ и $E \subset \bigcup\limits_{k = 1}^\infty P_k$

    Возьмем $(a_k, b_k) \supset P_k$, т.ч. $\lambda_m (a_k, b_k) < \lambda_m P_k + \frac{\varepsilon}{2^k}$

    $E \subset G := \bigcup\limits_{k = 1}^\infty (a_k, b_k)$ -- открытое

    $\lambda_m G \leq \sum\limits_{k = 1}^\infty \lambda_m (a_k, b_k) \leq \sum\limits_{k = 1}^\infty (\lambda_m P_k + \frac{\varepsilon}{2^k}) = \varepsilon + \sum\limits_{k = 1}^\infty \lambda_m P_k < 2\varepsilon + \lambda_m E$

    $\lambda_m (G \setminus E) = \lambda_m G - \lambda_m E < 2 \varepsilon$

    \item[$\lambda_m E = + \infty$. ] $E = \bigcup\limits_{n = 1}^\infty E_n$, т.ч. $\lambda_m E_n < + \infty$
    
    По предыдущему случаю $\exists G_n$ -- открытое, $G_n \supset E_n$ и $\lambda_m (G_n \setminus E_n) < \frac{\varepsilon}{2^n}$

    $G := \bigcup\limits_{n = 1}^\infty G_n$ -- открытое 

    $G \setminus E \subset \bigcup\limits_{n = 1}^\infty G_n \setminus E_n \Rightarrow \lambda_m (G \setminus E) \leq \sum\limits_{n = 1}^\infty \lambda_m (G_n \setminus E_n) < \sum\limits_{n = 1}^\infty \frac{\varepsilon}{2^n} = \varepsilon$
\end{itemize}

\begin{theo}{Следствие 1}
    $\varepsilon > 0,\ E \in \L^m$. Тогда существует замкнутое $F$, т.ч. $F \subset E$ и $\lambda_m (E \setminus F) < \varepsilon$
\end{theo}

\textit{Доказательство:}

По теореме найдется $G$ -- открытое, т.ч. $G \supset \R^m \setminus E$ и $\lambda_m (G \setminus (\R^m \setminus E)) < \varepsilon \Rightarrow \\
\Rightarrow F := \R^m \setminus G$ -- замкнутое, $F \subset E$ и $E \setminus F = G \setminus (\R^m \setminus E)$

\begin{theo}{Cледствие 2}
    $E \in \L^m$. Тогда 
        
    $\lambda_m E = \inf \{\lambda_m G : G \text{ -- открытое и } E \subset G\}$

    $\lambda_m E = \sup \{\lambda_m F : F \text{ -- замкнутое и } E \supset F\}$

    $\lambda_m E = \sup \{\lambda_m K : K \text{ -- компакт и } K \subset E\}$
\end{theo}

\textit{Доказательство:}

\begin{enumerate}
    \item Из теоремы $\Rightarrow \exists G \supset E$ -- открытое, т.ч. $\lambda_m (G \setminus E) < \varepsilon \Rightarrow \lambda_m G < \lambda_m E + \varepsilon$
    \item Если $\lambda_m E < + \infty$, то по следствию 1 $\exists F \subset E$ -- замкнутое, т.ч. $\lambda_m (E \setminus F) < \varepsilon \Rightarrow \lambda_m F > \lambda_m E - \varepsilon$
    
    Если $\lambda_m E = + \infty \ldots \ldots \Rightarrow \lambda_m F = + \infty$

    \item Выберем замкнутое $F \subset E$, т.ч. $\lambda_m F > \lambda_m E - \varepsilon$
    
    $K_n := \underbrace{[-n, n]^m}_\text{компакт} \cap F$

    $K_1 \subset K_2 \subset \ldots$ и $\bigcup\limits_{n = 1}^\infty K_n = F \xRightarrow[\text{непр. меры снизу}]{} \lambda_m K_n \to \lambda_m F > \lambda_m E - \varepsilon \Rightarrow$ найдется $K_n$, т.ч. $\lambda_m K_n > \lambda_m F - \varepsilon$

    В случае с $\lambda_m E = + \infty$ доказательство меняется несильно
\end{enumerate}

\begin{theo}{Следствие 3}
    $E \in \L^m$. Тогда существуют компакты $K_1 \subset K_2 \subset \ldots$ и $e$ нулевой меры, т.ч. $E = e \sqcup \bigcup\limits_{n = 1}^\infty K_n$
\end{theo}

\textit{Доказательство:}

\begin{itemize}
    \item[$\lambda_m E < + \infty$. ] Возьмем $K_n \subset E$ -- компакт, т.ч. $\lambda_m K_n > \lambda_m E - \frac{1}{n}$
    
    $\bigcup\limits_{n = 1}^\infty K_n \subset E$ и $\underbrace{E \setminus \bigcup\limits_{n = 1}^\infty K_n}_{e} \subset E \setminus K_n \Rightarrow \lambda_m e < \lambda_m (E \setminus K) = \lambda_m E - \lambda_m K_n < \frac{1}{n} \Rightarrow \lambda_m e = 0$

    Как сделать вложенность? $K_1, K_1 \cup K_2, K_1 \cup K_2 \cup K_3, \ldots$

    \item[$\lambda_m E = + \infty$. ] $E = \bigsqcup\limits_{n = 1}^\infty E_n;\ \lambda_m E_n < + \infty$. Тогда $\exists K_{n1}, K_{n2} \ldots$ -- компакты и $\lambda_m e_n = 0$, \\
    т.ч. $E_n = e_n \sqcup \bigcup\limits_{k = 1}^\infty K_{nk} \Rightarrow E = \underbrace{\bigcup\limits_{n = 1}^\infty e_n}_{e} \sqcup \bigcup\limits_{n = 1}^\infty \bigcup\limits_{k = 1}^\infty K_{nk}$
\end{itemize}

\end{document}