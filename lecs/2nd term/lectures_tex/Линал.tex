\documentclass[12pt]{article}
\usepackage{config}
\usepackage{subfiles}

\def\multiset#1#2{\ensuremath{\left(\kern-.3em\left(\genfrac(){0pt}{}{#1}{#2}\right)\kern-.3em\right)}}
\def\divby{%
  \mathrel{\text{\vbox{\baselineskip.65ex\lineskiplimit0pt\hbox{.}\hbox{.}\hbox{.}}}}%
}
\newcommand{\q}[1]{\langle #1 \rangle}

\begin{document}

\begin{flushright}
    Конспект Шорохова Сергея

    Если нашли опечатку/ошибку - пишите @le9endwp
\end{flushright}

\tableofcontents
\newpage

\section{Линейная алгебра и геометрия}

Типичная система линейных уравнений: $\left\{\begin{array}{l}
    ax + by = e \\
    cx + dy = f
\end{array}\right.;\ a, b, c, d, e, f \in R$ -- кольцо или $\in K$ -- поле

Неизвестные здесь: ${x \choose y} \in K \times K$

Множество линейных уравнений: $\{ px + qy = r \}$

Операции:

\begin{itemize}
    \item Их можно складывать
    \item Умножать на константу (элемент $K$)
\end{itemize}

\vspace{5mm}

\begin{defin}{Векторное пространство}
    $K$ -- поле. Векторное пространство над $K$ это $(V, +, \cdot)$, где V -- множество, $+: V \times V \rightarrow V$, $\cdot: K \times V \rightarrow V$
\end{defin}

\vspace{5mm}

\textbf{Аксиомы:}

\begin{enumerate}
    \item[1-4.] $(V, +)$ -- абелева группа
    \item[5.] $(ab)v = a(bv)\ \forall a, b \in K, v \in V$
    \item[6.] $(a + b)v = av + bv\ \forall a, b \in K, v \in V$
    \item[7.] $a(v + u) = av + au\ \forall a \in K, v, u \in V$
    \item[8.] $1v = v\ \forall v \in V$
\end{enumerate}

\vspace{5mm}

\begin{lem}{}
    $0 \cdot v = \overrightarrow{0}\ \forall v \in V$

    $(-1) \cdot v = -v\ \forall v \in V$
\end{lem}

\textit{Доказательство:}

$(0 + 0)v = 0v + 0v \Rightarrow 0v = 0v + 0v$

$(-0)v + 0v = (-0)v + 0v + 0v \Rightarrow \overrightarrow{0} = 0v$

Тогда $\overrightarrow{0} = 0v = (1 + (-1))v = 1v + (-1)v = v + (-1)v$, т.е. $v + (-1)v = \overrightarrow{0} \Rightarrow (-1)v = -v$

\begin{Remark}{}
    $u + v = v + u\ \forall u, v \in V$ следует из остальных 7 аксиом пространства (упражнение)
\end{Remark}

\begin{Example}{}
    Тут рисуночки, говорящие что два вектора задают пространство, в котором выполнены аксиомы 1-8

    Заметим, что есть биекция $vec \leftrightarrow R^2$, т.е. $v \rightarrow {a \choose b}$
\end{Example}

\begin{Example}{Самый главный пример}
    $K^n = \{\left( \begin{gathered}
        a_1 \\
        a_2 \\
        \vdots \\
        a_n
    \end{gathered} \right) | a_i \in K\}$

    А еще тут выполнены все аксиомы (доказано методом очев): можем складывать, домножать итд

    Это называем пространство столбцов

    \vspace{3mm}

    $^nK = \{ (a_1, a_2 \ldots a_n) | a_i \in K\}$

    А это то же самое, но называем пространством строк
\end{Example}

\vspace{5mm}

\begin{defin}{Линейное отображение}
    $V_1, V_2$ -- векторные пространства над $K$

    $f : V_1 \rightarrow V_2$ -- линейное отображение (гомоморфизм), если:

    \begin{enumerate}
        \item $f(v_1 + v_2) = f(v_1) + f(v_2)\ \forall v_1, v_2 \in V_1$
        \item $f(kv) = kf(v)\ \forall k \in K, v \in V_1$
    \end{enumerate}
\end{defin}

\begin{defin}{Изоморфизм}
    $f$ -- линейное отображение и биекция, тогда $f$ -- изоморфизм

    $V_1 \cong V_2$ если существует изоморфизм $V_1 \rightarrow V_2$

    А есть изоморфизм $vect_2 \cong R^2$, то есть вектор изоморфен его координатам 
\end{defin}

\begin{Example}{}
    $M$ -- множество, $R \equiv K$

    $V = HOM(M | R)$ -- множество всех функций $M \rightarrow R$

    $f_1, f_2 \in V$

    $(f_1 + f_2)(x) := f_1(x) + f_2(x)$

    $(kf)(x) := k \cdot f(x)$

    Значит $V$ -- векторное пространство

\begin{Example}{}
    $M = \{x_1, x_2 \ldots x_n\}$

    $f \in V \leftrightarrow \left( \begin{gathered}
        f(x_1) \\
        f(x_2) \\
        \vdots \\
        f(x_n)
    \end{gathered} \right) \in R^n$

    $V \cong R^n$

    $M = [0, 1];\ (f : M \rightarrow R$ -- непрерывная функция$)$
\end{Example}

\end{Example}

\begin{Example}{}
    $V = \{(a_1, a_2 \ldots) | a_i \in R;\ a_{i + 2} = a_i + a_{i + 1}\}$

    Заметим, что если $a \in V$, то $ka \in V$. Более того, если и $b \in V$, то $a + b \in V$

    Но любую фиббоначиеву последовательность можно задать двумя начальными элементами, т.е. $(a_i) \in V \leftrightarrow (a_1, a_2) \in R^2$

    Тогда $V \cong R^2$ но этот изоморфизм не лучший
\end{Example}

\begin{Example}{}
    $M$ -- множество, $V = 2^M$

    \begin{enumerate}
        \item $|M| = n;$
        \item $A + B = (A \cup B) \setminus (A \cap B)$
        \item $K = Z/2Z$
        \item $0A = \varnothing$
        \item $1A = A$
    \end{enumerate}

    $1A + 1A = 2A \Rightarrow 1A + 1A = \varnothing$

    $2A = \overrightarrow{0}\ \forall A$
\end{Example}

\begin{defin}{Линейная комбинация}
    $V$ -- векторное пространство над $K$

    $x_1 \ldots x_n \in V;\ a_1 \ldots a_n \in K$

    Тогда $a_1x_1 + a_2x_2 + \ldots + a_nx_n$ -- линейная комбинация векторов $x_1 \ldots x_n$ с коэффициентами $a_1 \ldots a_n$
\end{defin}

\begin{defin}{Подпространство}
    $V$ -- векторное пространство над $K$. $U \subseteq V$ 

    $U$ -- подпространство $V$, если $U$ -- векторное пространство над $K$ с теми же операциями
\end{defin}

\begin{Remark}{}
    $U$ -- подпространство $V \Leftrightarrow$

    \begin{enumerate}
        \item $\forall u_1, u_2 \in U \Rightarrow u_1 + u_2 \in U$
        \item $\forall u \in U, k \in K \Rightarrow ku \in U$
    \end{enumerate}

    Где $U \neq \varnothing$
\end{Remark}

\begin{Example}{}
    $U = \{ V \parallel l \}$ -- подпространство $V$

    $K^3$, $U \subset K^3$

    $U = \{ (x, y, z) | x + y + z = 0 \}$ -- подпространство $K^3$
\end{Example}

\begin{defin}{Линейная оболочка}
    $V$ -- векторное пространство над $K$

    $V_1, \ldots V_n \in V$

    Линейная оболочка $\langle V_1, \ldots V_n \rangle$ -- их множество линейных комбинаций с произвольными коэффициентами

    $\langle V_1, \ldots V_n \rangle = \{ a_1V_1 + \ldots + a_nV_n | a_i \in K \}$
\end{defin}

\begin{Remark}{}
    \begin{enumerate}
        \item $\langle V_1, \ldots V_n \rangle$ -- подпространство $V$
        
        $\langle V_1, \ldots V_n \rangle < V$
    
        \item $U < V;\ V_1 \ldots V_n \in U \Rightarrow \langle V_1, \ldots V_n \rangle \subset U$
    \end{enumerate}
    
    Т.е. $\langle V_1, \ldots V_n \rangle$ -- нелинейное подпространство содержит $V_1 \ldots V_n$
\end{Remark}

\textit{Доказательство:}

$V_i = 0 V_1 + \ldots + 1V_i + \ldots + 0V_n \Rightarrow V_i \in \langle V_1, \ldots V_n \rangle$

$u, w \in \langle V_1, \ldots V_n \rangle$

$ku + w \in \langle V_1, \ldots V_n \rangle$

\vspace{2mm}

$U < V\ V_i \in U \Rightarrow a_iV_i \in U$

$a_1V_1 \ldots a_nV_n \in U \Rightarrow a_1V_1 + \ldots + a_nV_n \in U$

Т.е. $U$ содержит все линейные комбинации $V_1 \ldots V_n$

\begin{Remark}{}
    Аналогично определяется линейная оболочка для любого числа векторов
\end{Remark}

\begin{defin}{Порождающая система}
    $M$ называется порождающей системой в $V$, если $\langle M \rangle = V$, т.е. $\forall v \in V$ -- линейная комбинация векторов из $M$
\end{defin}

\begin{defin}{Конечномерные пространства}
    $V$ -- векторное пространство над $K$

    $V$ называется конечномерным, если $\exists$ конечная порождающая система. Будем изучать конечномерные пространства
\end{defin}

\begin{lem}{}
    $\q{V_1 \ldots V_n}$

    $\q{V_1 + \sum\limits_2^n a_iV_i, V_2 \ldots V_n} = \q{V_1, V_2 \ldots V_n}$
\end{lem}

\textit{Доказательство:}

$V_1 + \sum\limits_2^n a_iV_i \in \q{V_1, V_2 \ldots V_n}$ и $V_2 \ldots V_n \in \q{V_1 \ldots V_n}$

Тогда $\q{V_1 + \sum\limits_2^n a_iV_i, V_2 \ldots V_n} = \q{V_1, V_2 \ldots V_n}$ по Rem2.

\begin{defin}{Линейная независимость}
    $M \subset V$

    $M$ называется линейно независимым, если $\forall v_1 \ldots v_n \in M$ и $\forall a_1 \ldots a_n \in K$ : $\sum a_iv_i = 0 \Rightarrow a_1 = \ldots = a_n = 0$

    Т.е. никакая линейнай комбинация элементов $M$ не равна 0
\end{defin}

\begin{propos}{}
    $v_1 \ldots v_n \in V$

    Тогда $v_1 \ldots v_n$ -- линейно зависимы (не линейно независимы) $\Leftrightarrow \exists i : v_i \in \q{v_1 \ldots v_{i - 1}, v_{i + 1} \ldots v_n}$

    $v_i = \sum\limits_{j \neq i} a_jv_j$

    $(-1)v_i + \sum\limits_{j \neq i} a_jv_j = \overrightarrow{0}$ -- нетривиальная линейная комбинация 

    \vspace{2mm}

    Пусть $\sum a_iv_i = 0$ -- нетривиальная линейная комбинация

    $\exists i : a_i \neq 0$

    $-a_iv_i = \sum\limits_{j \neq i} a_jv_j \Rightarrow v_i = \sum\limits_{j \neq i} -\frac{a_j}{a_i}v_j$

$v_i \in \q{v_j}$
\end{propos}

\begin{Remark}{}
    $K$ не поле (ассоциативное кольцо)

    $V$ над $k$ (с теми эе операциями) называется модулем над $K$. Для модулей это утверждение (и большинство других) неверно
\end{Remark}

\begin{defin}{Базис}
    $V$ -- векторное пространство над $K$

    $v_1 \ldots v_n$ -- базис $V$, если это порождающая система и линейно независима
\end{defin}

\begin{defin}{Размерность}
    $V$ -- конечномерное векторное пространство. Мощность его базиса называется размерностью $V$ и обозначается $dim(V)$
\end{defin}

\begin{Example}{}
    $dim(K^n) = n$

    Базис стандартный $e_1 = \left( \begin{gathered}
        1 \\
        0 \\
        \vdots \\
        0
    \end{gathered} \right)$ итд
\end{Example}

\begin{defin}{}
    $a_1 \ldots a_n$ -- координаты вектора $v$ в базисе $v_1 \ldots v_n$
\end{defin}

\begin{theo}{}
    Следующие условия равносильны:

    \begin{enumerate}
        \item $v_1 \ldots v_n$ -- базис $V$
        \item $v_1 \ldots v_n$ -- порождающая линейно независимая система
        \item $v_1 \ldots v_n$ -- максимальная по включению линейно независимая система
        \item $\forall v \in V\ \exists! a_1 \ldots a_n : v = \sum a_iv_i$
    \end{enumerate}
\end{theo}



\begin{theo}{}
    $V$ -- конечное векторное пространство

    \begin{enumerate}
        \item Базисы существуют
        \item Любые два базиса равномощны
    \end{enumerate}
\end{theo}

\textit{Доказательство:}

\begin{enumerate}
    \item[$1 \Rightarrow 2$] $v_1 \ldots v_n$ -- базис $\Rightarrow v_1 \ldots v_n$ -- порождающая система
    
    Почему лнз? $a_1v_1 + \ldots + a_nv_n = 0$ и $\exists a_i \neq 0 \Rightarrow v_i = \sum\limits_{j \neq i} c_jv_j \Rightarrow$

    $\Rightarrow \q{v_1 \ldots v_n} = \q{v_1 \ldots v_{i - 1}, v_{i + 1} \ldots v_n}$

    \item[$2 \Rightarrow 1$] $v_1 \ldots v_n$ лнз
    
    Пусть не минимальная порождающая. НУО $v_2 \ldots v_n$ -- порождающая система, в частности $v_1 = \sum a_iv_i \Rightarrow v_1 \ldots v_n$ -- линейно зависимая 

    \item[$2 \Rightarrow 4$] $v_1 \ldots v_n$ -- порождающая лнз
    
    Т.к. порождающая $\forall v = \sum a_iv_i$

    Единственность: пусть $\sum a_iv_i = \sum a_i'v_i : \sum (a_i - a_i')v_i = 0 \Rightarrow a_i = a_i'\ \ \forall i$

    \item[$4 \Rightarrow 2$] $\forall v \exists a_i : v = \sum a_iv_i$, т.е. $v_1 \ldots v_n$ -- порождающая
    
    Лнз-ть: пусть $v_i = \sum\limits_{j \neq i} a_jv_j$. Тогда $v_i = 0 \cdot v_1 + 0 \cdot v_2 + \ldots + 1 \cdot v_i + \ldots + 0 \cdot v_n = \\ = 0 \cdot v_1 + \ldots + 0 \cdot v_{i - 1} + 1 \cdot v_i + 0 \cdot v_{i + 1} + \ldots + 0 \cdot v_n$
\end{enumerate}

\begin{Exercise}{}
    $2 \Leftrightarrow 3$
\end{Exercise}

\begin{lem}{Линейная зависимости линейных комбинаций}
    $V$ -- векторное пространство над $K$

    $v_1 \ldots v_n \in \q{u_1 \ldots u_m};\ n > m$

    Тогда $v_1 \ldots v_n$ -- линейно зависимы
\end{lem}

\textit{Доказательство:}

ММИ по $m$. База $m = 1$

$\begin{cases}
    v_1 = a_1u_1 \\
    v_2 = a_2u_1 \\
    \ldots
\end{cases}$

$a_2v_1 - a_1v_2 = 0$. Либо $v_1, v_2$ -- линейно зависимы, либо $a_1, a_2 = 0 \Rightarrow v_1 = \overline{0} = v_2$

$1 \cdot v_1 + 1 \cdot v_2 + 0 \cdot v_3 \ldots = 0 \Rightarrow v_1 \ldots v_n$ -- линейно зависимы

Переход: $m \rightarrow m + 1$

$\begin{cases}
    v_1 = a_{1_1}u_1 + \ldots + a_{1_{m + 1}}u_{m + 1} \\
    v_2 = a_{2_1}u_1 + \ldots + a_{2_{m + 1}}u_{m + 1} \\
    \ldots \\
    v_n = a_{n_1}u_1 + \ldots + a_{n_{m + 1}}u_{m + 1}
\end{cases}$

\begin{enumerate}
    \item $a_{1_{m + 1}} = a_{2_{m + 1}} = \ldots = a_{n_{m + 1}} = 0$

    $v_1 \ldots v_n \in \q{u_1 \ldots u_m}$

    $n > m + 1 \Rightarrow n > m \Rightarrow v_1 \ldots v_n$ -- линейно зависимы

    \item НУО $a_{1_{m + 1}} \neq 0$
    
    Вычтем из $i$ равенства ($i = 2 \ldots n$) первое умноженное на $\frac{a_{i_{m + 1}}}{a_{1_{m + 1}}}$

    Тогда $\tilde{v_i} = v_i - \frac{a_{i_{m + 1}}}{a_{1_{m + 1}}}v_1 = \sum\limits_{k = 1}^{m + 1} (a_{i_k} - \frac{a_{i_{m + 1}}}{a_{1_{m + 1}}}a_{1_k})u_k \in \q{u_1 \ldots u_m}$

    $\tilde{v_2} \ldots \tilde{v_n} \in \q{u_1 \ldots u_m}$, но $n > m + 1 \Rightarrow n - 1 > m \Rightarrow \tilde{v_2} \ldots \tilde{v_n}$ -- линейно зависимы

    $\exists a_1 \ldots a_n$ -- не все нули:

    $0 = \sum a_i \tilde{v_i} = \sum a_i(v_i - \frac{a_{i_{m + 1}}}{a_{1_{m + 1}}}v_1) = \sum a_iv_i + (\ldots)v_1 \Rightarrow v_1 \ldots v_n$ -- линейно зависимы
\end{enumerate}

\begin{theo}{Следствие}
    $v_1 \ldots v_n$ -- базис и $u_1 \ldots u_m$ -- базис $\Rightarrow n = m$ (теорема часть 2)
\end{theo}

\textit{Доказательство:}

Пусть НУО $n > m$

$u_1 \ldots u_m$ -- базис $\Rightarrow$ порождающая $\Rightarrow v_1 \ldots v_n \in \q{u_1 \ldots u_m};\ \ n > m \Rightarrow v_1 \ldots v_n$ -- линейно зависимы ???

\begin{enumerate}
    \item $v_1 \ldots v_s$ -- порождающая система (существует, т.к. $V$ конечномерно)
    
    Пусть $v_1 \ldots v_s$ -- линейно зависимы

    $\exists i : v_i \in \q{v_j};\ v_i = \sum\limits_{j \neq i} a_jv_j$

    НУО $i = 1$

    Тогда $\q{v_1 \ldots v_n} = \q{v_1 - \sum\limits_{j \neq 1} a_jv_j, v_2 \ldots v_n} = \q{v_2 \ldots v_n}$

    $v_2 \ldots v_n$ -- порождающая система. Продолжаем выкидывать $v_i$ пока не получим базис

    \item 
    
    \begin{Example}{За что мы боремся?}
        Векторные пространства $\rightarrow$ абелевы группы

        $Z = \q{1, 2, 3} = \q{1, 2} = \q{1}$

        С другой стороны $Z = \q{1, 2, 3} = \q{2, 3}$ -- минимальная порождающая система
    \end{Example}
    
    $dimV = n$, если $\exists$ базис $v_1 \ldots v_n \Leftrightarrow$ в любом базисе $n$ элементов

    \begin{lem}{}
        $V$ -- конечномерное пространство, $u_1 \ldots u_k$ -- линейно независимы $\Rightarrow \exists u_{k + 1} \ldots u_n : u_1 \ldots u_n$ -- базис
    \end{lem}

    \textit{Доказательство:}

    $u_1 \ldots u_k$ -- не максимальная лнз. $\exists u_{k + 1} : u_1 \ldots u_{k + 1}$ -- лнз

    $u_1 \ldots u_{k + 1}$ -- не максимальная лнз. $\exists u_{k + 2} : u_1 \ldots u_{k + 2}$ -- лнз итд

    Заметим: не может быть $u_1 \ldots u_{n + 1}$ -- лнз (по лзлк), $u_1 \ldots u_{n + 1} \in \q{v_1 \ldots v_n}$

    $\Rightarrow$ не позже $n$ шага процесс закончится. На самом деле ровно на $n$ шаге
\end{enumerate}

\begin{theo}{Следствие}
    $n = dimV$, $u_1 \ldots u_m \in V$

    $m > n \Rightarrow u_1 \ldots u_m$ -- линейно зависимы

    $m < n \Rightarrow u_1 \ldots u_m$ -- не порождающая система
\end{theo}

\begin{theo}{Следствие}
    $U \leq V$, тогда $dimU \leq dimV$ и $dimU = dimV \Leftrightarrow U = V$
\end{theo}

\begin{theo}{}
    $V$ -- $k$-мерное над $K$. $dimV = n \Rightarrow V \cong K^n$
\end{theo}

\textit{Доказательство:}

$v_1 \ldots v_n$ -- базис $V$. Рассмотрим отображение $p : K^n \rightarrow V$

$\left( \begin{gathered}
    a_1 \\
    a_2 \\
    \vdots \\
    a_n
\end{gathered} \right) \rightarrow a_1v_1 + a_2v_2 + \ldots + a_nv_n$

$f(x + y) = f(\left( \begin{gathered}
    a_1 \\
    a_2 \\
    \vdots \\
    a_n
\end{gathered} \right) + \left( \begin{gathered}
    b_1 \\
    b_2 \\
    \vdots \\
    b_n)
\end{gathered} \right) ) = f(\left( \begin{gathered} 
    a_1 + b_1 \\
    a_2 + b_2 \\
    \vdots \\
    a_n + b_n
\end{gathered} \right)) = \sum (a_i + b_i)v_i = \sum a_iv_i + \sum b_iv_i = f(x) + f(y)$

\begin{Exercise}{}
    $f(kx) = kf(x)$
\end{Exercise}

$f$ -- сюръективно и инъективно: по определению базиса

\begin{Example}{}
    $v = \{ f \in K[x] | deg(f) \leq 2 \} = \q{1, x, x^2} = \q{1, 1 + x, x^2}$ -- оба базисы
\end{Example}

\begin{Example}{Числа Фиббоначи}
    $V = \{ (a_1 \ldots) | a_{i + 1} = a_i + a_{i - 1} \}$

    $V \leftrightarrow (a_1, a_2)$, $V \cong R^2$

    Хороший базис:

    $\varphi_1 = (1, \varphi, \varphi^2 \ldots) \in V$

    $\varphi_2 = (1, (-\frac{1}{\varphi}), (-\frac{1}{\varphi})^2 \ldots) \in V$

    $\varphi_1, \varphi_2$ -- базис

    $f = a\varphi_1 + b\varphi_2$

    $f \rightarrow u_n = a \cdot \varphi^n + v(-\frac{1}{\varphi})^n$

    $a = b = \frac{1}{\sqrt{5}}$
\end{Example}

\section{Система линейных уравнений (СЛУ)}

\begin{defin}{Линейное уравнение}
    Линейное уравнение: $a_1x_1 \ldots a_nx_n = b$
    
    где $a_1 \ldots a_n, b \in K$, а $x_1 \ldots x_n$ -- переменные
\end{defin}

\begin{defin}{Система линейных уравнений}
    СЛУ -- это набор линейных уравнений: $\sum\limits_{i = 1}^n a_{k_i}x_i = b_k$, $k = 1 \ldots m$

    СЛУ соответствует отображение $A: K^n \rightarrow K^m$

    $\left( \begin{gathered}
        x_1 \\
        \vdots \\
        x_n
    \end{gathered} \right) \rightarrow \left( \begin{gathered}
        \sum a_{1_i}x_i \\
        \vdots \\
        \sum a_{m_i}x_i
    \end{gathered} \right)$

    Это отображение уважает сложение (просто поверьте), и вообще $A$ -- линейное отображение
\end{defin}

\begin{defin}{Ядро и образ}
    $A : U \rightarrow V$ -- линейное

    Ядро: $ker(A) = \{x \in U | A(x) = \overline{0} \} \subset U$

    $Im(A) = \{ A(x) | x \in U \} \subset V$

    \begin{Example}{}
        В нашем примере

        $Im(A) = \{ \left( \begin{gathered}
            b_1 \\
            \vdots \\
            b_n
        \end{gathered} \right) | \text{СЛУ} A(x) = \left( \begin{gathered}
            b_1 \\
            \vdots \\
            b_n
        \end{gathered} \right) \}$

        $Ker(A) = $ множество решений системы

        $\begin{cases}
            \sum a_{1_i}x_i = 0 \\
            \vdots \\
            \sum a_{m_i}x_i = 0
        \end{cases}$

        Такие системы называются однородными
    \end{Example}
\end{defin}

\begin{lem}{}
    $A : U \rightarrow V$ -- линейное отображение $\Rightarrow Ker(a) \leq U$ и $Im(A) \leq V$ -- подпространство
\end{lem}

\textit{Доказательство:}

\begin{enumerate}
    \item Надо проверить замкнутость

    $u_1, u_2 \in Ker(A)$, т.е. $A(u_1) = 0$ и $A(u_2) = 0$
    
    $A(u_1 + ku_2) = A(u_1) + kA(u_2) = 0 + 0 = 0$

    \item $v_1, v_2 \in Im(A)$, $v_1 = A(u_1)$ и $v_2 = A(u_2)$
    
    $v_1 + kv_2 = A(u_1) + kA(u_2) = A(u_1 + ku_2) = A(u) \Rightarrow v_1 + kv_2 \in Im(A)$
\end{enumerate}

\begin{propos}{}
    В нашем примере:

    Множество решений однородной линейной системы -- подпространство в $K^n$

    Тривиальный случай: $dim(Ker(A)) = 0$, т.е. $Ker(A) = \{\left( \begin{gathered}
        0 \\
        \vdots \\
        0
    \end{gathered} \right) \}$ -- всегда решение однородной СЛУ (есть только тривиальное решение)
\end{propos}

\begin{theo}{}
    В однородной СЛУ

    $n > m \Rightarrow dim(Ker(a)) > 1$, т.е.  существует нетривиальное решение СЛУ
\end{theo}

\textit{Доказательство:}

$\begin{cases}
    a_{11}x_1 + \ldots + a_{1n}x_n = 0 \\
    \vdots \\
    a_{m1}x_1 + \ldots + a_{mn}x_n = 0
\end{cases} \Leftrightarrow x_1 \cdot \left( \begin{gathered}
    a_{11} \\
    \vdots \\
    a_{m1}
\end{gathered} \right) + \ldots + x_n \cdot \left( \begin{gathered}
    a_{1n} \\
    \vdots \\
    a_{mn}
\end{gathered} \right) = \left( \begin{gathered}
    0 \\
    \vdots \\
    0
\end{gathered} \right)$

$u_1 \ldots u_n \in K^m;\ n > m \Rightarrow u_1 \ldots u_n$ -- лз, т.е. $\exists x_1 \ldots x_n$ -- не все нули: $\sum x_iu_i = 0$

$\A : K^n \rightarrow K^m$

$\A \left( \begin{gathered}
    x_1 \\
    x_2 \\
    \vdots \\
    x_n
\end{gathered} \right) = \left( \begin{gathered}
    \sum a_{1i}x_i \\
    \sum a_{2i}x_i \\
    \vdots \\
    \sum a_{mi}x_i
\end{gathered} \right)$

$\left( \begin{gathered}
    b_1 \\
    b_2 \\
    \vdots \\
    b_m
\end{gathered} \right) \in K^m$

Решить систему: найти $\A^{-1}$

$U, V$ -- векторные пространства. $A : U \rightarrow V$ линейное отображение

Как описать $A$?

\begin{lem}{}
    $U_1, U_2 \ldots U_n$ -- базис $U$ и $V_1, V_2 \ldots V_n \in V$

    Тогда $\exists!$ линейное отображение $A : U \rightarrow V : A(U_i) = V_i\ \ \forall i = 1 \ldots n$
\end{lem}

\textit{Доказательство:}

\begin{itemize}
    \item[Единственность:] Пусть $A_1(U_i) = V_i$ и $A_2(U_i) = V_i$
    
    $?A_1 = A_2 \Leftrightarrow A_1(u) = A_2(u)\ \ \forall u \in U$

    $u = \sum a_iu_i\ a_i \in K$

    Тогда $A_1(u) = A_1(a_1u_1 + a_2u_2 + \ldots + a_nu_n) = A_1(a_1u_1) + A_1(a_2u_2) + \ldots + A_1(a_nu_n) = a_1A_1(u_1) + a_2A_1(u_2) + \ldots + a_nA_1(u_n) = a_1V_1 + a_2V_2 + \ldots + a_nV_n$

    Аналогично для $A_2$

    Тогда $A_1(u) = A_2(u)\ \ \forall u \in U$

    \item[Существование:] Построим $A$. Рассмотрим какой-то $u \in U$
    
    $\exists! a_1, a_2 \ldots a_n : u = \sum a_iu_i$

    Положим $A(u) = \sum a_iV_i$

    $U_i = 0 \cdot u_1 + \ldots + 1 \cdot u_i + \ldots + 0 \cdot u_n \Rightarrow A(U_i) = 0 \cdot V_1 + \ldots + 1 \cdot V_i + \ldots + 0 \cdot V_n = V_i$

    $u = \sum a_iu_i;\ v = \sum a_iV_i$

    $A(u + v) = A(\sum a_iu_i + \sum b_iu_i) = A(\sum (a_i + b_i)u_i) = \sum (a_i + b_i)V_i = \sum a_iV_i + \sum b_iV_i = A(u) + A(v)$

    $A(kv) = kA(v)$ -- очев
\end{itemize}

\begin{defin}{Матрица линейного отображения в базисах}
    Итак. $u_1, u_2 \ldots u_n$ -- базис $U$

    Задать $A : U \rightarrow V \Leftrightarrow$ зафиксировать $A(u_1) \ldots A(u_n) \in V$

    $A : U \rightarrow V$ линейно $v_1, v_2 \ldots v_m$ -- базис $V$

    $A(u_1) = a_{11}v_1 + a_{21}v_2 + \ldots + a_{m1}v_m$

    \vdots

    $A(u_n) = a_{1n}v_1 + a_{2n}v_2 + \ldots + a_{mn}v_m$

    $A = \begin{pmatrix}
        a_11 & a_12 & \ldots & a_1n \\
        a_21 & a_22 & \ldots & a_2n \\
        \vdots & \vdots & \ddots & \vdots \\
        a_m1 & a_m2 & \ldots & a_mn
    \end{pmatrix}$ называется матрицей линейного отображения $A$ в базисах $\{u_i\}$ и $\{v_i\}$

    Обозначение: $[A]_{\{u_i\}, \{v_i\}}$ -- зависит от $\{u_i\}$ и $\{v_i\}$
\end{defin}

\begin{nota}{Итог}
    Такая матрица:

    \begin{itemize}
        \item Отображение $\{1 \ldots m\} \times \{1 \ldots n\} \rightarrow R$ -- $R$ кольцо
        \item Отображение $I \times I \rightarrow R$ $I, I$ -- конечные множества
    \end{itemize}

    Обозначение: $M_{m, n}(R)$ -- матрицы $m \times n$ над $R$
\end{nota}

Изоморфизм:

$K$ -- поле, $M_{1, n} \cong ^nK$ и $M_{n, 1} \cong K^n$

$M_{m, n}(K)$ -- векторное пространство над $K$

$(a_{ij}) + (b_{ij}) = (a_{ij} + b_{ij})_{i = 1 \ldots m; \\ j = 1 \ldots n}$

$k(a_{ij}) = (ka_{ij})_{i = 1 \ldots m; \\ j = 1 \ldots n}$

Операции:

$^nK \times K^n$

$((a_1a_2 \ldots a_n), \left( \begin{gathered}
    b_1 \\
    b_2 \\
    \vdots \\
    b_n
\end{gathered}\right)) \rightarrow \sum a_ib_i$ -- умножение строки на столбец

$M_{m, n} \times K^n \rightarrow K^m$ -- пример с прошлой лекции

\begin{theo}{Свойства:}
    $(A, X) \rightarrow AX$

    $A(X_1 + X_2) = AX_1 + AX_2$

    $(A_1 + A_2)X = A_1X + A_2X$

    $A(kX) = k(AX)$
\end{theo}

\textit{Доказательство:}

$m = 1$

$(a_1 + a_1')b_1 + \ldots + (a_n + a_n')b_n = \sum a_ib_i + \sum a_i'b_i$ и наоборот $\sum (ka_i)b_i = \sum a_i(kb_i) = k\sum a_ib_i$

В частности $A \in M_{m, n}$ -- fix

$X \rightarrow AX$ -- линейное отображение $K^n \rightarrow K^m$

\begin{lem}{}
    $A : U \rightarrow V$, $\{u_i\}$ -- базис $U$ и $\{v_i\}$ -- базис $V$

    $A = [A]_{\{u_i\}, \{v_i\}}$

    $u \in U;\ X = \begin{pmatrix}
        x_1 \\
        x_2 \\
        \vdots \\
        x_n
    \end{pmatrix}$ -- координаты $u$ в базисе $\{u_i\}$

    Тогда $AX$ -- координаты $A(u)$ в базисе $\{v_i\}$
\end{lem}

\textit{Доказательство:}

$u = \sum x_iu_i$

$A(u) = \sum x_iA(u_i) = \sum\limits_{i = 1}^n x_i(\sum\limits_{j = 1}^m a_{ji}v_j) = \sum\limits_{j = 1}^m (\sum\limits_{i = 1}^n a_{ji}x_i)v_j$

$\Rightarrow \begin{pmatrix}
    \sum\limits_{i = 1}^n a_{1i}x_i \\
    \sum\limits_{i = 1}^n a_{2i}x_i \\
    \vdots \\
    \sum\limits_{i = 1}^n a_{mi}x_i
\end{pmatrix}$ -- координаты $A(u)$ в базисе $\{v_i\}$ и это $A \cdot X$

\begin{Remark}{Мораль}
    Любое линейное отображение при координатизации (отождествлении с $K^n$) превращается в умножение на матрицу

    $x \rightarrow A(x)$

    $\tilde{x} \rightarrow A(\tilde{x})$
\end{Remark}

$A : U \rightarrow V$ и знаем $KerU, ImU$ -- подпространства

$KerA = \{ x | A(x) = 0_v \}$

$ImA = \{ A(x) | x \in U \}$

$A : K^n \rightarrow K^m\ X \rightarrow AX$

$A \in M_{m, n}(K)$

$KerA$ -- множество решений однородной СЛУ с матрицей $A$

$ImA = \{ B | \exists x : Ax = B \}$

$u_1 \ldots u_n$ -- базис $\Rightarrow ImA = \q{A(u_1) \ldots A(u_n)}$

$A(u) = \sum a_iA(u_i)$

$e_i$ -- стандартный базис ($i = 1 \ldots n$)

$ImA = \q{A_1e_1 \ldots A_ne_n}$

$\begin{pmatrix}
    a_{11} & a_{12} & \ldots & a_{1n} \\
    a_{21} & a_{22} & \ldots & a_{2n} \\
    \vdots & \vdots & \ddots & \vdots \\
    a_{m1} & a_{m2} & \ldots & a_{mn}
\end{pmatrix} \cdot \begin{pmatrix}
    0 \\
    1 \\
    \vdots \\
    0
\end{pmatrix} = \begin{pmatrix}
    a_{1i} \\
    a_{2i} \\
    \vdots \\
    a_{mi}
\end{pmatrix}$ -- i-ый столбец $A$

\begin{theo}{Теорема о ядре и образе}
    $A : \R^2 \rightarrow \R^2$

    \begin{Example}{}
        \begin{enumerate}
            \item $A$ -- поворот на $\frac{\pi}{2}$ -- линейное отображение
            
            \begin{Remark}{}
                Параллельный перенос не линейное отображение
            \end{Remark}
    
            \item $A(x) = 0$
            
            \item $A$ -- ортогональная проекция на $Ox$
        \end{enumerate}
    \end{Example}

    \begin{enumerate}
        \item $ImA = R^2$
        
        $KerA = \{0\}$

        \item $ImA = \{0\}$
        
        $KerA = \R^2$

        \item $ImA = < \begin{pmatrix}
            1 \\
            0
        \end{pmatrix} >$

        $KerA = < \begin{pmatrix}
            0 \\
            1
        \end{pmatrix} >$
    \end{enumerate}
\end{theo}

\begin{theo}{}
    $A : U \rightarrow V$ -- линейное

    \begin{enumerate}
        \item $\exists$ базис $u_1 \ldots u_n$ в $U$ и $k \leq n : \\ u_1 \ldots u_k$ -- базис $KerA$ и $u_{k + 1} \ldots u_n$ -- базис $ImA$
        \item $dimKerA + dimImA = dimU$
    \end{enumerate}
\end{theo}

\textit{Доказательство:}

\begin{itemize}
    \item[$1 \Rightarrow 2$:] $k = dimKerA$
    
    $n - k = dimImA$

    $n = dimU$

    \item[1:] Выберем $u_1 \ldots u_k$ -- базис $KerA$
    
    $u_1 \ldots u_k$ -- ЛНЗ $\Rightarrow$ дополним до базиса: $u_1 \ldots u_k, u_{k + 1} \ldots u_n$ -- базис $U$

    Осталось доказать: $A(u_{k + 1}) \ldots A(u_n)$ -- базис $ImA$

    \begin{enumerate}
        \item $A(u_i) \in ImA$ по определению
        \item Проверим $\q{A(u_{k + i})} = ImA$
        
        $v \in ImA \Rightarrow v = A(u)\ u \in U$, $a = a_1u_1 + \ldots + a_nu_n$

        $A(u) = \sum a_iA(u_i) = \sum\limits_{i = k + 1}^n a_iA(u_i) \Rightarrow v \in \q{A(u_{k + i})}$

        \item Проверим ЛНЗ: пусть $\sum\limits_{i = k + 1}^n a_iA(u_i) = 0 \stackrel{?}{\Rightarrow} a_i = 0\ \forall i$
        
        По линейности $0 = \sum a_{k + i}A(u_{k + i}) = A(\sum a_{k + i}u_{k + i})$

        То есть $\sum a_{k + i}u_{k + i} \in KerA = \q{u_1 \ldots u_k}$

        $(\sum\limits_{i = 1}^{n - k} a_{k + i}u_{k + i} = \sum\limits_{i = 1}^k (-a_i)u_i) \Rightarrow \sum a_iu_i = 0 \Rightarrow a_1 = \ldots = a_n = 0$

        В частности $a_{k + 1} = \ldots = a_n = 0$
    \end{enumerate}
\end{itemize}

\section{Операции над пространствами}

\begin{lem}{}
    $U_1, U_2 \leq U \Rightarrow U_1 \cap U_2 \leq U$. Д-во: очев

    $U_1, U_2 \leq U \Rightarrow U_1 \cup U_2 \not\leq U$ (почти никогда)

    $U_1 + U_2 := \{ u_1 + u_2 | u_1 \in U_1, u_2 \in U_2 \}$ -- сумма по Минковскому

    Сама лемма: $U_1, U_2 \leq U \Rightarrow U_1 + U_2 \leq U$
\end{lem}

\textit{Доказательство:}

$x, y \in U_1 + U_2$

$x = x_1 + x_2;\ y = y_1 + y_2$, где $x_1, y_1 \in U_1;\ x_2, y_2 \in U_2$

$x + y = (x_1 + y_1) + (x_2 + y_2) \in U_1 + U_2$

\begin{defin}{Прямая сумма}
    $U_1, U_2$ -- векторные пространства над $K$

    $U_1 + U_2 = U_1 \times U_2$ как множество с покомпонентными операциями -- (внешняя) прямая сумма $U_1$ и $U_2$
\end{defin}

\begin{lem}{}
    $i_1 \ldots u_n$ -- базис $U$ и $v_1 \ldots v_m$ -- базис $V$

    Тогда $\{ (u_1, 0) \ldots (u_n, 0), (0, v_1) \ldots (0, v_m) \}$ -- базис $U + V$
\end{lem}

\textit{Доказательство:}

$u \in U;\ v \in V$

$u = \sum a_iu_i;\ v = \sum b_iv_i$

$u + v = \sum a_iu_i + \sum b_iv_i = \sum (a_iu_i, 0) + \sum (0, b_iv_i)$

\begin{theo}{Следствие}
    $dim(U + V) = dimU + dimV$
\end{theo}

\begin{theo}{Формула Грассмана}
    $U_1, U_2 \leq U$, $U$ -- в.п. над $K$

    $dim(U_1 + U_2) = dimU_1 + dimU_2 - dim(U_1 \cap U_2)$
\end{theo}

\textit{Доказательство:}

Рассмотрим линейное отображение $A : U_1 + U_2 \rightarrow U$

$ImA = \{ u_1 + u_2 | u_1 \in U_1, u_2 \in U_2 \} = U_1 + U_2$

$dim(ImA) = dim(U_1 + U_2)$

$dim(U_1 + U_2) = dimU_1 + dimU_2$. Осталось понять: $dimKerA = dim(U_1 \cap U_2)$

Тогда $dim(U_1 + U_2) = dimU_1 + dimU_2 - dim(U_1 \cap U_2)$

$KerA = \{ (u_1, u_2) | u_1 + u_2 = 0 \} = \{ (u_1, u_2) | u_1 = -u_2 \} \Rightarrow$ отображение $U_1 \cap U_2 \rightarrow KerA$ -- изоморфизм векторного пространства

\textbf{TODO lec 02/03}

\begin{defin}{Канонеческий вид матрицы линейного отображения}
    $A \mapsto CAD = \begin{pmatrix}
        E & 0 \\
        0 & 0
    \end{pmatrix}$
\end{defin}

\begin{defin}{Ранг линейного отображения}
    $A : U \rightarrow V$

    $rkA = dimImA$ -- размерность линейной оболочки столбцов матрицы $[A]$ (в любом базисе)

    $A = [A];\ A = (c_1 | c_2 | \ldots | c_n)$

    $rkA = dim\q{c_1 \ldots c_n}$ -- максимальное количество ЛНЗ столбцов матрицы
\end{defin}

\begin{theo}{Свойства ранга}
    \begin{enumerate}
        \item $rk(A + B) \leq rkA + rkB;\ A, B \in M_{m, n}(K)$
        \item $rk(A \cdot B) \leq min(rkA, rkB);\ A \in M_{m, n}(K), B \in M_{n, l}(K)$
        \item Если в пункте 2 $A$ или $B$ обратимы (в том числе $(m = n)/(n = l)$), то \\ $rk(A \cdot B) = rkA = rkB$
        \item $rkA = rkA^T$
        
        \begin{Remark}{}
            Знаем: столбцы $A^T$ -- строки $A$, т.е. $rkA$ -- максимальное количество ЛНЗ строк

            \textbf{Строчный ранг совпадает со столбцовым}
        \end{Remark}
    \end{enumerate}
\end{theo}

\textit{Доказательство:}

\begin{enumerate}
    \item $A = (c_1 | c_2 | \ldots | c_n);\ B = (d_1 | d_2 | \ldots | d_n);\ c_i, d_i \in K^m$
    
    $A + B = (c_1 + d_1 | c_2 + d_2 | \ldots | c_n + d_n)$

    $dim\q{c_1 + d_1 \ldots c_n + d_n} \leq dim\q{c_1 \ldots c_n, d_1 \ldots d_n} \leq dim\q{c_1 \ldots c_n} + dim\q{d_1 \ldots d_n}$

    Значит $rk(A + B) \leq rkA + rkB$

    \item $A \cdot B \leftrightarrow A \circ B$
    
    Хотим $rk(A \circ B) \begin{gathered}
        \stackrel{(1)}{\leq} rkA \\
        \stackrel{(2)}{\leq} rkB
    \end{gathered}$

    \begin{itemize}
        \item[(1)] $rk(A \circ B) = \dim(Im(A \circ B)) = \dim\{A(B(x))\} \leq \dim\{A(y)\} = \dim(ImA) = rkA$
        \item[(2)] $Im(A \circ B) = \{A(B(x)) | x \in \ldots \} = \{ A(y) | y \in ImB \} = Im(A|_{ImB}) = dimImB - dim(Ker(A|_{ImB})) \leq dimImB = rkB$
    \end{itemize}

    \item Пусть $\exists A^{-1}$
    
    Тогда $rk(AB) \leq rk(B) = rk(A^{-1}AB) \leq rk(AB) \Rightarrow rk(B) = rk(AB)$

    \item Найдем $C, D$ -- обратимые
    
    $CAD = \begin{pmatrix}
        \begin{pmatrix}
            1 & \ldots & 0 \\
            \vdots & \ddots & \vdots \\
            0 & \ldots & 1
        \end{pmatrix} & 0 \\
        0 & 0
    \end{pmatrix}$

    $(CAD)^T = D^TA^TC^T = \begin{pmatrix}
        E & 0 \\
        0 & 0
    \end{pmatrix} = A_1^T$

    $rk(A_1) = rk(A_1^T) = l$

    $e_1 \ldots e_l$ -- что-то из стандартного базиса для $K^m$

    По пункту 3
    
    $\begin{gathered}
        l = rk(CAD) = rk(AD) = rk(A) \\
        l = rk(D^TA^TC^T) = rk(A^TC^T) = rk(A^T)
    \end{gathered} \Rightarrow rk(A) = rk(A^T)$

    \begin{Remark}{}
        $C$ -- обратима $\Leftrightarrow C^T$ -- обратима

        $CC^{-1} = C^{-1}C = E$

        $E = E^T = (C^{-1}C)^T = \begin{gathered}
            (C^{-1})^TC^T \\
            C^T(C^{-1})^T
        \end{gathered} \Rightarrow (C^{-1})^T = (C^T)^{-1}$, т.е. $C^T$ -- обратима
    \end{Remark}
\end{enumerate}

\begin{Remark}{Полуобратимость}
    $C \in M_{m, n}(K);\ D \in M_{n, m}(K)$ (т.е. $\exists CD, DC$). Пусть $m < n$

    $\Rightarrow rkC \leq m \Rightarrow rk(DC) \leq m \Rightarrow DC \neq E_n$ ($rkE = n$)

    Но может быть, что $CD = E_m$ -- полуобратные матрицы
\end{Remark}

\begin{theo}{}
    Следующие условия равносильны для $A \in M_n(K)$:

    \begin{enumerate}
        \item Строки $A$ -- ЛНЗ
        \item Столбцы $A$ -- ЛНЗ
        \item $A$ обратима
        \item $KerA = \{0\}$
        \item $ImA = K^n$
        \item СЛУ с матрицей $A$ имеет единственное решение для любой правой части 
    \end{enumerate}
\end{theo}

\textit{Доказательство:}

\begin{enumerate}
    \item $1 \Leftrightarrow rkA = n \Leftrightarrow 2$
    \item В две стороны:
    
    \begin{itemize}
        \item[$3 \Rightarrow 2$:] $n = rkE = rk(AA^{-1}) \leq rkA \geq n \Rightarrow rkA = n$
        \item[$2 \Rightarrow 3$:] $\exists CAD = \begin{pmatrix}
            E & 0 \\
            0 & 0
        \end{pmatrix}$

        $l = rkA = n \Rightarrow CAD = E$

        $A \cdot (DC^{-1}) = E = (DC^{-1}) \cdot A \Rightarrow A$ обратима
    \end{itemize}
    \item $3 \Leftrightarrow 4 \Leftrightarrow 5 \Leftrightarrow 6$
    
    Знаем: $A : K^n \to K^n$, т.е. $A$ -- инъекция ($KerA = 0$) $\Leftrightarrow A$ -- сюръекция ($ImA = K^n$) $\Leftrightarrow A$ -- изоморфизм ($\exists A^{-1}$) 

    \item[6.] $A$ -- обратима СЛУ $AX = B \Rightarrow X = A^{-1}B$
    
    Если $\forall B \exists! X " AX = B \Rightarrow (x \mapsto AX)$ -- биекция $\Leftrightarrow \exists A^{-1}$
\end{enumerate}

\begin{defin}{????}
    $A$ называется обратимой/невырожденной/неособенной/неосовой матрицей полного ранга \dots
\end{defin}

\begin{defin}{Полная линейная группа}
    $(M_n(K))^* = GL(n, k)$ -- полная линейная группа (обратимые матрицы относительно умножения)
\end{defin}

\section{Элементарные матрицы и метод Гаусса}

Хотим: систему простых образующих $GL(n, K) = \q{\{s_i\}} : \forall g \in GL(n, K)\ g = s_{i_1} \cdot s_{i_2} \cdot \ldots \cdot s_{i_k}$ (не единственность разложения)

Приложение: $g^{-1} = (s_{i_1} \cdot \ldots \cdot s_{i_k})^{-1} = s_{i_k}^{-1} \cdot \ldots \cdot s_{i_1}^{-1}$ -- алгоритм для вычисления $g^{-1}$

\begin{defin}{Трансвекция}
    $n \in N$ -- fix $(M,_n(K));\ i, j \in \{1 \ldots n\};\ i \ne j$

    Трансвекциея $t_{ij}(a) = E + aE_{ij};\ e_{ij} \in M_n(K)$ и $(e_{ij})_{kl} = \begin{cases}
        1 & k = i, l = j \\
        0 & \text{otherwise}
    \end{cases}$
\end{defin}

\begin{Example}{}
    Пусть $x \in K^n;\ x = \begin{pmatrix}
        x_1 \\
        x_2 \\
        \vdots \\
        x_n
    \end{pmatrix}$
    
    $t_{ij}(a) \cdot \begin{pmatrix}
        x_1 \\
        x_2 \\
        \vdots \\
        x_n
    \end{pmatrix} = \begin{pmatrix}
        x_1 \\
        \vdots \\
        x_i + ax_j \\
        \vdots \\
        x_n
    \end{pmatrix}$

    К $i$-ой координате прибавляется $j$-ая, умноженная на $a$

    $t_{ij}(a) \in GL(n)$

    $(t_{ij}(a))^{-1} = t_{ij}(-a)$
\end{Example}

\begin{Example}{Действия на матрице}
    Слева $t_{ij}(a) \cdot A = t_{ij}(a)(c_1 | c_2 | \ldots | c_m) = (t_{ij}(a) \cdot c_1 | \ldots | t_{ij}(a) \cdot c_m) = \tilde{A}$

    $\tilde{A}$ получается из $A$ прибавлением к $i$-ой строке $j$-ой строки, умноженной на $a$

    Справа $A \cdot t_{ij}(a) = (A^T)^T((t_{ij}(a))^T)^T = (t_{ij}(a)^TA^T)^T = (t_{ji}(a)A^T)^T$

    К $j$-ому столбцу прибавляется $i$-ый, умноженный на $a$
\end{Example}

\begin{defin}{Дилатация}
    $m_i(a) = E + (a - 1)e_{ij} = \begin{pmatrix}
        1 & \ldots & 0 \\
        \vdots & \ddots & \vdots \\
        \ldots & a & \ldots \\
        \vdots & \ddots & \vdots \\
        0 & \ldots & 1
    \end{pmatrix}$
\end{defin}

\begin{Example}{}
    $m_i(a) \cdot \begin{pmatrix}
        x_1 \\
        x_2 \\
        \vdots \\
        x_n
    \end{pmatrix} = \begin{pmatrix}
        x_1 \\
        \vdots \\
        ax_i \\
        \vdots \\
        x_n
    \end{pmatrix}$

    $m_i(a) \in GL(n)$

    $(m_i(a))^{-1} = m_i(a^{-1})$

    $m_i(a) \cdot A$ -- умножение $i$-ой строки на $a$

    $A \cdot m_i(a)$ -- умножение $i$-ого столбца на $a$
\end{Example}

\begin{defin}{Транспозиция}
    $s_{ij} = E - e_{ii} - e_{jj} + e_{ij} + e_{ji}$
    
    $s_{ij} \cdot \begin{pmatrix}
        x_1 \\
        \vdots \\
        x_i \\
        \vdots \\
        x_j \\
        \vdots \\
        x_n
    \end{pmatrix} = \begin{pmatrix}
        x_1 \\
        \vdots \\
        x_j \\
        \vdots \\
        x_i \\
        \vdots \\
        x_n
    \end{pmatrix}$

    Умножение слева -- перестановка строки, умножение справа -- перестановка столбца
\end{defin}

\begin{propos}{}
    $s_{ij}$ выражаема через трансвенции и дилатации

    $\begin{pmatrix}
        x \\
        y
    \end{pmatrix} \to \begin{pmatrix}
        x + y \\
        y
    \end{pmatrix} \to \begin{pmatrix}
        x + y \\
        -x 
    \end{pmatrix} \to \begin{pmatrix}
        y \\
        -x
    \end{pmatrix} \to \begin{pmatrix}
        y \\
        x 
    \end{pmatrix}$

    $s_{12} = m_2(-1) \cdot t_{21}(1) \cdot t_{12}(-1) \cdot t_{21}(1)$
\end{propos}

\begin{theo}{Метод Гаусса}
    \begin{enumerate}
        \item $A \in M_{m, n}(K) \Rightarrow \exists \text{ элем. } e_1 \ldots e_k : e_1e_2 \ldots e_kA$ -- ступенчатая (типа треугольная но не очень)
        \item $A \in GL(n, K) \Rightarrow \exists \text{ элем. } e_1 \ldots e_s : e_1e_2 \ldots e_sA = E$
        \item[2'.] $\forall A \in GL(n, K) \exists \text{ элем. } f_1 \ldots f_s : A = f_1f_2 \ldots f_s$
        \item[3.] $A \in M_{m, n}(K) \Rightarrow \exists \text{ элем. } e_1 \ldots e_k, g_1 \ldots g_l : e_1e_2 \ldots e_kAg_1 \ldots g_l = \begin{pmatrix}
            E_r & 0 \\
            0 & 0
        \end{pmatrix}$
    \end{enumerate}
\end{theo}

\textit{Доказательство:}

\begin{itemize}
    \item[$2 \Rightarrow 2'$:] $e_1e_2 \ldots e_sA = E$
    
    $A = e_s^{-1} \ldots e_1^{-1} = f_1f_2 \ldots f_s;\ f_i = e_i^{-1}$

    \begin{theo}{Следствие}
        $e_1 \ldots e_sA = E$

        $(e_1 \ldots e_s) = A^{-1}$

        Алгоритм для нахождения $A^{-1}$ (если существует)

        $(A | E) \to (e_sA | e_sE) \to \ldots \to (e_1 \ldots e_sA | e_1 \ldots e_s) = (E | A^{-1})$
    \end{theo}
\end{itemize}

\begin{theo}{Теорема формализующая метод Гаусса}
    \begin{enumerate}
        \item $A \in M_{m, n}(K) \Rightarrow \exists e_1 \ldots e_k$ -- Элементарные

        $e_1 \ldots e_kA = \begin{pmatrix}
            a_{11} & 0 & 0 & \ldots \\
            0 & a_{22} & 0 & \ldots \\
            0 & 0 & a_{33} & \ldots \\
            \vdots & \vdots & \vdots & \ddots
        \end{pmatrix}$

        \item $A \in GL_n(K)\ \exists e_1 \ldots e_s : e_1 \ldots e_sA = E$
        \item $A \in M_{m, n}(K)\ \begin{pmatrix}
            E & 0 \\
            0 & 0
        \end{pmatrix} = e_1 \ldots e_kAg_1 \ldots e_l$
    \end{enumerate}
\end{theo}

\textit{Доказательство:}

\begin{enumerate}
    \item Индукция по $n$
    
    База $n = 0$ очев или $n = 1$ там то же, что и в переходе

    Переход $n \to n + 1$

    $A = \begin{pmatrix}
        a_{11} & \ldots \\
        \vdots & \ddots
        a_{1n} & \ldots
    \end{pmatrix}$

    \begin{itemize}
        \item $a_{11} \neq 0$
        
        Домножим слева на $\prod t_{i1}(- \frac{a_{1i}}{a_{11}}) = T$

        $TA = \begin{pmatrix}
            a_{11} & \ldots \\
            0 & \ldots \\
            \vdots & \ddots \\
            0 & \ldots
        \end{pmatrix}$

        По ИП $\exists u_1 \ldots u_k$ -- элементарные ($u_1 \ldots u_k \in GL_{m - 1} \Rightarrow \tilde{u_i} \in GL_m$)

        $u_1 \ldots u_k \tilde{A} = \begin{pmatrix}
            a_{11} & 0 & \ldots \\
            0 & a_{22} & \ldots \\
            0 & 0 & \ldots
        \end{pmatrix}$

        Тогда $\tilde{u_1} \ldots \tilde{u_k}TA = \begin{pmatrix}
            a_{11} & * & \ldots \\
            0 & a_{22} & \ldots \\
            0 & 0 & \ldots
        \end{pmatrix}$, т.е. получили треугольную

        \item $a_{11} = 0$, но $\exists i : a_{1i} \neq 0$
        
        $\exists$ матрица перестановки строк (произведение элементарных)

        Переставим, перейдем к случаю 1

        \item $\forall i : a_{1i} = 0$
        
        По ИП $\exists e_1 \ldots e_k : e_1 \ldots e_k \tilde{A} = \begin{pmatrix}
            a_{11} & * & \ldots \\
            0 & a_{22} & \ldots \\
            0 & 0 & \ldots \\
            \vdots & \vdots & \ddots
        \end{pmatrix} = \tilde{\tilde{A}}$

        Если у нас матрица с нулевым первым столбцом, то такие же преобразования оставят первый столбик нулевым
    \end{itemize}

    \item $A \in GL_n(K)$
    
    По пункту 1 $\exists e_1 \ldots e_k$ -- элементарные, такие что

    $e_1 \ldots e_kA = \begin{pmatrix}
        a_{11} & a_{12} & \ldots & a_{1n} \\
        0 & a_{22} & \ldots & a_{2n} \\
        \vdots & \vdots & \ddots & \vdots \\
        0 & 0 & \ldots & a_{nn}
    \end{pmatrix} = \tilde{A} \Rightarrow \tilde{A} \in GL_n(K)$

    \begin{lem}{}
        $\tilde{A}$ -- треугольная

        $\tilde{A}$ обратима $\Leftrightarrow$ все $a_{ii} \neq 0$
    \end{lem}

    \textit{Доказательство:}

    $A = (C_1 | \ldots | C_n)$

    Все $a_{ii} \neq 0 \Rightarrow \forall i\ c_i \notin \q{c_1 \ldots c_{i - 1}} \Rightarrow c_1 \ldots c_n$ -- ЛНЗ $\Rightarrow rk\tilde{A} = n \Rightarrow \tilde{A}$ обратима

    А если $\tilde{A}$ обратима $\Rightarrow rk\tilde{A} = n \Rightarrow c_1 \ldots c_n$ -- ЛНЗ

    Пусть $a_{ii} = 0 (\exists i)$, все $c_1 \ldots c_i \in \q{e_1 \ldots e_{i - 1}} \Rightarrow c_n \ldots c_i$ -- ЛЗ???????????????????????? 

    \vspace{5mm}

    Вернемся к теореме

    Теперь доможножим слева на $\prod t_{in}(- \frac{a_{in}}{a_{nn}})$

    Потом на $\prod t_{i(n - 1)}(- \frac{a_{i(n - 1)}}{a_{n(n - 1)}})$ и так далее

    Итого будет какая-то $\tilde{\tilde{A}} = \begin{pmatrix}
        a_{11} & 0 & \ldots \\
        0 & a_{22} & \ldots \\
        0 & 0 & \ldots \\
        \vdots & \vdots & \ddots
    \end{pmatrix}$

    Потом набор дилатаций, которые превратят $\tilde{\tilde{A}} \to E$

    \item Знаем: $\forall A\ \exists C, D$ -- обратимые: $CAD = \begin{pmatrix}
        E & 0 \\
        0 & 0
    \end{pmatrix}$

    По пункту 2 $C = e_1 \ldots e_k;\ D = g_1 \ldots g_l$, где $e_i, g_i$ -- элементарные

    $\Rightarrow e_1 \ldots e_kAg_1 \ldots g_l = \begin{pmatrix}
        E & 0 \\
        0 & 0
    \end{pmatrix}$
\end{enumerate}

\begin{nota}{Разложение Гаусса}
    Знаем: $A \in GL_n(K)$

    $e_1 \ldots e_k A = \begin{pmatrix}
        a_{11} & \ldots & * \\
        \vdots & \ddots & \vdots \\
        0 & \ldots & a_{nn}
    \end{pmatrix} = u;\ a_{ii} \neq 0$

    $A = e_k^{-1} \ldots e_1^{-1}u$

    Пусть всегда в методе Гаусса был случай 1 ($a_{ii} \neq 0$)

    Тогда $\forall i\ e_i = t_{k_il_i}(a_i)$

    $e_i^{-1} = t_{k_il_i}(-a_i)$

    $e_i^{-1} = \begin{pmatrix}
        1 & \ldots & 0 \\
        \vdots & \ddots & \vdots \\
        * & \ldots & 1
    \end{pmatrix}$ -- нижнетреугольная матрица

    Тогда $e_1^{-1} \ldots e_k^{-1}$ -- тоже нижнетреугольная матрица

    Итого: $A = LU$, где $L$ -- нижнетреугольная, $U$ -- верхнетреугольная

    $LU$ -- разложение Гаусса

    В общем случае $\exists P$ -- матрица перестановки

    $PA = LU \Rightarrow A = \tilde{P}LU$

    $\tilde{P}$ -- матрица, где в каждой строке одна единичка на рандомной позиции
\end{nota}

\section{Явные формулы линнейной алгебры}

СЛАУ $AX = B \Rightarrow X = A^{-1}B$, где $A^{-1}$ ищется методом Гаусса

В общем случае: Гаусс

$\begin{cases}
    ax + by = e \\
    cx + dy = f 
\end{cases}$

В общем виде $x = \frac{ed - bf}{ad - bc};\ y = \frac{af - ec}{ad - bc}$, если $ad - bc \neq 0$

Вот эти вот штуки после равно называют определителями. Они выражают идею площади

Что значит, что $ad - bc = 0$? Значит столбцы в матрице ЛЗ, тогда $S(v_1, v_2) = 0$

Хотим функцию $det(K^n)^n \to K$. Причем такую, что:

\begin{enumerate}
    \item $\forall i\ \forall a_1 \ldots a_{i - 1} a_{i + 1} \ldots a_n \in K^n$
    
    Отображение $x \mapsto det(a_1 \ldots a_{i - 1} x a_{i + 1} \ldots a_n)$ -- линейно

    $(K^n \to K)$ -- полилинейность

    \item $\exists i \neq j : x_i = x_j \Rightarrow det$ -- кососимметричность
    \item $det(e_1 \ldots e_n) = 1$, где $e_i$ -- стандартный базис
\end{enumerate}

\begin{Remark}{}
    $(K^n)^n \cong M_n(K)$
    
    Тогда $3 \Leftrightarrow det(E) = 1$
\end{Remark}

\begin{Example}{}
    $n = 2$

    \begin{enumerate}
        \item[3.] Вот столбики $\begin{pmatrix}
                0 \\
                1
            \end{pmatrix}$ и $\begin{pmatrix}
                1 \\
                0
            \end{pmatrix}$. Тогда $S := 1$

        \item[2.] $det(x, x) = 0$
        \item[3.] $det(x_1 + x_2, y) = det(x_1, y) + det(x_2, y)$
    \end{enumerate}
\end{Example}

\begin{Remark}{}
    $f$ -- полилинейная и кососимметричная $\Rightarrow \\
    \Rightarrow \forall i, j\ f(x_1 \ldots x_i \ldots x_j \ldots x_n) = -f(x_1 \ldots x_j \ldots x_i \ldots x_n)$
\end{Remark}

\textit{Доказательство:}

В общем виде доказывать не будем, нам лень

$n = 2$

В $(x, y) = f(x_1 \ldots x_{i - 1}x_i \ldots g_1 x_{j + 1} \ldots x_n)$

$x_k$ -- fix при $k \neq i, j$

$g(x, x)= 0\ \forall x$

$g(x + y, x + y) = g(x, x) + g(x, y) + g(y, x) + g(y, y) = 0$

\begin{Remark}{}
    Это похоже на свойство из определения, но равносильность есть только тогда, когда $charK \neq 2$
\end{Remark}

\begin{theo}{}
    Если $det_1, det_2$ -- функции удовлетворяющие аксиомам 1-3, то 
    
    $det_1(x_1 \ldots x_n) = det_2(x_1 \ldots x_n)$ ($\forall x_1 \ldots x_n \in K^n$), т.е. $det_1 = det_2$
\end{theo}

\begin{theo}{}
    $det$ существует
\end{theo}

\textit{Начало доказательства теоремы 5.2:}

Явная формула для $det$

$A = (x_1 | \ldots | x_n) = (a_{ij});\ i, j = 1 \ldots n$

$detA = \sum\limits_{i_1 \neq i_2 \neq \ldots \neq i_n} \pm a_{1i_1} a_{2i_2} \ldots a_{ni_n}$

$n!$ слагаемых, каждому нужен знак

Слагаемое: $n \to i_n$ -- биекция (перестановка), назовем $\pi$

$\pi(k) = i_k$

$S_n$ -- группа перестановок $|S_n| = n!$

$det = \sum\limits_{\pi \in S_n} \varepsilon(\pi) a_{1\pi(1)} \ldots a_{n\pi(n)}$

$s_{ij}$ -- транспозиция, которая $s_{ij}(i) = j;\ s_{ij}(j) = i;\ s_{ij}(k) = k$ при $k \neq i, j$

$\pi = s_{i_1j_1} \circ \ldots \circ s_{i_kj_k}$. Тогда $\varepsilon(\pi) = (-1)^k$

Такое разложение существует (очев), но не единственно!

$k$ -- не однозначно определено, но $k \mod 2$ -- однозначно определено

$\Rightarrow \varepsilon(\pi)$ -- корректно определено

\end{document}

