\documentclass[14pt, letter paper]{article}
\usepackage[utf8]{inputenc}
\usepackage[russian]{babel}
\usepackage[normalem]{ulem}
\usepackage{amsmath}
\usepackage{amssymb}
\usepackage{dsfont}
\usepackage{textcomp}
\usepackage{mathabx}
\def\multiset#1#2{\ensuremath{\left(\kern-.3em\left(\genfrac{}{}{0pt}{}{#1}{#2}\right)\kern-.3em\right)}}
\usepackage[left=2cm,right=2cm, top=2cm,bottom=2cm,bindingoffset=0cm]{geometry}
\usepackage[unicode, pdftex]{hyperref}
\usepackage{listings}

\begin{document}

\begin{flushright}
    Конспект Шорохова Сергея

    Если нашли опечатку/ошибку - пишите @le9endwp
\end{flushright}

\begin{center}
    \subsection*{\S 0. Методы математического доказательства}
\end{center}

\begin{enumerate}
    \item Индукция

    \begin{enumerate}
        \item База индукции
        \item Индукционное предположение
        \item Индукционный переход
    \end{enumerate}

    $P_1, P_2 \ldots P_n$

    \begin{itemize}
        \item 1 аксиома индукции

        $\begin{cases}
            P_1 \text{-- истина} \\
            \forall i$ $P_i \rightarrow P_{i_1}
        \end{cases}$
        $\Rightarrow \forall i$ $P_i$ - истина

        \item 2 аксиома индукции

        $\begin{cases}
            P_1 \text{-- истина} \\
            \forall i$ $P_1 \ldots P_i \rightarrow P_{i+1}
        \end{cases}$
        $\Rightarrow \forall i$ $P_i$ - истина
    \end{itemize}
    \item "От противного"

    $A \rightarrow B$ $\overline{B} \rightarrow \overline{A}$

    \item Полный перебор

    \item Прямой вывод

    $A \rightarrow B \rightarrow C \rightarrow D$, от $A$ к $D$

    \item Контрпример

    \item Комбинаторное доказательство (сведение к известной задаче)

    \item Двусторонние оценки

    $\begin{cases}
        A \geq B \\
        B \geq A
    \end{cases}$
    $\Rightarrow A = B$

    \item Оценка + пример

    \item Дедукция + рекурсия

    $A \rightarrow B \rightarrow C \rightarrow D$, от $D$ к $A$

    \item Принцип Дирихле

    Биективное отображение для множеств разного размера оставит "лишние" элементы в одном из них

    \item Инвариант

    \textbf{Ex}. Доказательство баланса красно-черного дерева

    \item Доказательство эквивалентных утверждений

    $A \rightarrow B \rightarrow C \rightarrow D \rightarrow A$
\end{enumerate}

\begin{center}
    \subsection*{\S 1. Множества}
\end{center}

\textbf{Def.} $|A|$ - мощность множества (количество элементов в множестве

$A \bigcap B = \o \Rightarrow |A \bigcup B| = |A| + |B|$ 

\begin{itemize}
    \item $A_1 \ldots A_n$

    $\forall i \neq j$ $A_i \bigcap A_j = \o$

    $|\bigcup\limits_{i = 1}^{n} A_i| = \sum\limits_{i=1}^{n} |A_i|$

    \item $A_1 \ldots A_n$

    $|\bigtimes\limits_{i=1}^{n} A_i| = \prod\limits_{i=1}^{n} |A_i|$
\end{itemize}

\textbf{Def.} $|A \bigcup B| = |A| + |B| - |A \bigcap B|$ - правило включения-исплючения

\begin{center}
    Доказательство
\end{center}

$A = (A \backslash B) \bigcup (A \bigcap B)$

$|A| = |A \backslash B| + |A \bigcap B|$

$|B| = |B \backslash A| + |A \bigcap B|$

$|A| + |B| = |A \backslash B| + |A \bigcap B| + |B \backslash A| + |A \bigcap B| = |(A \backslash B) \bigcup (A \bigcap B) \bigcup (B \backslash A)| = |A \bigcup B| + |A \bigcap B|$

\rule{\paperwidth}{0.4pt}

\textbf{Дома} обобщение для произвольного $n$

$|A \bigcup B \bigcup C| = |A| + |B| + |C| - |A \bigcap B| - |A \bigcap C| - |B \bigcap C| + |A \bigcap B \bigcap C|$

\rule{\paperwidth}{0.4pt}

$L = \{ A, C, G, T\}$

$|L^k| = |L|^k = 4^k$

$\begin{cases}
    f(n) = n \cdot f(n-1) \\ 
    f(0) = 1
\end{cases}$ -- количество перестановок

$n(n-1)\ldots(n-k+1) = A_n^k = \frac{n!}{(n-k)!} = C_n^k \cdot k!$

$C_n^k = {n \choose k} = \frac{A_n^k}{k!} = \frac{n!}{k!(n-k)!}$

$(a + b)^n = (a + b)(a + b) \ldots (a + b) = \sum\limits_{i=0}^n c_i a^i b^{n-i} = \sum\limits_{i=0}^n C_n^i a^i b^{n-i}$

Если представить $a_1, a_2 \ldots a_n$ как двоичное число или из $(1 + 1)^n = \sum\limits_{i=0}^n C_n^i \cdot 1^i \cdot 1^{n-i} = \sum\limits_{i=0}^n C_n^i$

Тогда $\sum\limits_{i=0}^n C_n^i = 2^n$

Дома найти $\sum\limits_{k=0}^n (-1)^k C_n^k$

Посчитаем рекуррентно:

В $a_1 \ldots a_n$ $a_1$ либо берем, либо не берем
\begin{itemize}
    \item Если берем, то $C_{n-1}^{k-1}$
    \item Если не берем, то $C_{n-1}^k$
\end{itemize}

Значит $C_n^k = C_{n-1}^{k-1} + C_{n-1}^k$

Другое доказательство: $C_{n-1}^k + C_{n-1}^{k-1} = \frac{(n-1)!}{k!(n-1-k)!} + \frac{(n-1)!}{(k-1)!(n-k)!} = \frac{(n-1)!}{(k-1)!(n-1-k)!}(\frac{1}{k} + \frac{1}{n-k}) = \frac{(n-1)!}{(k-1)!(n-1-k)!} \cdot \frac{n}{k(n-k)} = \frac{n!}{k!(n-k)!} = C_n^k$

Воспользоваться суммой можно из \href{https://ptri1.tripod.com/ptreal1r.gif}{треугольника Паскаля}. Его можно представить и в виде квадрата. Тогда можем посчитать $C_n^i$ за $i(n-i+1) - (n+1)$, по формуле только $n!$ считали бы $lgn \cdot n$

\textbf{Свойства:}
\begin{enumerate}
    \item $C_n^0 = C_n^n = 1$
    \item $C_n^i = C_n^{n-i}$

    $\frac{n!}{k!(n-k)!} = \frac{n!}{(n-n+k)!(n-k)!}$
\end{enumerate}

\textbf{Задача}

Пусть есть $n$ книг и $k$ полок. Способов разделить на полки (= поставить $k-1$ перегородок) $\frac{(n+1)(n+2) \ldots (n+k-1)}{(k-1)!} = \frac{A_{n+k-1}^{k-1}}{(k-1)!} = C_{n+k-1}^{k-1}$

\textbf{Def.} Отношения $A, B$ $\rho \subset A \times B$ 

$a \rho b$ $\forall a \in A, b \in B$, если $(a, b) \in \rho$

\textbf{Свойства:}

\begin{enumerate}
    \item $\forall a \in A$ $a \rho a$ - рефлексивность
    \item $\forall a, b \in A$ $a \rho b \Rightarrow b \rho a$ - симметричность
    \item $\forall a, b, c \in A$ $\begin{cases}
        a \rho b \\
        b \rho c
    \end{cases} \Rightarrow a \rho c$ - транзитивность
\end{enumerate}

Если выполняются все 3, то это оношение эквивалентности. Все элементы разобьются на классы эквивалентности

$A, B$; $f: A \rightarrow B \Leftrightarrow \forall a \in A \exists b \in B:(a, b) \in f$

\textbf{Def.}

Пусть $A$ - позиции в слове, $B$ - символы алфавита

Количество отображений - количество строк длины $|A|$

$f(a) = f(b) \Rightarrow a = b$ - инъективность

$\forall b \in B \exists a \in A: f(a) = b$ - сюръективность

Если $f: A \rightarrow B$ - биективно, то $|A| = |B|$, при этом количество биекций - количество перестановок

Количество инъекций - $A_n^k$

$A, B$ -- конечные множества 

Отображение -- правило, сопоставляющее $a \in A$ $b \in B$, т.е. 

$f: A \rightarrow B$

$\forall x \in A$ $\exists y : f(x) = y$

$\begin{gathered}
    (x, f(x)) \\
    x \in A; y = f(x) \in B
\end{gathered}$
-- график отображений

$|B|^{|A|}$ -- количество отображений

$Im(M) = \{ f(x) | x \in M\}$ -- образ $M$

Виды отображений:

\begin{itemize}
    \item Инъективные

    $f(x_1) = f(x_2) \Rightarrow x_1 = x_2$

    $|Im(A)| = |A|$

    На $|B|$ позиций $|A|$ элементов

    $A_{|B|}^{|A|}$ -- количество отображений

    \item Сюръективные

    $\forall y \in B \exists x \in A: f(x) = y$

    $Im(A) = B$

    $\forall y \in B$; $P_y = \{ x | f(x) = y\} \neq \o$

    $Im(P_y) = \{y\}$

    $\hat{S} (n, k)$ -- количество сюръективных отображений $A \rightarrow B, |A| = n, |B| = k$
\end{itemize}

$k^n = \sum\limits^k_{i=0} (\hat{S}(n, i) \cdot C_k^i)$

$\begin{cases}
    f_0, f_1 \ldots; g_0, g_1 \ldots \\
    f_k = \sum\limits_i C_k^i g_i
\end{cases}$
$\Rightarrow g_i = \sum\limits_i^k (-1)^{k-i} C_k^i f_i$, если докажем, получим $\hat{S}(n,k) = \sum (-1)^{k-i} C_k^i k^i$

\begin{center}
    \textbf{Доказательство}
\end{center}

TODO, из-за отсутствия практик пока не доказываем

$\frac{\hat{S}(n,k)}{k!} = S(n,k)$ -- число Стирлинга первого рода

$k$ предметов (множество $X$), $n$ ящиков (множество $Y$)

\begin{center}
    \begin{tabular}{|c|c|c|c|c|}
        \hline
        X & Y & Произвольно & $\leq 1$ & $\geq 1$ \\
        \hline
        Различимы & Различимы & $k^n$ & $A_k^n$ & $\hat{S}(n, k)$ \\
        \hline
        Неразличимы & Различимы & $C_{n+k}^k$ & $C_k^n$ & $C_{k-1}^{n-1}$ \\
        \hline
        Различимы & Неразличимы & $B(n, k)$ & $\begin{gathered} 0, k > n \\ 1, k \leq n \end{gathered}$ & $S(n, k)$ \\
        \hline
    \end{tabular}
\end{center}

$B(n, k) = \sum\limits_i^n S(i, k)$

\begin{center}
    \section*{Рекуррентные соотношения}
\end{center}

$f_{n+m} = a_0f_n + a_1f_{n+1} + \ldots a_{m-1}f_{n+m-1}$

$f_0 \ldots f_{n-1}$

Прогой рекурсия удобно преображается в динамику (без проги нет)

\vspace{5mm}

\textbf{Числа Фиббоначи:} $f_{n+2} = f_{n+1} + f_n$

$f_0 = 0\ f_1 = 1$

Явная формула (сложно): $\frac{1}{\sqrt{5}}((\frac{1 + \sqrt{5}}{2})^n - (\frac{1 - \sqrt{5}}{2})^n)$

\begin{center}
    \textbf{Доказательство}
\end{center}

База $n = 0, 1$ -- верно

Переход $n \rightarrow n + 1$

$f_{n+2} = \frac{1}{\sqrt{5}}((\frac{1 + \sqrt{5}}{2})^n(\frac{1 + \sqrt{5}}{2} + 1) - (\frac{1 - \sqrt{5}}{2})^n(\frac{1 - \sqrt{5}}{2} + 1))$

$\frac{3 + \sqrt{5}}{2} = (\frac{1 + \sqrt{5}}{2})^2 = \frac{1 + 5 + 2 \sqrt{5}}{4}$

\vspace{5mm}

$f_n = \lambda^n;\ \lambda \neq 0$

$\lambda^{n+m} = a_0\lambda^n \ldots a_{m-1}\lambda^{n+m-1}$

$\lambda^m = a_0 + \ldots + a_{m-1}\lambda^{m-1}$

$\lambda_{1\ldots n} = $

$f_n = c_1\lambda_1^n + c_2\lambda_2^n \ldots c_m\lambda_m^n$ -- характеристическое уравнение

\vspace{5mm}

\textbf{На примере чисел Фиббоначи}

$\lambda^2 = \lambda + 1$

$\lambda^2 - \lambda - 1 = 0$

$\lambda_{1, 2} = \frac{1 \pm \sqrt{1 + 4}}{2}$

$f_n = c_1\lambda_1^n + c_2\lambda_2^n$

$\begin{cases}
    c_1 + c_2 = 0 \\
    c_1(\frac{1 + \sqrt{5}}{2}) + c_2(\frac{1 - \sqrt{5}}{2}) = 1
\end{cases}$

$c_2 = -c_1$

$c_1(\frac{1 + \sqrt{5}}{2} - \frac{1 - \sqrt{5}}{2}) = 1$

$c_1 = \frac{1}{\sqrt{5}};\ c_2 = -\frac{1}{\sqrt{5}}$

\vspace{5mm}

Корней не всегда $n$

Для $f_{n+2} = 4f_{n+1} - 4f_n$ неправда (корни кратные)

Что делать?

Дифференцируем!

\vspace{5mm}

$(n+m)\lambda^{n+m-1} = a_0n\lambda^{n-1} \ldots a_{m-1}(n+m-1)\lambda^{n+m-2}$

$c_1\lambda_1^n + c_2n\lambda_2^{n-1}$ -- может быть решением

$\lambda_{1,2} = 2$

$c_12^n + c_22^{n-1}n$

$\begin{cases}
    c_1 + 0 = 0 \\

    c_1\cdot 2 + c_2 \cdot 2^0 \cdot 1 = 1
\end{cases} \Rightarrow \begin{cases}
    c_1 = 0 \\
    c_2 = 1
\end{cases}$

Ответ: $2^{n-1} \cdot n$

\vspace{5mm}

А что если корней нет вовсе?

$f_{n+2} = 4f_{n+1} - 5f_n$

$\lambda^2 - 4\lambda + 5 = 0$

$\lambda_{1,2} = 2 \pm \sqrt{2^2 - 5} = 2 \pm i$

Корни вида $c_1\lambda_1^n + c_2\lambda_2^n$ будут удовлетворять равенству, но в комплексных числах

Из $f_n = c_1\lambda_1^n + c_2\lambda_2^n$ мнимая часть будет $= 0$

\vspace{5mm}

$a \pm bi = 2(\cos{\alpha} + i\sin{\alpha}) = r \cdot e^{i\alpha}$

$c_12^n \cos{\alpha}^2 + c_22^n\sin{\alpha}^2$

\vspace{5mm}

$\frac{e^{i\alpha} + e^{-i\alpha}}{2} = \frac{\cos{\alpha} + i\sin{\alpha}}{2} + \frac{\cos{\alpha} - i\sin{\alpha}}{2} = \cos{\alpha}$

\vspace{5mm}

$(\cos{\alpha} + i\sin{\alpha})^n = (e^{i\alpha})^n = e^{in\alpha} = \cos{n\alpha} + i\sin{n\alpha}$

\vspace{5mm}

$c_1\cos{n\alpha} + c_2\sin{n\alpha}$

$\lambda_{1,2} = 2 \pm \sqrt{2^2 - 5} = 2 \pm i = \sqrt{5}(\frac{2}{\sqrt{5}} \pm \frac{i}{\sqrt{5}})$

$\alpha = a2\cos{\frac{2}{\sqrt{5}}}$

$c_1\cos{n\alpha} + c_2\sin{n\alpha}$

\vspace{5mm}

$f_n = a_1f_{n-1} + a_2f_{n-2} + 2^n$

$\lambda_{1,2}$

$f_n = c_1\lambda_1^n + c_2\lambda_2^n + K(n)$

$K(n) - K(n-1) \cdot a_1 - K(n-2) \cdot a_2 = 2^n$

$K(n) = W \cdot 2^n$

$W \cdot 2^n - W \cdot 2^{n-1} \cdot a_1 - W \cdot 2^{n-2} \cdot a_2 = 2^n$

$4W - 2a_1W - a_2W = 4$

\vspace{5mm}

\begin{center}
    \section*{Теория вероятностей}

    \subsection*{Классическая вероятность}
\end{center}

$P(\omega_i) = P_i$

$P_i = \frac{|\text{успех}|}{\Omega}$

\textbf{Свойства:}

\begin{enumerate}
    \item $\sum P_i = 1;\ P(\Omega) = 1$
    \item $A, B : A \bigcap B = \o;\ P(A \bigcup B) = P(A) + P(B)$
    \item $P(\Omega \backslash A) = 1 - P(A)$
    \item $A \subseteq B;\ P(A) \leq P(B)$
    \item $0 \leq P(A) \leq 1$
\end{enumerate}

\textbf{Def.} Случайность -- разультат конкретных воздействий, влияние которых мы не можем объяснить

\begin{center}
    \subsection*{Частотный способ определения вероятности}
\end{center}

На определенном периоде считаем вероятность, на следующем периоде (их много) ситуация $\sim$ та же

\textbf{Def.} Условная вероятность: $P(A|B) = \frac{P(A \bigcap B)}{P(B)};\ P \neq 0$

\textbf{Def.} $A$ не зависит от $B$ если $P(A|B) = P(A)$

\textbf{Def.} Незавитсимость совокупности: $P(A_{i_1} \ldots A_{i_k}) = \bigsqcap P(A_{i_j})$

\vspace{5mm}

\begin{center}
    \item \subsection*{Общее определение вероятности}
\end{center} 

$\Omega$ -- множетсво элементарных исходов

$F$ -- множество событий

$P : F \rightarrow R$ -- функция вероятности

$F \subset 2^\Omega$

\begin{enumerate}
    \item $\Omega \in F$
    
    \item $\omega \in F \Rightarrow \Omega \backslash \omega = \overline{\omega} \in F$
    
    \item $\omega_1, \omega_2 \in F \Rightarrow \omega_1 \bigcup \omega_2 \in F$

    \item[3'.] $\omega_1 \ldots \omega_n \ldots \in F \Rightarrow \bigcup\limits_{i=1}^\infty \omega_i \in F$
\end{enumerate}

Если выполняются 1-3 -- это алгебра

Если выполняются 1, 2, 3' -- это $\sigma$-алгебра

\begin{enumerate}
    \item[4.] $P(\Omega) = 1$
    \item[5.] $P(\omega) \geq 0$
    \item[6.] $P(\bigcup \omega_i) = \sum P_i$, если $\omega_i \bigcap \omega_j = 0$
\end{enumerate}

Как следствие:

\begin{itemize}
    \item $\omega \bigcup \overline{\omega} = \Omega$
    \item $1 = P(\omega \bigcup \overline{\omega}) = P(\omega) + P(\overline{\omega})$
    \item $P(0) = 0$
\end{itemize}

\textbf{Def.} $(\Omega, F, P)$ -- вероятностное пространство

\vspace{5mm}

\textbf{Def.} Формула полной вероятности

Пусть $\Omega = \Omega_1 \bigsqcap \Omega_2 \ldots \bigsqcap \Omega_n$

$P(A) = P(A | \Omega_1) \cdot P(\Omega_1) + \ldots + P(A | \Omega_n) \cdot P(\Omega_n)$

$A = A \bigcap \Omega = (A \bigcap \Omega_1) \bigcup \ldots \bigcup (A \bigcap \Omega_n)$

$P(A | \Omega_1) = \frac{P(A \bigcap \Omega_1)}{P(\Omega_1)}$

\vspace{5mm}

\textbf{Th.} Теорема (формула) Байеса 

$A, B$

$P(A | B) = \frac{P(B|A) \cdot P(A)}{P(B)}$

\vspace{5mm}

Для $x_i$:

$p$ -- успех; $1 - p$ -- неудача

$M_n(k) = P$(ровно $k$ успехов)

$x_1 \ldots x_n$

$M_n(K) = {n \choose k} p^k (1-p)^{n-k}$

$M_n(0), M_n(1), \ldots M_n(n)$

$M_n(a, b) = \sum\limits_{k=a}^b {n \choose k} p^k (1-p)^{n-k}$

$\frac{M_n(k)}{M_n(k+1)} = \frac{{n \choose k} p^k (1-p)^{n-k}}{{n \choose k + 1} p^{k+1} (1-p)^{n-k-1}} = \frac{1-p}{p} \cdot \frac{\frac{n!}{k!(n-k)!}}{\frac{n!}{(k+1)!(n-k-1)!}} = \frac{1-p}{p} \cdot \frac{k+1}{n-k}$

$\frac{(1-p)(k+1)}{p(n-k)} > 1$

$(1-p)(k+1) > pn - pk$

$1 - p + k - kp > pn - pk$

$1 - p + k > pn$

$k > pn - (1-p)$

\vspace{5mm}

\textbf{Th.} Теорема Пуассона

$\lambda = np$

$\lambda = const$ при $n \rightarrow \infty$

$M_n(k) \approx \frac{\lambda^k}{k!} e^{-\lambda}$

Почему?

$M_n(k) = \frac{n!}{k!(n-k)!} p^k (1-p)^{n-k} = \frac{1}{k!} \cdot n \cdot (n-1) \cdot \ldots \cdot (n-k+1) \cdot p^k \cdot (1-p)^{n-k} = \frac{1}{k!} (1 - \frac{1}{n}) \ldots (1 - \frac{k-1}{n}) \cdot \\ \cdot n^k \cdot p^k \cdot (1-p)^{n-k} = \frac{\lambda^k}{k!} e^{-\lambda}$

Здесь используется $(1-p)^{n-k} = (1 - \frac{\lambda}{n})^{n-k} = \frac{(1 - \frac{\lambda}{n})^n}{(1 - \frac{\lambda}{n})^k} \rightarrow e^{-\lambda}$

\vspace{5mm}

\textbf{Th.} Локальная теорема Муавра-Лапласа

$P_n(k);\ x_n = \frac{k -np}{\sqrt{np(1-p)}}$. Пусть при $\begin{gathered}
    n \rightarrow \\
    k \rightarrow
\end{gathered} \infty;\ x_k$ не ограничена

$\sqrt{np(1-p)} \cdot P_n(k) \approx \frac{1}{\sqrt{\lambda \pi}} e^{-\frac{x_n^2}{2}}$

$k! \approx \sqrt{2\pi k} (\frac{k}{e})^k$ -- формула Стирлинга

$ln(1+z) = z - \frac{z^2}{2} + o(z^2)$

$|\frac{k-np}{\sqrt{np(1-p)}}| < c \Rightarrow k < np \pm \sqrt{np(1-p)} \cdot c$

$q = 1 - p$

$k = np + \sqrt{npq} \cdot x_n$

$\frac{k}{np} = 1 + \sqrt{\frac{q}{np}} x_n \rightarrow 1$

$n-k = n - np - \sqrt{npq} x_n = nq - \sqrt{npq} x_n$

$\frac{n-k}{nq} = 1 - \sqrt{\frac{p}{nq}} x_n \rightarrow 1$

$k - n \rightarrow - \infty$

$a^b = e^{lna \cdot b}$

$\sqrt{npq} \cdot P_n(k) = \sqrt{npq} \cdot \frac{n!}{(n-k)!k!}p^kq^{n-k} \approx \frac{1}{\sqrt{2\pi}} \cdot \frac{\sqrt{n}n^n}{\sqrt{n-k}(n-k)^{n-k}\sqrt{k}k^k} p^k q^{n-k} \sqrt{npq} = \frac{1}{\sqrt{2\pi}} (\frac{np}{k})^k(\frac{nq}{n-k})^{n-k}\sqrt{\frac{np}{k}}\sqrt{\frac{nq}{n-k}} \approx$

$\approx \frac{1}{\sqrt{2\pi}} (\frac{k}{np})^{-k}(\frac{n-k}{nq})^{k-n} = \frac{1}{\sqrt{2\pi}} (1 + \frac{\sqrt{q}x_n}{\sqrt{np}})^{-k}(1 - \frac{\sqrt{p}x_n}{\sqrt{nq}})^{k-n} = \frac{1}{\sqrt{2\pi}} exp(-k \cdot ln(1 + \frac{\sqrt{q}x_n}{\sqrt{np}}) \cdot exp(-(n-k) \cdot ln(1 - \frac{\sqrt{p}x_n}{\sqrt{nq}}) =$

$= \frac{1}{\sqrt{2\pi}} exp(-k(\frac{\sqrt{q}x_n}{\sqrt{np}} - \frac{qx^2_n}{2np}(1 + o(1))) - (n-k)( - \frac{\sqrt{p}x_n}{\sqrt{nq}} - \frac{px^2_n}{2nq}(1 + O(1)))) = \frac{1}{\sqrt{2\pi}}exp(-x_n^2(1 - (\frac{1}{2} + o(1)))) =$

$= \frac{1}{\sqrt{2\pi}}e^{\frac{-x_n^2}{2} + o(x_n^2)}$

\vspace{3mm}

$x_n(\frac{-k\sqrt{q}}{\sqrt{np}} + \frac{(n-k)\sqrt{p}}{\sqrt{nq}}) = x_n \frac{-kq + (n-k)p}{\sqrt{npq}} = x_n \frac{np -k}{\sqrt{npq}} = -x_n^2$

\vspace{3mm}

$\frac{kq}{2np}(1 + o(1)) + \frac{(n - k)p}{2nq}(1 + o(1)) = \frac{1}{2}(q(1 + \ldots x_n)(1 + o(1)) + p(1 - \ldots x_n)(1 + o(1))) = \frac{1}{2}(p + q)(1 + o(1))$

\vspace{5mm}

$P_n(k_1, k_2)$

$a_n = \frac{k_1-np}{\sqrt{npq}};\ P_n = \frac{k_2 - np}{\sqrt{npq}}$

$\lim{((P_n(k_1, k_2) - \frac{1}{\sqrt{2\pi}})\int e^{\frac{-x^2}{2}}} dx) = 0$

\vspace{5mm}

При броске кубика множество элементарных исходов -- количество точек на верхней грани, количество очков за бросок мы ставим самостоятельно (иногда исходя из количества точек, но не обязательно)

$\xi : F \rightarrow R$ на $(\Omega, F, P)$

$\xi(\text{выпала 1}) = \{1, 2, 1, -2, -1 \ldots\}$

$\{x | \xi(x) = t \in R\} \rightarrow P(\xi = t) = P_\xi(t)$

\begin{center}
    \begin{tabular}{|c|c|c|c|}
        \hline
        $t$ & $t_1$ & $t_2$ \ldots & $t_k$ \\
        \hline
        $P$ & $P_1$ & $P_2$ \ldots & $P_k$ \\
        \hline
    \end{tabular}
\end{center}

$F_\xi(t) = P(\xi < t)$

$\begin{cases}
    \lim\limits_{t \rightarrow - \infty}{F_\xi(t)} = 0 \\
    \lim\limits_{t \rightarrow + \infty}{F_\xi(t)} = 1
\end{cases}$

\begin{center}
    \begin{tabular}{|c|c|c|c|c|c|c|c|c|c|c|c|}
        \hline
        $\xi_i$ & 2 & 3 & 4 & 5 & 6 & 7 & 8 & 9 & 10 & 11 & 12 \\
        \hline
        $P_i$ & $\frac{1}{36}$ & $\frac{2}{36}$ & $\frac{3}{36}$ & $\frac{4}{36}$ & $\frac{5}{36}$ & $\frac{6}{36}$ & $\frac{5}{36}$ & $\frac{4}{36}$ & $\frac{3}{36}$ & $\frac{2}{36}$ & $\frac{1}{36}$ \\
        \hline
    \end{tabular}
    
    {\footnotesize Сумма чисел на кубике}
\end{center}

$F_\xi(t) = P(\{\xi < t\})$

Плотность -- $f_\xi(t) : \int\limits_{-\infty}^t f_\xi(x)dx = F_\xi(t)$

$f_\xi(t) = F_\xi'(t)$

При $t > q$ $F_\xi(t) - F_\xi(q) = \int\limits_q^t f_\xi(x) dx$

$U[a, b];\ f_{U[a,b]}(t) = \left[ \begin{gathered}
    0,\ t \not\in [a, b] \\
    c,\ t \in [a, b]
\end{gathered} \right.$

$l = \frac{1}{b-a}$

\vspace{3mm}

$\frac{c}{1 + (x - \Theta)^2}$ -- распределение Коши

\textbf{Совместное распределение}

$\xi, \eta$ -- с.в.

$(\xi, \eta);\ P((\{\xi = k\}),(\{\eta = m\})) = P){km}$

\begin{center}
    \begin{tabular}{|c|c|c|c|}
        \hline
        & $k_1$ & \ldots & $k_n$ \\
        \hline
        $m_1$ & $P_{1,1}$ & \ldots & $P_{1, n}$ \\
        \hline
        \vdots & & &\\
        \hline
    \end{tabular}
\end{center}

$\forall k, m;\ P_{km} = P(\xi = k) \cdot P(\eta = m)$

$E_\xi = \sum\limits_{i=1}^n k_i p_i$

$E_{c\xi} = \sum\limits_{i=1}^n (c k_i)p_i = c\sum\limits_{i=1}^n k_i p_i = c E_\xi$

$E_{\xi + c} = \sum\limits_{i=1}^n (k_i + c) p_i = \sum\limits_{i=1}^n k_i p_i + \sum\limits_{i=1}^n c p_i = E_\xi + c$

$E_{\xi + \eta} = E_\xi + E_\eta$ -- упражнение на дом

$E_{\xi\eta} \neq E_\xi \cdot E_\eta$, правда если $\xi$ и $\eta$ независимы

$E_\xi = \int\limits_{-\infty}^{+\infty} xf_\xi(x)dx = \int\limits_{-\infty}^{+\infty} xdF(x)$

$E_{\xi\eta} = \sum\limits_{i=1}^n\sum\limits_{j=1}^l k_im_jp_{ij} = \sum\limits{i=1}^n \sum\limits_{j=1}^l k_ip_i m_jq_j = \sum(k_ip_i(\sum(m_jq_j))) = E_\eta \sum k_ip_i = E_\eta \cdot E_\xi$

$E(\xi - E_\xi)^2 =: D_\xi$

$D_{c\xi} = c^2 D_\xi$

$d_{\xi + \eta} \neq D_\xi + D_\eta$, верно только в случае независимости $\xi$ и $\eta$

\textbf{Def.} $E_{\xi^k}$ -- $k$-ый момент

\textbf{Def.} $E_{|xi^k|}$ -- $k$-ый абсолютный момент

\textbf{Def.} $E_{(\xi - E_\xi)^k}$ -- $k$-ый центральный момент

$cov(\xi, \eta) = E_{(\xi - E_\xi)(\eta - E_\eta)} = E_{\xi\eta - \xi E_\eta - \eta E_\xi + E_{\xi\eta}} = E_{\xi\eta} - E_\xi \cdot E_\eta - E_\eta \cdot E_\xi + E_\xi \cdot E_\eta = E_{\xi\eta} - E_\xi \cdot E_\eta$

\textbf{Def.} $r(\xi, \eta) = \frac{cov(\xi, \eta)}{\sqrt{D_\xi D_\eta}}$ -- коэффициент корреляции

$E(\xi - E_\xi) = 0$

$r(\alpha\xi + x; \beta\eta + y) = r(\xi;\eta) \cdot sign(\alpha\beta)$

$-1 \leq r(\xi, \eta) \leq 1$

\begin{center}
    \subsection*{Производящие и характеристические функции}
\end{center}

\begin{center}
    \begin{tabular}{c | c | c}
        0 & 1 & 2 \ldots \\
        \hline 
        $p_0$ & $p_1$ & $p_2$
    \end{tabular}
\end{center}

$\psi(z) = \sum z^kp_k = E z^k$

$z \in \mathds{C};\ |z| \leq 1$

$\frac{d^k\psi}{dz^k}|_{z=0} = k!p_k$

$\frac{d^k\psi}{dz^k}|_{z=1} = \sum\limits_{k = n}^{+ \infty} {k \choose n} \cdot z^{k-n} \cdot p_k |_{z = 1} = E(\xi(\xi - 1)(\xi - 2) \ldots (\xi - n + 1))$

$\frac{d^k\psi}{dz^k} = E_{\xi^2 - \xi} = E_{\xi^2} - E_\xi$

$\psi_{a\xi + b}(z) = z^b \cdot \psi_{\xi}(z^a)$

$\psi_{\xi + \eta}(z) = \psi_\xi(z) \cdot \psi_\eta(z)$

\textbf{Def.} $P(\xi = k) = \frac{e^{-\lambda}\lambda^k}{k!}$ -- распределение Пуассона

$\psi_\xi(z) = \sum\limits_{k=0}^{+\infty} e^{-\lambda} \frac{\lambda^k}{k!}z^k = e^{-\lambda} \sum\limits_{k=0}^{+\infty} \frac{(\lambda z)^k}{k!} = e^{-\lambda}e^{\lambda z} = e^{\lambda(z-1)}$

$\lambda e^{\lambda(z-1)}|_{z=1} = \lambda$

$D_\xi = E_{\xi^2} - E_\xi^2 = \lambda + \lambda^2 - \lambda^2 = \lambda$

$\varphi(t) = Ee^{it\xi}$ -- характеристическая функция

$\varphi_{a\xi + b}(t) = e^{itb} \cdot \varphi_\xi(at)$

$\varphi_{\xi + \eta}(t) = \varphi_\xi(t) \cdot \varphi_\eta(t)$

$i^{-n}\frac{d^n\varphi(t)}{dt^n}|_{t=0} = E\xi^n$

$f_{N_{0, 1}} = \frac{1}{\sqrt{2\pi}}e^{-\frac{x^2}{2}}$

$\varphi_{N_{0, 1}}(t) = e^{-\frac{t^2}{2}}$

$E_{N_{0, 1}} = 0$

$D_{N_{0, 1}} = 1$

\vspace{5mm}

Мы можем использовать характеристические уравнения для:

\begin{itemize}
    \item Рекуреннтных соотношений
    \item Дифференциальных уравнений
    \item Случайных процессов
\end{itemize}

\vspace{5mm}

$\xi$ -- с.в., $\xi \geq 0$

$\forall \varepsilon > 0;\ P(\xi \geq \varepsilon) \leq \frac{E\xi}{\varepsilon}$ -- неравенство Маркова

$\mathbb{1}(x) = \left\{ \begin{gathered}
    1,\ x \in A \\
    0,\ x \not\in A
\end{gathered} \right.$

$\mathbb{1}_{[0; \varepsilon]}(\xi) + \mathbb{1}_{[\varepsilon; +\infty)}(\xi) = 1$

$E_\xi = E_{\xi \cdot 1} = E_{\xi \cdot \mathbb{1}_{[0; \varepsilon]}(\xi)} + E_{\xi \cdot \mathbb{1}_{[\varepsilon; +\infty)}(\xi)} \geq E_{\xi \cdot \mathbb{1}_{[\varepsilon; +\infty)}(\xi)} \geq \varepsilon \cdot P(\xi \geq \varepsilon)$

$\xi$ -- с.в., $D_\xi < +\infty$

$\forall \delta\ P(|\xi - E\xi| \geq \delta) \leq \frac{D\xi}{\delta^2}$ -- неравенство Чебышева

$P(|\xi - E\xi| \geq \delta) = P((\xi - E\xi)^2 \geq \delta^2) \leq \frac{E(\xi - E\xi)^2}{\delta^2} = \frac{D\xi}{\delta^2}$

$\xi_1, \ldots \xi_n \ldots$ -- пнорсв 

\begin{enumerate}
    \item $\xi_i \rightarrow \xi$ в среднекрадратичном смысле, если $E(\xi_n - \xi)^2 \rightarrow 0$ при $n \rightarrow \infty$
    \item $\xi_i \rightarrow \xi$ по вероятности, если $P(|\xi_n - \xi| \geq \varepsilon) \rightarrow 0$ при $n \rightarrow \infty$
    \item $\xi_i \rightarrow \xi$ почти наверное, если $P(\omega : \xi_n(\omega) \rightarrow \xi(\omega)) = 1$
\end{enumerate}

\textbf{Th.} ЗБЧ Чебышева

$E_{\xi_i} = a;\ D_\xi < +\infty$

$\overline{\xi_n} = \frac{1}{n} \sum\limits_{i=1}^n \xi_i$

$\overline{\xi_n} \rightarrow E_\xi$ по всем пунктам. Это закон {\Huge Больших} чисел

$N(M, \sigma^2)$ -- нормальное распределение

\begin{center}
    \textbf{Доказательство}
\end{center}

Докажем только среднеквадратичную сходимость, остальное сложно, у нас лапки

$E_{\overline{\xi_n}} = \frac{1}{n} E_{\sum \xi_i} = \frac{1}{n} \sum E_{\xi_i} = \frac{na}{n} = a$

$E_{(\overline{\xi_n} - a)^2} = D_{\overline{\xi_n}} = \frac{1}{n} D_{\sum \xi_i} = \frac{1}{n} \sum D_{\xi_i} = \frac{D_{\xi_1}}{n} \rightarrow 0$

\vspace{5mm}

\textbf{Th.} ЦПТ Ляпунова

$S_n = \sum\limits_{i=1}^n \xi_i$

$Z_n = \frac{S_n - na}{\sqrt{n\sigma^2}}$. $F_n = P(Z_n < x)$

$\sup|F_n(x) - N_{0, 1}(x)| \rightarrow 0$. Другими словами $Z_n \sim N_{0, 1}$ при $n \rightarrow \infty$

$S_n \sim N_{na, n\sigma^2}$ при $n \rightarrow \infty$

\vspace{5mm}

$V$ -- вершина $|V| < +\infty$

$E = \{e_i\}$ -- ребра

$e_i = (V_i, V_j)$, $V_i, V_j \in V$

$I : V \times E$

$I(v, e)$ если $e = \left[ \begin{gathered}
    (v, *) \\
    (*, v)
\end{gathered} \right.$

$|V|$, $E \subset V \times V$

\begin{enumerate}
    \item $e(V_1, V_2) \Rightarrow e(V_2, V_1)$
    \item $\overline{e(V_1, V_1)}$
\end{enumerate}

$G = (V, E)$ -- граф (бинарное отноешние смежности)

\begin{center}
    \begin{tabular}{|c|c|c|c|c|c|}
    \hline
    & $V_1$ & $V_2$ & $V_3$ & $V_4$ & $V_5$ \\
    \hline
    $V_1$ & 0 & 1 & 1 & 0 & 0 \\
    \hline
    $V_2$ & 1 & 0 & 1 & 0 & 0 \\
    \hline
    $V_3$ & 1 & 1 & 0 & 1 & 1 \\
    \hline
    $V_4$ & 1 & 0 & 1 & 0 & 0 \\
    \hline
    $V_5$ & 0 & 0 & 1 & 0 & 0 \\
    \hline
    \end{tabular}
\end{center}

$S(V_i, V_j)$ существует, если существуют

$V_i, E_{\alpha_1}, V_{\alpha_1}, E_{\alpha_2}, \ldots, E_{\alpha_{n-1}}, V_j$

$\tilde{G}$ -- подграф $G$

$V_{\tilde{G}} \subset V_{G}$

$\forall e \in E_{\tilde{G}} \Rightarrow e \in E_G$

\end{document}