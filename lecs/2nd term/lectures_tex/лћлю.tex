\documentclass[12pt]{article}
\usepackage{config}
\usepackage{subfiles}

\def\multiset#1#2{\ensuremath{\left(\kern-.3em\left(\genfrac(){0pt}{}{#1}{#2}\right)\kern-.3em\right)}}
\def\divby{%
  \mathrel{\text{\vbox{\baselineskip.65ex\lineskiplimit0pt\hbox{.}\hbox{.}\hbox{.}}}}%
}
\newcommand{\q}[1]{\langle #1 \rangle}

\begin{document}

\section*{План на 3 модуль (или 2 сем...)}

\begin{enumerate}
    \item Множества
    \item ЧУМ
    \item Исчисление высказываний
    \item Исчисление предикатов
    \item Теория кодирования
\end{enumerate}

Почитать можно А. Х. Шеня

\section*{Множества}

\begin{enumerate}
    \item $x \in A;\ y \not \in A$
    \item Арифметика множеств: $\bigcup, \bigcap, \backslash, \triangle$
    \item $\varnothing$
    \item $A = \{ a, b, c \};\ B = \{d\} \bigcup A$
    \item $A \subset B \Leftrightarrow \forall x \in A \Rightarrow x \in B$
\end{enumerate}

\begin{Remark}{}
    Чисто синтаксически вот такой бред: $\{ \varnothing, \{ \varnothing \}, \{ \varnothing, \{ \varnothing \} \} \}$ имеет смысл
\end{Remark}

$X$ -- множество: $X \neq \varnothing$. Рассмотрим $x \in X$

$Term(x)$ -- проблема, потому что мы не знаем, к каким характеристикам обращаемся и вообще не понятно, что мы выбрали

Спасают аксиомы ZFC

\begin{defin}{Равномощность}
    $A, B$ -- равномощны $\Leftrightarrow \exists f : A \rightarrow B$ -- биекция

    А что с бесконечностями? Давайте возьмем функцию $f : N \rightarrow 2N$

    Хотя множество четных чисел -- подмножество всех, но они равномощны, т.к. $f$ -- биекция
\end{defin}

\begin{defin}{Характеристическая функция}
    $X$ -- множество. Есть $\chi : X \rightarrow \{0, 1\}$, т.е. $\chi(x) = \begin{cases}
        1,\ x \in X \\
        0,\ x \not\in X
    \end{cases}$ -- характеристическая функция
\end{defin}

А пусть $X \subset Y$

\begin{itemize}
    \item произведение характеристических функций $X$ и $Y$ -- это характеристическая функция $X \bigcap Y$
    \item $1 - \chi(x)$ -- характеристическая функция дополнения $X$
    \item $max(\chi_X(x), \chi_Y(x))$ -- характеристическая функция $X \bigcup Y$
    \item $|X| = \sum\limits_{x \in Y} \chi_X(x)$
\end{itemize}

\begin{Example}{}
    Возьмем $2^N;\ B = \{ 0, 1 \}$ и $B^\infty$

    Равномощны ли они? Берем $x \in 2^N$, теперь $b_i = \begin{cases}
        1,\ i \in x \\
        0,\ i \not\in x
    \end{cases}$
\end{Example}

\begin{defin}{Счетное множество}
    $X$ -- счетное, если $X$ равномощно $N$
\end{defin}

\begin{Example}{}
    Например, множество целых чисел счетно, т.к. $x \in Z \Rightarrow \begin{cases}
        2x,\ x \geq 0 \\
        -2x + 1,\ x < 0
    \end{cases}$
\end{Example}

\begin{propos}{}
    \begin{enumerate}
        \item $X$ -- счетно и $Y \subset X \Rightarrow Y$ или счетно, или конечно
        \item $X$ -- бесконечно. Тогда $\exists Y$ -- счетное: $Y \subset X$
        \item $X_1, \ldots X_n \ldots$ -- конечные или счетные. Тогда $\bigcup X_i$ -- конечное или счетное
    \end{enumerate}
\end{propos}

\textit{Доказательство:}

\begin{enumerate}
    \item $X$ -- счетно, т.е. соответствует последовательности $\{x_1, \ldots x_n \ldots\} = \xi$
    
    Возьмем $\xi \cdot \xi(Y)$. Т.е. что-то типа $\{ 0, 0 \ldots x_{i_1}, 0 \ldots x_{i_2}, 0 \ldots \}$ который равносилен $y_1, y_2, \ldots y_n \ldots = Y$

    В свою очередь эта штука либо конечна, либо счетна, т.к. счетен $X$

    \item Просто выбираем по 1 элементу из $X$. Если они кончатся на каком-то шаге -- $X$ не бесконечно
    
    \item Рисуем табличку. Берем элемент (1, 1), потом (1, 2), потом (2, 1), потом (1, 3) и так далее. То есть по диагоналям. Так переберем вообще все элементы (если не понятно, погуглите метод Кантора)
\end{enumerate}

\begin{Exercise}{}
    В качестве следствия попробуйте построить явную биекцию между множеством рациональных чисел и натуральных
\end{Exercise}

\begin{theo}{}
    $A$ -- бесконечно, $B$ -- нбчс, т.е. $B$ -- конечно или счетно

    $A \bigcup B$ равномощно $A$
\end{theo}

\textit{Доказательство:}

$\exists Y \subset A$ -- счетное

$Y$ и $Y \bigcup B$ -- равномощны

$A \bigcup B = (A \backslash Y) \bigcup (Y \bigcup B)$

$A = Y \bigcup (A \backslash Y)$

Биекция между $Y$ и $Y \bigcup B$ сущесвтует, значит $A$ и $A \bigcup B$ равномощны

\begin{Example}{}
    $[0; 1]$ и $B^\infty$. Равномощны ли? Да. Последовательность единиц и нулей -- это бинпоиск числа
\end{Example}

\end{document}