\documentclass[12pt]{article}
\usepackage{config}
\usepackage{subfiles}

\def\multiset#1#2{\ensuremath{\left(\kern-.3em\left(\genfrac(){0pt}{}{#1}{#2}\right)\kern-.3em\right)}}
\def\divby{%
  \mathrel{\text{\vbox{\baselineskip.65ex\lineskiplimit0pt\hbox{.}\hbox{.}\hbox{.}}}}%
}
\newcommand{\q}[1]{\langle #1 \rangle}

\begin{document}

\tableofcontents
\newpage

\begin{flushright}
    Конспект Шорохова Сергея

    Если нашли опечатку/ошибку - пишите @le9endwp
\end{flushright}

\section{Интегральчики}

\subsection{\S 1. Первообразная и неопределенный интеграл}

\begin{defin}{Первообразная функция}
    $f : \q{a, b} \rightarrow \R;\ \ F: \q{a, b} \rightarrow \R$ 

    $F$ -- первообразная функция $f$, если $F$ дифференцируема на $\q{a, b}$ и $F'(x) = f(x)$ при всех $x \in \q{a, b}$

    \begin{Example}{}
        $f(x) = \cos{x}$

        $F(x) = \sin{x}$ 
    \end{Example}
\end{defin}

\begin{propos}{}
    Не всякая функция имеет первообразную

    \begin{Example}{}
        $f(x) = \begin{cases}
            0,\ x \in (-1, 0] \\
            1,\ x \in (0, 1)
        \end{cases}$
    \end{Example}
\end{propos}

\begin{propos}{}
    Непрерывная на $\q{a, b}$ функция имеет первообразную
\end{propos}

\begin{theo}{}
    $f, F : \q{a, b} \rightarrow \R, F$ -- первообразная $f$. Тогда

    \begin{enumerate}
        \item $F + C$ -- первообразная $f$
        \item Если $\Phi$ -- первообразная $f$, то $\Phi = F + C$ для некоторой константы $C$
    \end{enumerate}
\end{theo}

\textit{Доказательство:}

\begin{enumerate}
    \item $(F + C)' = F' = f$
    \item $\Phi' = f = F'$
    
    $g = \Phi - F$

    $g' = 0 \Rightarrow g = C \Rightarrow \Phi = F + C$
\end{enumerate}

\begin{defin}{Неопределенный интеграл}
    Неопределенный интеграл -- множество первообразных функции $f$

    Обозначение: $\int f(x)dx$
\end{defin}

\begin{Remark}{}
    Для доказательства равенства $\int f(x)dx = F(x) + C$ достаточно проверить, что 
    
    $F'(x) = f(x)$
\end{Remark}

\textbf{Действия с множествами функций:}

$A$ и $B$ -- множества функций $\q{a, b} \rightarrow \R$

$\lambda \in \R,\ h : \q{a, b} \rightarrow \R$

\begin{enumerate}
    \item $A + B = \{f + g : f \in A,\ g \in B\}$
    \item $\lambda A = \{\lambda f : f \in A\}$
    \item $A + h = \{f + h : f \in A\}$
    \item $(A)' = \{f' : f \in A\}$
    
    \begin{Example}{}
        $(\int f(x)dx)' = \{ f \}$
    \end{Example}
\end{enumerate}

\textbf{Таблица интегралов:}

\begin{enumerate}
    \item $\int adx = ax + C$
    \item $\int x^pdx = \frac{x^{p + 1}}{p + 1} + C,\ p \neq -1$
    \item $\in \frac{dx}{x} = \ln{|x|} + C$
    \item $\int a^xdx = \frac{a^x}{\ln{a}} + C;\ a > 0,\ a \neq 1$
    \item $\int \sin{x}dx = -\cos{x} + C$
    \item $\int \cos{x}dx = \sin{x} + C$
    \item $\int \frac{dx}{x^2 + 1} = \arctg{x} + C$
    \item $\int \frac{dx}{\sqrt{1 - x^2}} = \arcsin{x} + C$
\end{enumerate}

\begin{theo}{Линейность интеграла}
    $f, g : \q{a, b} \Rightarrow \R$ имеют первообразные

    $\alpha, \beta \in \R$, не равные нулю одновременно

    Тогда $\int (\alpha f(x) + \beta g(x))dx = \alpha \int f(x)dx + \beta \int g(x)dx$
\end{theo}

\textit{Доказательство:}

$F$ и $G$ -- первообразные

Правая часть $= \{ \alpha F(x) + \beta G(x) + C : C \in \R \}$

$(\alpha F(x) + \beta G(x) + C)' = \alpha F'(x) + \beta G'(x) = \alpha f(x) + \beta g(x)$

\begin{theo}{Замена переменной в интеграле}
    $f : \q{a, b} \rightarrow \R,\ F$ -- первообразная
    
    $\varphi : \q{c, d} \rightarrow \q{a, b}$ -- дифференцируемая функция

    Тогда $\int f(\varphi(x))\varphi'(x)dx = F(\varphi(x)) + C$
\end{theo}

\textit{Доказательство:}

$(F(\varphi(x)) + C)' = F'(\varphi(x))\varphi'(x) = f(\varphi(x))\varphi'(x)$

\begin{Remark}{}
    $y = \varphi(x);\ \ dy = \varphi'(x)dx$

    $\frac{dy}{dx} = y'$

    $\int f(\varphi(x)) \varphi'(x)dx = \int f(y)dy = F(y) + C = F(\varphi(x)) + C$
\end{Remark}

\begin{Example}{}
    \begin{enumerate}
        \item $\int \frac{x}{x^2 + 1}dx = \frac{1}{2} \int \frac{(x^2 + 1)'}{x^2 + 1}dx = \frac{1}{2} \int \frac{dy}{y} = \ln{|y|} + c = \ln{|x^2 + 1|} + C$
        
        Здесь $y = \varphi(x) = x^2 + 1$

        \item $\int \frac{dx}{\sin{x}} = \int \frac{dx}{2\sin{\frac{x}{2}}\cos{\frac{x}{2}}} = \int \frac{dy}{\sin{y}\cos{y}} = \int \frac{dy}{\tg{y}\cos^2{y}} = \int \frac{(\tg{y})'}{tg{y}}dy = \int \frac{dz}{z} = \ln{|z|} + C =$
        
        $= \ln{|\tg{y}|} + C = \ln{|\tg{\frac{x}{2}}|} + C$

        Здесь $y = \frac{x}{2}$ и $z = \tg{y}$

        \item $\int \frac{dx}{1 + \sqrt[3]{x}} = \int \frac{3t^2dt}{1 + t} = 3\int \frac{t^2 - 1 + 1}{t + 1}dt = 3\int (t - 1 + \frac{1}{t + 1})dt = 3(\int tdt - \int dt + \int \frac{dt}{t + 1}) =$
        
        $= 3t^2 - 3t + 3\int \frac{d(t + 1)}{t + 1} = 3t^2 - 3t + 3\ln{|t + 1|} + C$
    \end{enumerate}
\end{Example}

\begin{theo}{Интегрирование по частям}
    $f, g : \q{a, b} \rightarrow \R$ дифференцируемые

    Если $f'g$ имеет первообразную, то $\int f(x)g'(x)dx = f(x)g(x) - \int f'(x)g(x)dx$
\end{theo}

\textit{Доказательство:}

$H$ -- первообразная функции $f'g$

$(fg - H + C)' = (fg)' - H' = f'g + fg' - f'g = fg'$

\begin{nota}{Традиционная запись формулы}
    $\int udv = uv - \int vdu$

    $\begin{cases}
        du = u'(x)dx \\
        dv = v'(x)dx
    \end{cases}$
\end{nota}

\begin{Example}{}
    \begin{enumerate}
        \item $\int \ln{x}dx = x\ln{x} - \int x \frac{dx}{x} = x\ln{x} - \int dx = x\ln{x} - x + C$
        
        Здесь $u = \ln{x},\ v = x$ и $du = (\ln{x})'dx = \frac{dx}{x}$ 

        \item $\int x^2 e^x dx = \int x^2 de^x = x^2 e^x - \int 2x e^x dx = x^2 e^x - 2\int xde^x = x^2e^x - 2(xe^x - \int e^xdx) = \\ = x^2e^x - 2xe^x + 2e^x + C$
        
        Здесь сначала берем $u = x^2, v = e^x$, а потом $u = x, v = e^x$
    \end{enumerate}
\end{Example}

\subsection{\S 2. Площадь}

\begin{defin}{Площадь}
    $F$ -- семейство всех ограниченных подмножеств плоскости

    Прямоугольник $\q{a_1, b_1} \times \q{a_2, b_2}$, площадь прямоугольника $(b_1 - a_1)(b_2 - a_2)$

    Площадь $S : F \rightarrow [0, + \infty)$

    \begin{enumerate}
        \item $S(\q{a_1, b_1} \times \q{a_2, b_2}) = (b_1 - a_1)(b_2 - a_2)$
        \item $S(E) = S(E_1) + S(E_2)$, если $E = E_1 \bigcup E_2,\ E_1 \bigcap E_2 = \varnothing$
    \end{enumerate}
\end{defin}

\begin{theo}{Свойство}
    Если $\tilde{E} \subset E$, то $S(\tilde{E}) \leq S(E)$
\end{theo}

\textit{Доказательство:}

$E = \tilde{E} \bigcup (E \setminus \tilde{E})$

$S(E) = S(\tilde{E}) + S(E \setminus \tilde{E}) \geq S(\tilde{E})$

\begin{defin}{(Квази)площадь}
    $\sigma : F \rightarrow [0, +\infty)$

    \begin{enumerate}
        \item $\sigma(\q{a_1, b_1} \times \q{a_2, b_2}) = (b_1 - a_1)(b_2 - a_2)$
        \item $\sigma(E) = \sigma(E_-) + \sigma(E_+)$, если $E_-$ и $E_+$ множества, получающиеся в результате разбиения $E$ вертикальной (горизонтальной) прямой
        \item Если $\tilde{E} \subset E$, то $\sigma(\tilde{E}) \leq \sigma(E)$
    \end{enumerate}
\end{defin}

\begin{Remark}{Свойство}
    Формула 2) верна и если $E_- \bigcap E_+ \neq \varnothing$

    Например, линию разбиения можно считать относящейся и к левой (верхней), и к правой (нижней) части
\end{Remark}

\textit{Доказательство:}

$e = E_- \bigcap E_+$, $\sigma(e) = 0$

$\sigma(E_+) = \sigma(E_+ \setminus e) + \sigma(e \bigcap E_+) = \sigma(E_+ \setminus e)$

$\sigma(E_-) + \sigma(E_+) = \sigma(E_-) + \sigma(E_+ \setminus e) = \sigma(E_- \bigcup (E_+ \setminus e)) = \sigma(E_- \bigcup E_+) = \sigma(E)$

\begin{Example}{Примеры площадей $E \in F$}
    \begin{itemize}
        \item Рассмотрим покрытие $E$ конечным числом прямоугольников $P_i$ (т.е. $\bigcup\limits_{i = 1}^n P_i \supset E$)
        
        $\sigma_1(E) = \inf\{ \sum\limits_{i = 1}^n \sigma(P_i) : \bigcup\limits_{i = 1}^n P_i \supset E \}$

        \item Рассмотрим покрытие $E$ последовательностью прямоугольников $P_i$ (т.е. $\bigcup\limits_{i = 1}^{\infty} P_i \supset E$)
        
        $\sigma_2(E) = \inf\{ \sum\limits_{i = 1}^{\infty} \sigma(P_i) : \bigcup\limits_{i = 1}^{\infty} P_i \supset E \}$

        \item Ясно, что $\sigma_1(E) \geq \sigma_2(E)$
        
        Но, если $E = ([0, 1] \bigcap \Q) \times ([0, 1] \bigcap \Q)$, то $\begin{cases}
            \sigma_1(E) = 1 \\
            \sigma_2(E) = 0
        \end{cases}$
    \end{itemize}
\end{Example}

\begin{theo}{}
    \begin{enumerate}
        \item $\sigma_1$ -- площадь
        \item $\sigma_1$ не меняется при параллельном переносе
    \end{enumerate}
\end{theo}

\textit{Доказательство:}

\textbf{1)}

\begin{enumerate}
    \item $\sigma_1(\q{a, b} \times \q{c, d}) = (b - a)(d - c)$
    
    Поскольку $[a, b] \times [c, d]$ -- покрытие $P$, $\sigma_1(P) \leq (b - a)(d - c)$

    В обратную сторону красиво доказано АИ. Там рисуночки, посмотрите!

    \item $E = E_- \bigcup E_+ \Rightarrow \sigma_1(E) = \sigma_1(E_-) + \sigma_1(E_+)$
    
    \begin{itemize}
        \item[$\leq :$] Если $P_1^+, \ldots P_m^+$ -- покрытие $E_+$, для которого $\sum\limits_{i = 1}^m \sigma(P_i^+) < \sigma_1(E_+) + \eps$
        
        А $P_1^-, \ldots P_n^-$ -- покрытие $E_-$, для которого $\sum\limits_{i = 1}^n \sigma(P_i^-) < \sigma_1(E_-) + \eps$, то

        $P_1^-, P_2^-, \ldots P_n^-, P_1^+, P_2^+, \ldots P_m^+$ -- покрытие $E$, для которого 
        
        $\sigma_1(E) \leq \sum\limits_{i = 1}^{n + m} \sigma(P_i) < \sigma_1(E_-) + \sigma_1(E_+) + 2\eps \Rightarrow \sigma_1(E) < \sigma_1(E_-) + \sigma_1(E_+) + 2\eps$

        \item[$\geq :$] Пусть $P_1, P_2, \ldots P_n$ -- покрытие $E$
        
        Разобьем $P_i$ на $P_i^-$ и $P_i^+$

        $\sigma(P_i) = \sigma(P_i^-) + \sigma(P_i^+)$

        $P_1^\pm, P_2^\pm, \ldots P_n^\pm$ -- покрытие $E^\pm$

        $\sum\limits_{i = 1}^n \sigma(P_i^\pm) \geq \sigma_1(E^\pm)$

        $\sum\limits_{i = 1}^n (\sigma(P_i^-) + \sigma(P_i^+)) \geq \sigma_1(E_-) + \sigma_1(E_+)$
    \end{itemize}

    \item $\tilde{E} \subset E \Rightarrow \sigma_1(\tilde{E}) \leq \sigma_1(E)$
    
    Если $\bigcup\limits_{i = 1}^n P_i \supset E$, то $\bigcup\limits_{i = 1}^n P_i \supset \tilde{E} \Rightarrow$ класс покрытий $\tilde{E}$ шире, чем класс покрытий $E$
\end{enumerate}

\textbf{2)}

Пусть $\tilde{E}$ -- параллельный перенос $E$ на вектор $\overrightarrow{v}$

$P_1, P_2, \ldots P_n$ -- покрытие $E$. Пусть $\tilde{P_i}$ -- параллельный перенос $P_i$ на вектор $\overrightarrow{v}$

Тогда $\tilde{P_1}, \tilde{P_2}, \ldots \tilde{P_n}$ -- покрытие $\tilde{E}$ и $\sum\limits_{i = 1}^n \sigma(P_i) = \sum\limits_{i = 1}^n \sigma(\tilde{P_i})$

\begin{defin}{}
    $f : [a, b] \rightarrow \R$

    $f_+ := \max\{f, 0\}$, т.е. $f_+(x) = \max\{f(x), 0\}$

    $f_- := \max\{-f, 0\}$, т.е. $f_-(x) = \max\{-f(x), 0\}$

    Свойства:

    \begin{enumerate}
        \item $f_\pm \geq 0$
        \item $f = f_+ - f_-$
        
        $|f| = f_+ + f_-$

        \item $f_+ = \frac{f + |f|}{2}$ и $f_- = \frac{|f| - f}{2}$
        \item Если $f \in C[a, b]$, то $f_\pm \in C[a, b]$
    \end{enumerate}
\end{defin}

\begin{defin}{Подграфик функции}
    $f : [a, b] \rightarrow \R,\ f \geq 0$

    Подграфик функции $f$ -- $P_f = \{ (x, y) \in \R : x \in [a, b],\ 0 \leq y \leq f(x) \}$
\end{defin}

\begin{defin}{Определенный интеграл}
    $\sigma$ -- зафиксированная квазиплощадь

    $f \in C[a, b]$ (пока что так)

    $\int\limits_a^b f = \int\limits_a^b f(x)dx  := \sigma(P_{f_+}) - \sigma(P_{f_-})$
\end{defin}

\textbf{Свойства:}

    \begin{enumerate}
        \item $\int\limits_a^a f = 0$
        \item $\int\limits_a^b 0 = 0$
        \item Если $f \geq 0$, то $\int\limits_a^b f = \sigma(P_f)$
        \item $\int\limits_a^b (-f) = -\int\limits_a^b f$
        
        \textit{Доказательство:}

        $(-f_+) = \max\{-f, 0\} = f_-$

        $(-f_-) = \max\{f, 0\} = f_+$

        $\int\limits_a^b (-f) = \sigma(P_{f_-}) - \sigma(P_{f_+}) = -\int\limits_a^b f$

        \item $\int\limits_a^b (c) = c(b - a)$
        
        \textit{Доказательство:}

        $c > 0 \Rightarrow \int\limits_a^b c = P(\text{прямоугольника}) = c(b - a)$

        \item Если $a < b,\ f \geq 0$ и $\int\limits_a^b f = 0$, то $f \equiv 0$
        
        \textit{Доказательство:} (от противного)

        Пусть $f(x_0) > 0$. Из непрерывности $f$ в $x_0 \Rightarrow \varepsilon = \frac{f(x_0)}{2} \Rightarrow \exists \delta > 0 : \forall |x - x_0| < \delta \Rightarrow \\
        \Rightarrow |f(x) - f(x_0)| < \varepsilon = \frac{f(x_0)}{2} \Rightarrow P_f \supset [x_0 - \delta, x_0 + \delta] \times [0, \frac{f(x_0)}{2}] \Rightarrow \\
        \Rightarrow \sigma(P_f) \geq \sigma(\text{прямоугольника}) = 2\sigma \frac{f(x_0)}{2} > 0$ Противоречие
    \end{enumerate}

\subsection{\S 3. Свойства интеграла}

\begin{nota}{Обозначение}
    $P_g(E)$ -- подграфик функции $g \geq 0$ над множество $E$, т.е. 

    $P_g(E) := \{ (x, y) \in \R^2 : x \in E,\ 0 \leq y \leq g(x) \}$
\end{nota}

\begin{theo}{Аддитивность интеграла}
    $f \in C[a, b]$ и $c \in [a, b]$

    Тогда $\int\limits_a^b f = \int\limits_a^c f + \int\limits_c^b f$
\end{theo}

\textit{Доказательство:}

$\int\limits_a^b f = \sigma(P_{f_+}) - \sigma(P_{f_-}) = \sigma(P_{f_+}([a, c])) + \sigma(P_{f_+}([c, b])) - \sigma(P_{f_-}([a, c])) - \sigma(P_{f_-}([c, b])) = \int\limits_a^c f + \int\limits_c^b f$

\begin{theo}{Следствие}
    $f \in C[a, b]$, $a \leq c_1 \leq c_2 \leq \ldots \leq c_n \leq b$. Тогда

    $\int\limits_a^b f = \int\limits_a^{c_1} f + \int\limits_{c_1}^{c_2} f + \ldots + \int\limits_{c_{n - 1}}^{c_n} f + \int\limits_{c_n}^b f$
\end{theo}

\textit{Доказательство:}

Индукция по $n$

\begin{theo}{Монотонность интеграла}
    $f, g \in C[a, b]$ и $f \leq g$ на $[a, b]$

    Тогда $\int\limits_a^b f \leq \int\limits_a^b g$
\end{theo}

\textit{Доказательство:}

$f \leq g \Rightarrow f_+ \leq g_+ \Rightarrow P_{f_+} \subset P_{g_+}$, а еще $-g \leq -f \Rightarrow g_- \leq f_- \Rightarrow P_{g_-} \subset P_{f_-}$

Значит $\sigma (P_{f_+}) \leq \sigma (P_{g_+})$ и $\sigma (P_{g_-}) \leq \sigma (P_{f_-})$

$\int\limits_a^b f = \sigma (P_{f_+}) - \sigma (P_{f_-}) \leq \sigma (P_{g_+}) - \sigma (P_{g_-}) = \int\limits_a^b g$

\begin{theo}{Следствия}
    \begin{enumerate}
        \item $f \in C[a, b] \Rightarrow \min\limits_{[a, b]} f \cdot (b - a) \leq \int\limits_a^b f \leq \max\limits_{[a, b]} f \cdot (b - a)$
        
        \textit{Доказательство:}

        $\min f \leq f \leq \max f$ и монотонность интеграла для двух постоянных функций и $f$

        \item $f \in C[a, b] \Rightarrow |\int\limits_a^b f| \leq \int\limits_a^b |f|$
        
        \textit{Доказательство:}

        $-|f| \leq f \leq |f| \xRightarrow{\text{монотонность}} -\int |f| = \int\limits_a^b (-|f|) \leq \int\limits_a^b f \leq \int\limits_a^b |f|$
    \end{enumerate}
\end{theo}

\begin{theo}{(Первая) (интегральная) теорема о среднем}
    $f \in C[a, b]$. Тогда существует $c \in [a, b] : \int\limits_a^b f = f(c)(b - a)$
\end{theo}

\textit{Доказательство:}

$\min f \leq \frac{1}{b - a} \int\limits_a^b f \leq \max f$, но множество значений $f$ на $[a, b]$ -- это отрезок $[\min f, \max f]$

Следовательно, число $\frac{1}{b - a} \int\limits_a^b f$ -- есть значение функции $f$ в какой-то точке $[a, b]$. Возьмем эту точку в качестве $c$

\begin{defin}{Среднее значение функции на отрезке}
    Среднее значение функции $f$ на отрезке $[a, b]$ -- это $\frac{1}{b - a} \int\limits_a^b f$
\end{defin}

\begin{defin}{Интеграл с переменным верхним пределом}
    $f \in C[a, b]$

    $\Phi(x) := \int\limits_a^x f$, где $x \in [a, b]$
\end{defin}

\begin{Remark}{}
    $\Phi(a) = 0$
\end{Remark}

\begin{defin}{Интеграл с переменным нижним пределом}
    $f \in C[a, b]$

    $\Psi(x) := \int\limits_x^b f$, где $x \in [a, b]$
\end{defin}

\begin{Remark}{}
    $\Psi(b) = 0$

    $\Phi(x) + \Psi(x) = \int\limits_a^b f$ (это аддитивность $\int$)
\end{Remark}

\begin{theo}{Теорема Барроу}
    Если $f \in C[a, b],\ \Phi(x) := \int\limits_a^x f$, то $\Phi$ -- первообразная функции $f$
\end{theo}

\textit{Доказательство:}

Надо доказать, что $\Phi'(x) = f(x)$. Пусть $x < y$

$R(y) = \frac{\Phi(y) - \Phi(x)}{y - x} = \frac{1}{y - x} (\int\limits_a^y f - \int\limits_a^x f) = \frac{1}{y - x} \int\limits_x^y f \stackrel{\text{т-ма о среднем}}{=} f(c_y)$, где $c_y \in [x, y]$

Возьмем последовательность $y_n > x$ и $\lim y_n = x$

$\Phi_+'(x) = \lim\limits_{y \to x_+} R(y) = \lim\limits_{n \to \infty} R(y_n) = \lim\limits_{n \to \infty} f(c_{y_n}) = f(x)$, т.к. $x \leq c_{y_n} \leq y_n \rightarrow x$

Если же $y < x$, то нужно смотреть на $\frac{1}{x - y} \int\limits_y^x f$ и дальше ровно так же 

Следовательно, $\Phi'(x) = f(x)$

\begin{theo}{Следствия}
    \begin{enumerate}
        \item $\Psi(x) := \int\limits_x^b f \Rightarrow \Psi'(x) = -f(x)$
        
        \textit{Доказательство:}

        $\Phi(x) + \Psi(x) = const$

        \item Если $f \in C\q{a, b}$, то у $f$ есть первообразная на $\q{a, b}$
        
        \textit{Доказательство:}

        Возьмем $c \in (a, b)$ и $F(x) := \begin{cases}
            \int\limits_c^x f,\ x \geq c \\
            -\int\limits_x^c f,\ x \leq c
        \end{cases}$
        
        Тогда $\begin{gathered}
            F'(x) = f(x) \text{при} x \geq c \text{ (по теореме Барроу)} \\
            F'(x) = -f(x) \text{при} x \leq c \text{ (по следствию 1)} \\
            F_+'(c) = f(c) = F_-'(c)
        \end{gathered}$
    \end{enumerate}
\end{theo}

\begin{theo}{Формула Ньютона-Лейбница}
    $f \in C[a, b],\ F$ -- первообразная $f$

    Тогда $\int\limits_a^b f = F(b) - F(a)$
\end{theo}

\textit{Доказательство:}

$\Phi(x) := \int\limits_a^x f$ -- первообразная $f$ (по теореме Барроу) $\Rightarrow \Phi = F + C$ для некоторой $C \in \R$

$\Rightarrow \int\limits_a^b f = \Phi(b) = F(b) + C = F(b) - F(a)$, т.к. $0 = \Phi(a) = F(a) + C$

\begin{nota}{Обозначение}
    $F|_a^b := F(b) - F(a)$ подстановка

    $\int\limits_a^b f = F|_a^b$
\end{nota}

\begin{theo}{Линейность интеграла}
    $f, g \in C[a, b],\ \alpha, \beta \in \R$

    Тогда $\int\limits_a^b (\alpha f + \beta g) = \alpha \int\limits_a^b f + \beta \int\limits_a^b g$
\end{theo}

\textit{Доказательство:}

Пусть $F$ и $G$ -- первообразные $f$ и $g$

Тогда $\alpha F + \beta G$ -- первообразная $\alpha f + \beta g \Rightarrow \int\limits_a^b (\alpha f + \beta g) = (\alpha F + \beta G)|_a^b = \alpha F|_a^b + \beta G|_a^b = \\
= \alpha \int\limits_a^b f + \beta \int\limits_a^b g$

\begin{theo}{Формула интегрирования по частям}
    $u, v \in C^1[a, b]$

    Тогда $\int\limits_a^b uv' = uv|_a^b - \int\limits_a^b u'v$
\end{theo}

\textit{Доказательство:}

Пусть $H$ -- первообразная $u'v$. Тогда $uv - H$ -- первообразная $uv'$

$(uv - H)' = u'v + uv' - H' = u'v + uv' - u'v = uv'$

$\int\limits_a^b uv' = (uv - H)|_a^b = uv|_a^b - H|_a^b = uv|_a^b - \int\limits_a^b u'v$

\begin{nota}{Соглашение}
    Если $a > b$, то $\int\limits_a^b f = -\int\limits_b^a f$
\end{nota}

\begin{theo}{Замена переменной в определенном интеграле}
    $f \in C\q{a, b},\ \varphi \in C^1\q{c, d},\ \varphi : \q{c, d} \rightarrow \q{a, b},\ p, q \in \q{c, d}$. Тогда

    $\int\limits_p^q f(\varphi(t))\varphi'(t)dtt = \int\limits_{\varphi(p)}^{\varphi(q)} f(x)dx$
\end{theo}

\textit{Доказательство:}

Пусть $F$ -- первообразная для $f$. Тогда $F \circ \varphi$ -- первообразная для $f(\varphi(t))\varphi'(t)$ (т.к. $(F(\varphi(t)))' = F'(\varphi(t))\varphi'(t)$)

$\int\limits_p^q f(\varphi(t))\varphi'(t)dt = (F \circ \varphi)|_p^q = F(\varphi(q)) - F(\varphi(p)) = \int\limits_{\varphi(p)}^{\varphi(q)} f$

\subsection{\S 4. Приложение формулы интегрирования по частям}

$W_n := \int\limits_0^{\frac{\pi}{2}} \cos^n{x}dx \stackrel{(*)}{=} \int\limits_0^{\frac{\pi}{2}} \sin^n{x}dx$

Пояснение к $(*)$: $\int\limits_0^\frac{\pi}{2} \sin^n{x}ds = \int\limits_0^\frac{\pi}{2} \cos^n{\frac{\pi}{2} - t}dt = - \int\limits_{\varphi(0)}^{\varphi(\frac{\pi}{2})} \cos^n{x}dx = -\int\limits_0^{\frac{\pi}{2}} \cos^n{x}dx$

Здесь $\varphi(t) = \frac{\pi}{2} - t$ и $\varphi'(t) = -1$

$W_0 = \frac{\pi}{2},\ W_1 = \int\limits_0^{\frac{\pi}{2}} \sin{x}dx = -\cos{x}|_0^{\frac{\pi}{2}} = 1,\ W_2 = \frac{1}{2} (\int\limits_0^\frac{\pi}{2} \cos^2 + \int\limits_0^\frac{\pi}{2} \sin^2) = \frac{\pi}{4}$

$W_n = \int\limits_0^\frac{\pi}{2} \sin^n{x}dx = \int\limits_0^\frac{\pi}{2} \sin^{n-1}x \cdot (-\cos{x})'dx = -\sin^{n-1}x \cos{x}|_0^\frac{\pi}{2} - \int\limits_0^\frac{\pi}{2} (n-1)\sin^{n-2}x \cos{x} (-\cos{x})dx = \\
= (n - 1) (\int\limits_0^\frac{\pi}{2} \sin^{n-2}x (1 - \sin^2{x})dx - \int\limits_0^\frac{\pi}{2} \sin^n{x}dx) = (n - 1)(W_{n-2} - W_n) \Rightarrow nW_n = (n - 1)W_{n-2}$

Если четно, то $W_{2n} = \frac{2n - 1}{2n}W_{2n - 2} = \frac{2n - 1}{2n} \cdot \frac{2n - 3}{2n - 2} W_{2n - 1} = \ldots = \frac{(2n - 1)(2n - 3) \ldots 1}{2n(2n - 2) \ldots 2} \cdot \frac{\pi}{2} = \frac{(2n - 1)!!}{2n!!} \cdot \frac{\pi}{2}$

Если нечетно, то $W_{2n + 1} = \frac{2n}{2n + 1}W_{2n - 1} = \frac{2n}{2n + 1} \cdot \frac{2n - 2}{2n - 1} W_{2n - 3} = \ldots = \frac{2n(2n - 2) \ldots 2}{(2n + 1)(2n - 1) \ldots 3} \cdot 1 = \frac{2n!!}{(2n + 1)!!}$

\begin{theo}{Формула Валлеса}
    $\lim\limits_{n \to \infty} \frac{2n!!}{(2n + 1)!!} \cdot \frac{1}{\sqrt{2n + 1}} = \sqrt{\frac{\pi}{2}}$
\end{theo}

\textit{Доказательство:}

$\sin^{2n + 2}x \leq \sin^{2n + 1}x \leq \sin^{2n}x$ при $x \in [0, \frac{\pi}{2}]$

$\int\limits_0^\frac{\pi}{2} \sin^{2n + 2}x dx \leq \int\limits_0^\frac{\pi}{2} \sin^{2n + 1}x dx \leq \int\limits_0^\frac{\pi}{2} \sin^{2n}x dx$

То есть $W_{2n + 2} \leq W_{2n + 1} \leq W_{2n}$

$\frac{2n + 1}{2n + 2} \cdot \frac{(2n - 1)!!}{(2n)!!} \cdot \frac{\pi}{2} \leq \frac{2n!!}{(2n + 1)!!} \leq \frac{(2n - 1)!!}{(2n)!!} \cdot \frac{\pi}{2}$

$\frac{2n + 1}{2n + 2} \cdot \frac{\pi}{2} \leq \frac{(2n)!!}{(2n + 1)!!} \cdot \frac{(2n)!!}{(2n - 1)!!} \leq \frac{\pi}{2} \Rightarrow \lim (\frac{(2n)!!}{(2n - 1)!!})^2 \cdot \frac{1}{2n + 1} = \frac{\pi}{2}$

\begin{theo}{Следствие}
    $C_{2n}^n \sim \frac{4^n}{\sqrt{\pi n}}$
\end{theo}

\textit{Доказательство:}

$C_{2n}^n = \frac{(2n)!}{n!n!} = \frac{(2n)!!(2n-1)!!}{n!n!} = \frac{(2n - 1)!!}{(2n)!!} 4^n \sim \sqrt{\frac{2}{\pi}} \cdot \frac{1}{\sqrt{2n + 1}} \cdot 4^n \sim \frac{4^n}{\sqrt{\pi n}}$

\begin{theo}{Формула Тейлора с остатком в интегральной форме}
    $f \in C^{n+1}\q{a, b},\ x_0, x \in \q{a, b}$

    Тогда $f(x) = \sum\limits_{k = 0}^n \frac{f^{(k)}(x_0)}{k!}(x - x_0)^k + \frac{1}{n!} \int\limits_{x_0}^x (x - t)^n f^{(n+1)}(t)dt$
\end{theo}

\textit{Доказательство:}

Индукция по $n$. База $n = 0$

$f(x) = f(x_0) + \frac{1}{0!} \int\limits_{x_0}^x f'(t)dt \stackrel{\text{Н-Л}}{=} f(x_0) + f(x) - f(x_0) = f(x)$ -- верно

Переход $n \to n + 1$

$f(x) = \sum\limits_{k = 0}^n \frac{f^{(k)}(x_0)}{k!}(x - x_0)^k + \frac{1}{n!} \int\limits_{x_0}^x (x - t)^n f^{(n+1)}(t)dt = (*)$

Берем $u = f^{(n + 1)}$, $v' = (x - t)^n$, $v = -\frac{(x - 1)^{n + 1}}{n + 1}$

$\int\limits_{x_0}^x (x-t)^n f^{(n + 1)}(t)dt = -f^{(n + 1)}(t) \frac{(x - t)^{n + 1}}{n + 1}|_{t = x_0}^{t = x} + \int\limits_{x_0}^x \frac{(x - t)^{n + 1}}{n + 1} f^{(n + 2)}(t)dt$

$(*) = \sum\limits_{k = 0}^n \frac{f^{(k)}(x_0)}{k!}(x - x_0)^k + \frac{f^{(n + 1)}(x_0)}{(n + 1)!}(x - x_0)^{n + 1} + \frac{1}{(n + 1)!} \int\limits_{x_0}^x (x - t)^{n + 1} f^{(n + 2)}(t)dt$

\begin{Example}{}
    $H_j := \frac{1}{j!} \int\limits_0^{\frac{\pi}{2}} ((\frac{\pi}{2})^2 - x^2)^j \cos{x}dx$

    \begin{theo}{Свойства:}
        \begin{enumerate}
            \item $0 < H_j \leq \frac{1}{j!} \int\limits_0^\frac{\pi}{2} (\frac{\pi}{2})^{2j} \cos{x}dx = \frac{(\frac{\pi}{2})^{2j}}{j!}$
            \item Если $c > 0$, то $c^jH_j \xrightarrow[j \to \infty]{} 0$
            \item $H_0 = 1,\ H_1 = 2$
            \item При $j \geq 2\ \ H_j = (4j - 2)H_{j - 1} - \pi^2H_{j - 2}$
            
            \textit{Доказательство:}

            Берем $v' = \cos{x}$, $u = ((\frac{\pi}{2})^2 - x^2)^j$, $v = \sin{x}$, $u' = -2jx((\frac{\pi}{2})^2 - x^2)^{j - 1}$

            $j!H_j = \int\limits_0^\frac{\pi}{2} ((\frac{\pi}{2})^2 - x^2)^j \cos{x}dx = ((\frac{\pi}{2})^2 - x^2)^j \sin{x}|{x = 0}^{x = \frac{\pi}{2}} + 2j \int\limits_0^\frac{\pi}{2} x((\frac{\pi}{2})^2 - x^2)^{j - 1} \sin{x}dx = (*)$

            Первое слагаемом занулится, второе еще раз интегрируем по частям

            $v' = \sin{x} \Rightarrow v = -\cos{x}$
            
            $u = x((\frac{\pi}{2})^2 - x^2)^{j - 1} \Rightarrow u' = ((\frac{\pi}{2})^2 - x^2)^{j - 1} - 2(j - 1)x^2((\frac{\pi}{2})^2 - x^2)^{j - 2} = (2j - 1)((\frac{\pi}{2})^2 - x^2)^{j - 1} - \frac{\pi^2}{2}(j - 1)((\frac{\pi}{2})^2 - x^2)^{j - 2}$

            $(*) = 2j(-\cos{x}x((\frac{\pi}{2})^2 - x^2)^{j - 1})|_{x = 0}^{x = \frac{\pi}{2}} + (2j - 1) \int\limits_{0}^\frac{\pi}{2} ((\frac{\pi}{2})^2 0 x^2)^{j - 1}\cos{x}dx - \frac{\pi^2}{2}(j - 1) \int\limits_0^\frac{\pi}{2} ((\frac{\pi}{2})^2 - x^2)^{j - 2}\cos{x}dx$

            Первое слагаемое зануляется, второе $= (j - 1)!H_{j - 1}$, третье $= (j - 2)!H_{j - 2}$

            $j!H_j = 2(2j - 1)j(j - 1)!H_{j - 1} - \pi^2j(j - 1)(j - 2)!H_{j - 2}$

            $H_j = (4j - 2)H_{j - 1} - \pi^2H_{j - 2}$

            \item Существует многочлен $P_j$ степени $\leq j$ с целыми коэффициентами, такой что $H_j = P_j(\pi^2)$
            
            \textit{Доказательство:}

            $P_0 \equiv 1,\ P_1 \equiv 2$

            $P_j(x) = (4j - 2)P_{j - 1}(x) - xP_{j - 2}(x) \Rightarrow P_j(\pi^2) = (4j - 2)P_{j - 1}(\pi^2) - \pi^2P_{j - 2}(\pi^2) = H_j$
        \end{enumerate}
    \end{theo}
\end{Example}

\begin{theo}{Теорема Ламберта}
    Числа $\pi$ и $\pi^2$ иррациональны
\end{theo}

\textit{Доказательство:}

Пусть $\pi^2 = \frac{m}{n} \Rightarrow 0 < H_j = P_j(\frac{m}{n}) = \frac{\text{целое число}}{n^j} \Rightarrow n^jP_j(\frac{m}{n}) = n^jH_j > 0$ и является целым числом $\Rightarrow n^jH_j \geq 1$, но $\lim\limits_{j \to \infty} n^jH_j = 0$ по свойству 2 -- противоречие

\begin{defin}{Равномерная непрерывность}
    $f : E \rightarrow \R$, $E \subset \R$

    $f$ равномерно непрерывна на $E$, если $\forall \varepsilon > 0\ \exists \delta > 0\ \forall x, y \in E : |x - y| < \delta \Rightarrow |f(x) - f(y)| < \varepsilon$
\end{defin}

\begin{Remark}{}
    Определение непрерывности во всех точках множества $E$

    $\forall y \in E\ \forall \varepsilon > 0\ \exists \delta > 0\ x \in E : |x - y| < \delta \Rightarrow |f(x) - f(y)| < \varepsilon$

    То есть в этом определении $\delta(\varepsilon, y)$, а в равномерной непрерывности $\delta(\varepsilon)$
\end{Remark}

\begin{Example}{}
    \begin{enumerate}
        \item $\sin$ и $\cos$ равномерно непрерывны на $\R$
        
        $|sin{x} - \sin{y}| \leq |x - y|$ и $|\cos{x} - \cos{y}| \leq |x - y| \Rightarrow \delta = \varepsilon$ подходит

        \item $x^2$ не является равномерно непрерывной на $\R$
        
        Возьмем $\varepsilon = 1$ и покажем, что никакая $\delta > 0$ не подходит

        Рассмотрим $x$ и $x + \frac{\delta}{2}$

        $f(x + \frac{\delta}{2}) - f(x) = (x + \frac{\delta}{2})^2 - x^2 = x\delta + \frac{\delta^2}{4} > x\delta \geq 1$ при $x \geq \frac{1}{\delta}$
    \end{enumerate}
\end{Example}

\begin{theo}{Теорема Кантора}
    $f \in C[a, b] \Rightarrow f$ равномерно непрерывна на $[a, b]$
\end{theo}

\textit{Доказательство:}

Возьмем $\varepsilon > 0$ и предположим, что никакое $\delta > 0$ не подходит

$\delta = 1$ не подходит $\Rightarrow$ найдутся $x_1, y_1 \in [a, b] : |x_1 - y_1| < 1$ и $|f(x_1) - f(y_1)| \geq \varepsilon$

$\delta = \frac{1}{2}$ не подходит $\Rightarrow$ найдутся $x_2, y_2 : |x_2 - y_2| < \frac{1}{2}$ и $|f(x_2) - f(y_2)| \geq \varepsilon$

\ldots

$\delta = \frac{1}{n}$ не подходит $\Rightarrow$ найдутся $x_n, y_n : |x_n - y_n| < \frac{1}{n}$ и $|f(x_n) - f(y_n)| \geq \varepsilon$

Выберем из $x_n$ схожящуюся подпоследовательность $x_{n_k} : \lim x_{n_k} = c$

$a \leq x_{n_k} \leq b \Rightarrow c \in [a, b]$ и $\lim y_{n_k} = \lim x_{n_k} + \lim (y_{n_k} - x_{n_k}) = c + 0 = c$

Функция $f$ непрерывна в $c \Rightarrow \exists \delta > 0\ \forall |x - c| < \delta \Rightarrow |f(x) - f(c)| < \frac{\varepsilon}{2}$

$\lim x_{n_k} = c \Rightarrow$ при больших $k$ $|x_{n_k} - c| < \delta \Rightarrow |f(x_{n_k}) - f(c)| < \frac{\varepsilon}{2}$

$\lim y_{n_k} = c \Rightarrow$ при больших $k$ $|y_{n_k} - c| < \delta \Rightarrow |f(y_{n_k}) - f(c)| < \frac{\varepsilon}{2}$

Тогда $\varepsilon \leq |f(x_{n_k}) - f(y_{n_k})| \leq |f(x_{n_k}) - f(c)| + |f(c) - f(y_{n_k})| < \varepsilon$ -- противоречие

\begin{Remark}{}
    Важно, что именно отрезок

    Для $x^2$ мы поняли, что на $[0, + \infty)$ нет равномерной непрерывности $\Rightarrow$ отрезок нельзя заменить на луч

    Поймем что на полуинтервал тоже нельзя

    $f(x) = \frac{1}{x}$ на $(0, 1]$ не равномерно непрерывна

    $\varepsilon = 1$ никакое $\delta > 0$ не подходит (если какое-то не подходит, то $\delta > \delta_0$ тоже не подходит)

    Возьмем $0 < \delta \leq 1$, $x = \frac{\delta}{2}$ и $y = \frac{\delta}{4}$

    $|x - y| = \frac{\delta}{4} < \delta$, но $|f(x) - f(y)| = \frac{2}{\delta} > 1$
\end{Remark}

\begin{defin}{Модуль непрерывности}
    $f : E \to \R$, $E \subset \R$

    $\omega_f(\delta) := \sup\{ |f(x) - f(y)| : x, y \in E,\ |x - y| < \delta \}$ определена при $\delta \geq 0$

    \begin{theo}{Свойства:}
        \begin{enumerate}
            \item $\omega_f(0) = 0$
            \item $\omega_f(\delta) \geq 0$
            \item $\omega_f$ нестрого возрастает
            \item $|f(x) - f(y)| \leq \omega_f(|x - y|)$
            \item Если $f$ липшицева к константе $M$, то $\omega_f(\delta) \leq M\delta$
            
            \textit{Доказательство:}

            Липшицевость с константой $M$ -- это $|f(x) - f(y)| \leq M|x - y|\ \forall x, y \in E$

            \item $f$ равномерно непрерывна на $E \Leftrightarrow \lim\limits_{\delta \to 0_+} \omega_f(\delta) = 0$ (т.е. $w_f$ непрерывна в 0)
            
            \textit{Доказательство:}

            \begin{itemize}
                \item[$\Rightarrow$] $f$ равномерно непрер $\Rightarrow \forall \varepsilon\ \exists \delta > 0\ \forall x, y \in E : |x - y| < \delta \Rightarrow |f(x) - f(y)| \leq \varepsilon \Rightarrow \\
                \Rightarrow \omega_f(\frac{\delta}{2}) < \varepsilon$, т.к. $\omega_f(\frac{\delta}{2}) \leq \sup\{|f(x) - f(y)| : |x - y| < \delta\}$

                Значит $\forall t < \frac{\delta}{2}\ \omega_f(t) \leq \varepsilon \Rightarrow \lim\limits_{t \to 0_+} \omega_f(t) = 0$

                \item[$\Leftarrow$] $|f(x) - f(y)| \leq w_f(|x - y|)$ по $\varepsilon > 0$ выберем $\delta > 0$ такое что $\omega_f(\delta) < \varepsilon \Rightarrow$ если $|x - y| \leq \delta$, то $|f(x) - f(y)| \leq \omega_f(\delta) < \varepsilon$
            \end{itemize}

            \item $f \in C[a, b] \Leftrightarrow \lim\limits_{\delta \to 0_+} \omega_f(\delta) = 0$
            
            \textit{Доказательство:}

            $f \in C[a, b] \Leftrightarrow f$ равномерно непрерывна на $[a, b] \Leftrightarrow \lim\limits_{\delta \to 0_+} \omega_f(\delta) = 0$
        \end{enumerate}
    \end{theo}
\end{defin}

\begin{defin}{Дробление отрезка}
    Дробление отрезка $[a, b]$ -- набор точек $x_0 = a < x_1 < x_2 < \ldots < x_n = b$
\end{defin}

\begin{defin}{Ранг дробления}
    Ранг дробления -- длина самого большого отрезка из дробления

    $\max\{x_1 - x_0, x_2 - x_1 \ldots x_n - x_{n - 1}\} =: \tau$
\end{defin}

\begin{defin}{Оснащение дробления}
    Оснащение дробления -- набор точек $\xi_k : \xi_k \in [x_{k - 1}, x_k]$
\end{defin}

\begin{defin}{Интегральная сумма (сумма Римана)}
    $f : [a, b] \to \R$, $\tau$ дробление отрезка и $\tau = \{ x_0, x_1 \ldots x_n \}$

    $\xi$ -- оснащение дробления и $\xi = \{ \xi_1, \xi_2 \ldots \xi_n \}$

    $S(f, \tau, \xi) := \sum\limits_{k = 1}^n f(\xi_k)(x_k - x_{k - 1})$
\end{defin}

\begin{Example}{}
    $S_p(n) := 1^p + 2^p + 3^p + \ldots + n^p;\ p > 0$

    $\lim \frac{S_p(n)}{n^{p + 1}}$?

    Возьмем $f(x) = x^p$ на $[0, 1]$ и $x_k = \xi_k = \frac{k}{n}$

    $\frac{S_p(n)}{n^{p + 1}} = \frac{1}{n} \cdot ((\frac{1}{n})^p + (\frac{2}{n})^p + \ldots + (\frac{n}{n})^p) = \sum f(\xi_k) \cdot (x_k - x_{k - 1}) \rightarrow \int\limits_0^1 f(t)dt = \int\limits_0^1 x^pdx = \frac{x^{p+1}}{p + 1}|_0^1 = \frac{1}{p + 1}$
\end{Example}

\begin{theo}{Теорема об интегральных суммах}
    $f \in C[a, b]$, $\tau$ -- дробление

    Тогда $|\int\limits_a^b f - S(f, \tau, \xi)| \leq (b - a)\omega_f(|\tau|)$
\end{theo}

\textit{Доказательство:}

$\Delta := \int\limits_a^b f - \sum\limits_{k = 1}^n f(\xi_k)(x_k - x_{k - 1}) = \sum\limits_{k = 1}^n \int\limits_{x_{k - 1}}^{x_k} f - \sum\limits_{k = 1}^n f(\xi_k)(x_k - x_{k - 1}) = \sum\limits_{k = 1}^n \int\limits_{x_{k - 1}}^{x_k} (f(x) - f(\xi_k))dx$

$|\Delta| \leq \sum\limits_{k = 1}^n \int\limits_{x_{k - 1}}^{x_k} |f(x) - f(\xi_k)|dx \leq \sum\limits_{k = 1}^n \int\limits_{x_{k - 1}}^{x_k} \omega_f(|\tau|)dx = \omega_f(|\tau|) \sum\limits_{k = 1}^n (x_k - x_{k - 1}) = \omega_f(|\tau|)(b - a)$

\begin{theo}{Следствия}
    \begin{enumerate}
        \item $f \in C[a, b]$. Тогда $\forall \varepsilon > 0\ \exists \delta > 0 : \forall$ дробления $\tau$ ранга $< \delta$ и $\forall$ его оснащения $\xi$
        
        $|\int\limits_a^b f - S(f, \tau, \xi)| < \varepsilon$

        \item $f \in C[a, b]$. Тогда для любой последовательности дроблений $\tau_n : |\tau_n| \to 0$ и любой последовательности оснащений $\xi_n$ $\lim S(f, \tau_n, \xi_n) = \int\limits_a^b f$
    \end{enumerate}
\end{theo}

\begin{defin}{Интеграл Римана}
    $f : [a, b] \to \R$

    $f$ интегрируема по Риману на $[a, b]$, и $I$ ее интеграл, если $\forall \varepsilon > 0\ \exists \delta > 0 : \forall$ дробления $\tau$ ранга $< \delta$ и $\forall$ его оснащения $\xi$

    $|I - S(f, \tau, \xi)| < \varepsilon$
\end{defin}

\begin{lem}{}
    $f \in C^2[\alpha, \beta]$. Тогда $\int\limits_\alpha^\beta f(x)ds - (\beta - \alpha)\frac{f(\alpha) + f(\beta)}{2}$ = $- \frac{1}{2} \int\limits_{\alpha}^{\beta} f''(t)(t - \alpha)(b - t)dt$
\end{lem}

\textit{Доказательство:}

$\gamma = \frac{\alpha + \beta}{2}$

$\int\limits_\alpha^\beta f(x)dx = \int\limits_\alpha^\beta f(x)(x - \gamma)'dx = f(x)(x - \gamma)|_{x = \alpha}^{x = \beta} - \int\limits_\alpha^\beta f'(x)(x - \gamma)dx = f(\beta) \cdot \frac{\beta - \alpha}{2} - f(\alpha) \cdot \frac{\alpha - \beta}{2} = \\ = \frac{f(\beta) + f(\alpha)}{2} (\beta - \alpha)$

$\Delta = - \int\limits_\alpha^\beta f'(x)(x - \gamma)dx = (*)$

Берем $u = f';\ v' = x - \gamma;\ -v = \frac{1}{2}(x - \alpha)(\beta - x) = \frac{1}{2}(-x^2 + (\alpha + \beta)x - \alpha\beta)$

$(*) = -f'(x) \cdot (- \frac{1}{2})(x - \alpha)(\beta - x)|_{x = \alpha}^{x = \beta} - \int\limits_\alpha^\beta f''(x) \frac{1}{2} (x - \alpha)(\beta - x)dx$

\begin{theo}{Формула трапеций}
    $S = \sum (x_k - x_{k - 1})\frac{f(x_{k - 1}) + f(x_k)}{2}$

    Как выглядит формула, если узлы на равных расстояниях?

    $x_k - x_{k - 1} = \frac{b - a}{n}$

    $S = \frac{b - a}{n} \cdot \sum\limits_{k = 1}^n \frac{f(x_{k - 1}) + f(x_k)}{2} = \frac{b - a}{n} (\frac{f(a)}{2} + \sum\limits_{k = 1}^{n - 1} f(x_k) + \frac{f(b)}{2})$
\end{theo}

\begin{nota}{Как выглядит интегральная сумма если узлы на равных расстояниях?}
    $\xi_k = x_k$

    $S(f, \tau, \xi) = \sum\limits_{k = 1}^n f(\xi_k)(x_k - x_{k - 1}) = \frac{b - a}{n}\sum\limits_{k = 1}^n f(x_k)$

    А если $\xi_k = x_{k - 1}$, то $S(f, \tau, \xi) = \frac{b - a}{n} \sum\limits_{k = 0}^{n - 1}f(x_k)$

\begin{Remark}{}
    Если $|f| \leq M$, то $\omega_f(\delta) \leq M\delta$

    $|S(f, \xi, \tau) - \int\limits_a^b f| \leq (b - a)\omega_f(\frac{b - a}{n}) \leq (b - a)^2 \cdot \frac{M}{n}$
\end{Remark}

\end{nota}

\begin{theo}{Оценка погрешности в формуле трапеций}
    $f \in C^2[a, b]$

    Тогда $|\int\limits_a^b f - \sum\limits_{k = 1}^n(x_k - x_{k - 1}) \cdot \frac{f(x_{k - 1}) + f(x_k)}{2}| \leq \frac{|\tau|^2}{8} \cdot \int\limits_a^b |f''|$
\end{theo}

\newpage

\textit{Доказательство:}

$\Delta = \sum\limits_{k = 1}^n \int\limits_{x_{k - 1}}^{x_k} f - \sum\limits_{k = 1}^n (x_k - x_{k - 1}) \frac{f(x_{k - 1}) + f(x_k)}{2} = \sum\limits_{k = 1}^n (\int\limits_{x_{k - 1}}^{x_k} f - (x_k - x_{k - 1}) \frac{f(x_{k - 1}) + f(x_k)}{2}) \stackrel{lemma}{=} \\ \stackrel{lemma}{=} \sum\limits_{k = 1}^n - \frac{1}{2}\int_{x_{k - 1}}^{x_k} f''(t)(x_k - t)(t - x_{k - 1})dt$

$|\Delta| \leq \frac{1}{2} \sum\limits_{k = 1}^n \int\limits_{x_{k - 1}}^{x_k} |f''(t)| \cdot |(x_k - t)(t - x_{k - 1})dt| \leq (*)$

$(x_k - t)(t - x_{k - 1}) \leq (\frac{(x_k - t) + (t - x_{k - 1})}{2})^2 = (\frac{x_k - x_{k - 1}}{2})^2 = \frac{|\tau|^2}{4}$

$(*) \leq \frac{|\tau|^2}{8} \sum \int\limits_{x_{k - 1}}^{x_k} |f''(t)|dt = \frac{|\tau|^2}{8} \int\limits_a^b |f''|$

\begin{Remark}{}
    Если $|f''| \leq M$ и узлы равноотстоящие друг от друга, то $|\int\limits_a^b f - \frac{b - a}{n} (\frac{f(a)}{2} + \sum f(x_k) + \frac{f(b)}{2})| \leq (\frac{b - a}{n})^2 \frac{M(b - a)}{8}$
\end{Remark}

\begin{theo}{Формула Эйлера-Маклорена для второй производной}
    $f \in C^2[a, b];\ m, n \in \Z$

    $\sum\limits_{k = m}^n f(k) = \frac{f(m) + f(n)}{2} + \int\limits_m^n f(t)dt + \frac{1}{2}\int\limits_m^n f''(t)\{t\}(1 - \{t\})dt$
\end{theo}

\textit{Доказательство:}

Пишем лемму (1.1) для $\alpha = k$ и $\beta = k + 1$

$\int\limits_k^{k + 1} f = \frac{f(k) + f(k + 1)}{2} - \frac{1}{2} \int\limits_k^{k + 1} f''(t)(t - k)(k + 1 - t)dt$

Просуммируем по $k$ от $m$ до $n - 1$

$\int\limits_m^n f = \sum\limits_{k = m}^{n - 1} \int\limits_{k}^{k + 1} f = \sum\limits_{k = m}^{n - 1} \frac{f(k) + f(k + 1)}{2} - \frac{1}{2} \sum\limits_{k = m}^{n - 1} \int\limits_k^{k + 1} f''(t)\{t\}(1 - \{t\})dt = \frac{f(m)}{2} + \sum f(k) + \frac{f(n)}{2} - \\
- \frac{1}{2} \int\limits_m^n f''(t)\{t\}(1 - \{t\})dt$

$\int\limits_m^n f + \frac{f(m) + f(n)}{2} = \sum f(k) - \frac{1}{2} \int\limits_m^n f''(t) \{t\}(1 - \{t\})dt$

\begin{Example}{}
    \begin{enumerate}
        \item $S_p(n) = 1^p + 2^p + \ldots + n^p;\ f(x) = x^p;\ f''(x) = p(p - 1)x^{p - 2}$
        
        $S_p(n) = \frac{1 + n^p}{2} + \int\limits_1^n x^pdx + \frac{1}{2} \int\limits_1^n p(p - 1)x^{p - 2} \{x\}(1 - \{x\})dx$

        Пусть $p > -1$

        $|S_p(n) - \frac{n^{p + 1} - 1}{p + 1} - \frac{n^p + 1}{2}| \leq \frac{|p||p - 1|}{8} \int\limits_1^n x^{p - 2}dx$

        Если $p \in (-1, 1)$, то $\int\limits_1^n x^{p - 2}dx = \frac{x^{p - 1}}{p - 1}|_1^n \leq \frac{1}{1 - p} \Rightarrow S_p(n) = \frac{n^{p + 1}}{p + 1} + \frac{n^p}{2} + O(1)$

        Если $p > 1$, то $\int\limits_1^n x^{p - 2}dx = \frac{x^{p - 1}}{p - 1}|_1^n \leq \frac{n^{p - 1}}{p - 1} = O(n^{p - 1}) \Rightarrow S_p(n) = \frac{n^{p + 1}}{p + 1} + \frac{n^p}{2} + O(n^{p - 1})$

        \item Гармонические числа $H_n := 1 + \frac{1}{2} + \ldots + \frac{1}{n}$
        
        $f(x) = \frac{1}{x};\ f''(x) = \frac{2}{x^3};\ m = 1$

        $H_n = \frac{1 + \frac{1}{n}}{2} + \int\limits_1^n \frac{dx}{x} + \frac{1}{2}\int\limits_1^n \frac{2}{x^3}\{x\}(1 - \{x\})dx = \ln{n} + \frac{1}{2} + \frac{1}{2n} + a_n = (*)$

        $a_n := \int\limits_1^n  \frac{\{x\}(1 - \{x\})}{x^3}dx$

        $a_{n + 1} = a_n + \int\limits_n^{n + 1} \frac{\{x\}(1 - \{x\})}{x^3}dx > a_n$

        Поймем что $a_n$ ограничена: $a_n \leq \int\limits_1^n \frac{\frac{1}{4}}{x^3}dx = \frac{1}{4} \cdot (- \frac{1}{2x^2})|_1^n = \frac{1}{8} - \frac{1}{8n^2} < \frac{1}{8} \Rightarrow \\
        \Rightarrow$ существует конечный $\lim a_n =: a$

        $(*) = \ln{n} + \frac{1}{2} + \frac{1}{2n} + a + O(1) = \ln{n} + \gamma + O(\frac{1}{n})$, где $\gamma$ -- постоянная Эйлера

        \begin{nota}{}
            $\gamma \approx 0.5772156043\ldots$
        \end{nota}

        \item Формула Стирлинга
        
        $f(t) = \ln{t};\ f''(t) = -\frac{1}{t^2};\ m = 1$

        $\sum\limits_{k = 1}^n \ln{k} = \frac{\ln{1} + \ln{n}}{2} + \int\limits_1^n \ln{t}dt - \frac{1}{2} \int\limits_1^n \frac{\{t\}(1 - \{t\})}{t^2}dt = n\ln{n} - n + \frac{1}{2}\ln{n} + 1 - b_n$

        $b_{n + 1} = b_n + \frac{1}{2}\int\limits_n^{n + 1} \frac{\{t\}(1 - \{t\})}{t^2}dt > b_n$

        $b_n \leq \frac{1}{8}\int\limits_1^n \frac{dt}{t^2} = \frac{1}{8} (- \frac{1}{t})|_1^n = \frac{1}{8} - \frac{1}{8n} < \frac{1}{8} \Rightarrow \lim b_n = b \in \R \Rightarrow b_n = b + o(1)$

        $\ln{n!} = n\ln{n} - n + \frac{1}{2}\ln{n} + (1 - b) + o(1)$

        $n! = n^ne^{-n}\sqrt{n}e^{1 - b}e^{o(1)} \sim C n^n e^{-n}\sqrt{n}$ (где $C = e^{1 - b}$)

        Хотим найти $C > 0$ из формулы $n! \sim n^ne^{-n}\sqrt{n}C$

        Знаем, что $C_{2n}^n \sim \frac{4^n}{\sqrt{\pi n}}$, но $C_{2n}^n = \frac{(2n)!}{(n!)^2} \sim \frac{C(2n)^{2n}e^{-2n}\sqrt{2n}}{(Cn^ne^{-n}\sqrt{n})^2} = \frac{C 2^{2n}\sqrt{2n}}{C^2 \sqrt{n} \sqrt{n}} = \frac{4^n\sqrt{n}}{C\sqrt{n}} \sim \frac{4^n}{\sqrt{\pi n}}$

        $\Rightarrow \lim\limits_{n \to \infty} \frac{4^n\sqrt{2}}{C\sqrt{n}}\cdot \frac{\sqrt{\pi n}}{4^n} = 1 \Rightarrow C = \sqrt{2\pi}$

        \begin{nota}{Формула Стирлинга}
            $n! \sim n^n e^{-n}\sqrt{2\pi n}$

            $\ln{n!} = n\ln{n} - n + \frac{1}{2}(\ln n + \ln{2\pi}) + o(1)$
        \end{nota}

        \begin{Remark}{}
            Чуть более точные вычисления дают $O(\frac{1}{n})$ вместо $o(1)$

            $n! = n^ne^{-n} \sqrt{2\pi n} e^{O(\frac{1}{n})}$
        \end{Remark}
    \end{enumerate}
\end{Example}

\subsection{\S 5. Несобственные интегралы}

\begin{defin}{}
    $- \infty < a < b \leq + \infty;\ f \in C[a, b)$

    $\int\limits_a^{\to b} f := \lim\limits_{B \to b_-} \int\limits_a^B f$, если такой предел существует

    $- \infty \leq a < b < + \infty;\ f \in C(a, b]$

    $\int\limits_{\to a}^b f := \lim\limits_{A \to a_+} \int\limits_A^b f$, если такой предел существует
\end{defin}

\begin{Remark}{}
    \begin{enumerate}
        \item Если $F$ -- первообразная $f$ на $[a, b)$, то 
        
        $\int\limits_a^{\to b} f = \lim\limits_{B \to b_-} \int\limits_a^B f = \lim\limits_{B \to b_-} (F(B) - F(a)) = \lim\limits_{B \to b_-} F(B) - F(a)$

        Если $F$ -- первообразная $f$ на $(a, b]$, то

        $\int\limits_{\to a}^b f = F(b) - \lim\limits_{A \to a_+} F(A) =: F|_a^b$, т.е. подстановку теперь понимаем как предел (в случае, если она не определена в какой-то точке)

        \item Если $f \in C[a, b]$, то новое определение совпадает со старым
        
        $\int\limits_a^{\to b} f - \int\limits_a^b f = \lim\limits_{B \to b_-} \int\limits_a^B f - \int\limits_a^b f = -\lim\limits_{B \to b_-} \int\limits_B^b f$

        $f \in C[a, b] \Rightarrow f$ ограничена $\Rightarrow |f| \leq M$

        $|\int\limits_B^b f| \leq \int\limits_B^b |f| \leq \int\limits_B^b M = (b - B)M \xrightarrow[B \to b_-]{} 0$
    \end{enumerate}
\end{Remark}

\begin{Example}{}
    $\int\limits \frac{dx}{x^p}$ на $(1, \infty)$ и $(0, 1)$
    \begin{enumerate}
        \item $p \neq 1$, первообразная для $\frac{1}{x^p} = x^{-p}$ -- это $\frac{x^{-p + 1}}{1 - p}$

        $\int\limits_1^{+ \infty} \frac{dx}{x^p} = \frac{x^{1 - p}}{1 - p}|_1^\infty = \lim\limits_{x \to \infty} \frac{x^{1 - p}}{1 - p} - \frac{1}{1 - p} = \begin{cases}
            + \infty & p < 1 \\
            \frac{1}{p - 1} & p > 1
        \end{cases}$

        Если $p = 1$, то первообразная для $\frac{1}{x}$ -- это $\ln{x}$

        $\int\limits_1^{+ \infty} \frac{dx}{x} = \ln{x}|_1^{+ \infty} = \lim\limits_{x \to \infty}\ln{x} - \ln{1} = + \infty$

        Итого $\int\limits_1^{+ \infty} \frac{dx}{x^p} = \begin{cases}
            + \infty & p \leq 1 \\
            \frac{1}{p - 1} & p > 1
        \end{cases}$

        \item $p \neq 1$, первообразная $\frac{x^{1 - p}}{1 - p} \Rightarrow \int\limits_0^1 \frac{dx}{x^p} = \frac{x^{1 - p}}{1 - p}|_0^1 = \frac{1}{1 - p} - \lim\limits_{x \to 0_+} \frac{x^{1 - p}}{1 - p} = \begin{cases}
            \frac{1}{1 - p} & p < 1 \\
            + \infty & p > 1
        \end{cases}$

        $p = 1 \Rightarrow$ первообразная $\ln{x} \Rightarrow \int\limits_0^1 \frac{dx}{x} = \ln{x}|_0^1 = \ln{1} - \lim\limits_{x \to 0_+} \ln{x} = + \infty$

        Итого $\int\limits_0^1 \frac{dx}{x^p} = \begin{cases}
            \frac{1}{1 - p} & p < 1 \\
            + \infty & p \geq 1
        \end{cases}$
    \end{enumerate}
\end{Example}

\begin{defin}{Сходящийся интеграл}
    Несобственный интеграл $\int\limits_a^b f$ называется сходящимся, если $\lim$ из определения существует и конечен и называется расходящимся в противном случае
\end{defin}

\begin{Remark}{}
    $\int\limits_1^{+ \infty} \frac{dx}{x^p}$ сходится $\Leftrightarrow p > 1$

    $\int\limits_0^1 \frac{dx}{x^p}$ сходится $\Leftrightarrow p < 1$
\end{Remark}

\begin{theo}{Критерий Коши для сходимости интегралов}
    $f \in C[a, b];\ \ - \infty < a < b \leq + \infty$

    Тогда $\int\limits_a^{\to b} f$ сходится $\Leftrightarrow \forall \varepsilon > 0\ \exists c \in (a, b) : \forall A, B \in (c, b) \Rightarrow |\int\limits_A^B f| < \varepsilon$
\end{theo}

\textit{Доказательство:}

\begin{itemize}
    \item[$\Rightarrow$] $\int\limits_a^{\to b} f$ сходится $\Rightarrow$ существует конечный $\lim\limits_{y \to b_-} F(y)$, где $F$ -- первообразная $f$
    
    $\forall \varepsilon > 0$ найдется такая окрестность $(c, b)$, что $\forall y \in (c, b)\ |F(y) - L| < \frac{\varepsilon}{2}$

    $\Rightarrow$ если $A, B \in (c, b)$, то $|F(A) - L| < \frac{\varepsilon}{2}$ и $|F(B) - L| < \frac{\varepsilon}{2}$

    $\Rightarrow |F(B) - F(A)| \leq |F(B) - L| + |L - F(A)| < \frac{\varepsilon}{2} + \frac{\varepsilon}{2} = \varepsilon$

    \item[$\Leftarrow$] Пусть $F$ -- первообразная $f$, тогда $\forall \varepsilon > 0\ \exists c \in (a, b) : \forall A, B \in (c, b) \Rightarrow |F(B) - F(A)| < \varepsilon$
    
    Но это условие из критерия Коши для $\lim\limits_{y \to b_-} F(y) \Rightarrow$ этот предел существует и конечен $\Rightarrow \int\limits_a^{\to b} f = \lim\limits_{y \to b_-} F(y) - F(a)$ существует и конечен
\end{itemize}

\begin{theo}{Свойства несобственных интегралов}
    \begin{enumerate}
        \item Аддитивность. $f \in C[a, b);\ c \in (a, b)$
        
        Тогда $\int\limits_a^b f$ сходится $\Leftrightarrow \int\limits_c^b f$ сходится и в этом случае $\int\limits_a^b f = \int\limits_a^c f + \int\limits_c^b$

        \item $f \in C[a, b)$. Если $\int\limits_a^b$ сходится, то $\lim\limits_{B \to b_-} \int\limits_B^b f = 0$
        
        \item Линейность $f, g \in C[a, b);\ \alpha, \beta \in \R$
        
        Если $\int\limits_a^b f$ и $\int\limits_a^b$ сходится, то $\int\limits_a^b(\alpha f + \beta g)$ сходится и $\int\limits_a^b (\alpha f + \beta g) = \alpha \int\limits_a^b f + \beta \int\limits_a^b g$

        \item Монотонность. $f, g \in C[a, b);\ f \leq g$ на $[a, b)$ и интегралы существуют $\Rightarrow \int\limits_a^b f \leq \int\limits_a^b g$
        
        \item Интегрирование по частям. $f, g \in C^1[a, b)$. Тогда $\int\limits_a^b fg' = fg|_a^b - \int\limits_a^b f'g$
        
        (если существуют какие-то два предела из трех, то существует и третий и есть равенство)
    \end{enumerate}
\end{theo}

\textit{Доказательство:}

\begin{enumerate}
    \item $\int\limits_a^B f = \int\limits_a^c f + \int\limits_c^B$ и переходим к $\lim\limits_{B \to b_-} \Rightarrow \int\limits_a^b f = \int\limits_a^c f + \int\limits_c^b f$
    
    \item $\int\limits_a^b f = \int\limits_a^B f + \int\limits_B^b f \Rightarrow \int\limits_B^b f = \int\limits_a^b f - \int\limits_a^B$ и пишем $\lim\limits_{B \to b_-}$
    
    \item $a < B < b \Rightarrow \int\limits_a^B (\alpha f + \beta g) = \alpha \int\limits_a^B f + \beta \int\limits_a^B g$ и переходим к $\lim\limits_{B \to b_-}$
    
    \item $a < B < b$. Тогда $\int\limits_a^B f \leq \int\limits_a^B g$ и переходим к $\lim\limits_{B \to b_-}$
    
    \item $a < B < b$. Тогда $\int\limits_a^B fg' = fg|_a^B - \int\limits_a^B f'g$ и переходим к $\lim\limits_{B \to b_-}$
\end{enumerate}

\begin{Remark}{}
    Если $\int\limits_a^b f$ сходится и $\int\limits_a^b g$ расходится, то $\int\limits_a^b (f + g)$ расходится

    Если бы $\int\limits_a^b (f + g)$ сходится, то $\int\limits_a^b g = \int\limits_a^b ((f + g) - f)$ сходится
\end{Remark}

\begin{theo}{Замена переменной в несобственном интеграле}
    $\varphi : [\alpha, \beta) \to [a, b);\ \varphi \in C^1[\alpha, \beta)$ и существует $\lim\limits_{\gamma \to \beta_-} \varphi(\gamma) =: c$

    $f \in C[a, b)$. Тогда $\int\limits_\alpha^\beta f(\varphi(t)) \varphi'(t)dt = \int\limits_{\varphi(\alpha)}^c f(x)dx$ (если существует один из интегралов, то существует другой и есть равенство)
\end{theo}

\textit{Доказательство:}

$F(y) := \int\limits_{\varphi(\alpha)}^y f(x)dx;\ y \in [a, b)$

$\Phi(\gamma) := \int\limits_\alpha^\gamma f(\varphi(t))\varphi'(t)dt;\ \gamma \in [\alpha, \beta)$

Тогда $\Phi(\gamma) = F(\varphi(\gamma))$ по формуле замены переменной в собственном интеграле

\begin{enumerate}
    \item Пусть существует $\lim\limits_{y \to c_-} F(y)$. Покажем, что сузествует $\lim\limits_{\gamma \to \beta_-} \Phi(\gamma)$
    
    Проверяем по Гейне. Возьмем $\gamma_n \nearrow \beta \Rightarrow \Phi(\gamma_n) = F(\varphi(\gamma_n))$

    $\gamma_n \nearrow \beta \Rightarrow \varphi(\gamma_n) \to c \Rightarrow \Phi(\gamma_n) = F(\varphi(\gamma_n)) \to \lim\limits_{y \to c_-} F(y) = \int\limits_{\varphi(\alpha)}^c f(x)dx$

    $\Rightarrow \lim\limits_{\gamma \to \beta_-} \Phi(\gamma) = \int\limits_{\varphi(\alpha)}^c f(x)dx$

    \item Пусть существует $\lim\limits_{\gamma \to \beta_-} \Phi(\gamma)$. Проверим, что тогда $\exists \lim\limits_{y \to c} F(y)$. Тогда по пункту 1 будет равенство. Если $c < b$, то очевидно существует (т.к. $F$ непрерывно при $y < b$)
    
    Считаем, что $c = b$. Проверим по Гейне, что $\exists \lim\limits_{y \to b_-} F(y)$. Возьмем $b_n \nearrow b$. Можно считать, что $b_n \in [\varphi(\alpha), b)$. Т.е. сколь угодно близко к $b$ найдутся значения $\varphi \Rightarrow$ найдется \\ $\varphi(\beta_n) > b_n$

    $\Rightarrow$ по БК $\exists \gamma_n \in (\alpha, \beta_n) : \varphi(\gamma_n) = b_n \Rightarrow \Phi(\gamma_n) = F(\varphi(\gamma_n)) = F(b_n)$

    Осталось проверить, что $\gamma_n$ имеют предел. Предположим, что $\lim \gamma_n \neq \beta \Rightarrow$ найдется подпоследовательность $\gamma_{n_k} \to \tilde{\beta} \neq \beta$

    $\gamma_n \in [\alpha, \beta) \Rightarrow \gamma_{n_k} \to \tilde{\beta} < \beta \Rightarrow \varphi$ непрерывна в $\tilde{\beta}$

    $b \leftarrow b_{n_k} = \varphi(\gamma_{n_k}) \to \varphi(\tilde{\beta}) < b$. Противоречие. Следовательно $\lim \gamma_n = \beta \Rightarrow \lim \Phi(\gamma_n) = \\ = \lim\limits_{\gamma \to \beta_-} \Phi(\gamma)$, т.е. он существует
\end{enumerate}

\begin{Remark}{}
    $\int\limits_a^b f(x)dx$ заменой $x = b - \frac{1}{t}$ и $\varphi(t) = b - \frac{1}{t}$ сводится к $\int\limits_{\frac{1}{b - a}}^{+ \infty} f(b - \frac{1}{t})\frac{dt}{t^2}$

    То есть точку, где нет непрерывности можно записать на $\infty$. $\varphi(\frac{1}{b - a}) = a$ и $\varphi(\infty) = b$
\end{Remark}

\begin{theo}{}
    $f \in C[a, b)$ и $f \geq 0$. Тогда $\int\limits_a^b f$ всегда определен. Он сходится $\Leftrightarrow F$ -- ограниченная функция на $[a, b)$
\end{theo}

\textit{Доказательство:}

$F(y) := \int\limits_a^y f$. Если $y < z$, то $F(z) = \int\limits_a^z f = \int\limits_a^y f + \int\limits_y^z f \geq F(y) \Rightarrow F$ нестрого возрастает $\Rightarrow \\
\Rightarrow \lim\limits_{y \to b_-} F$ существует и $\lim\limits_{y \to b_-} F = \sup\limits_{y \in [a, b)} F(y)$

Тогда конечность предела равносильна ограниченности функции $F$

\begin{theo}{Следствие 1 (признак сравнения)}
    $f, g \in C[a, b),\ f, g \geq 0$ и $f \leq g$. Тогда
     
    \begin{enumerate}
        \item Если $\int\limits_a^b g$ сходится, то $\int\limits_a^b f$ сходится
        \item Если $\int\limits_a^b f$ сходится, то $\int\limits_a^b g$ расходится
    \end{enumerate}
\end{theo}

\textit{Доказательство:}

$F(y) := \int\limits_a^y f, G(y) := \int\limits_a^y g;\ f \leq g \Rightarrow F(y) \leq G(y)$

\begin{enumerate}
    \item $\int\limits_a^b g$ cходится $\Leftrightarrow G$ -- ограничена $\Rightarrow F$ -- ограничена $\Leftrightarrow \int\limits_a^b f$ cходится
    \item Если бы $\int\limits_a^b g$ сходится, то $\int\limits_a^b f$ сходится. Противоречие
\end{enumerate}

\begin{Remark}{}
    \begin{enumerate}
        \item Достаточно наличия неравенства $f \leq g$ при аргументах близких к точке $b$
        \item Неравенство $f \leq g$ можно заменить на $f = O(g)$
        
        $f = O(g)$ означает, что $f \leq Cg$ для некоторого C и $\int\limits_a^b Cg = C\int\limits_a^b g$

        \item Если $f = O(\frac{1}{x^{1 + \varepsilon}})$ при $\varepsilon > 0$, то $\int\limits_a^{+ \infty} f$ сходится
        
        $g(x) = \frac{1}{x^{1 + \varepsilon}}$ и $\int\limits_1^{+ \infty} \frac{dx}{x^{1 + \varepsilon}}$ сходится 
    \end{enumerate}
\end{Remark}

\begin{theo}{Следствие}
    $f, g \in C[a, b);\ f, g \geq 0$ и $f \stackrel{x \to b}{\sim} g$. Тогда $\int\limits_a^b f$ и $\int\limits_a^b g$ ведут себя одинаково
\end{theo}

\textit{Доказательство:}

$f \sim g \Rightarrow f = \varphi g$, где $\lim\limits_{x \to b_-} \varphi(x) = 1 \Rightarrow \frac{1}{2} \leq \varphi(x) \leq 2$ при $x$ близких к $b$

$\frac{1}{2}g(x) \leq f(x) \leq 2g(x)$ при $x$ близких к $b$ $\Rightarrow f = O(g)$ и $g = O(f)$ в окрестности $b$

\begin{Remark}{}
    Если $f \geq 0$ и $\int\limits_a^{+ \infty} f$ сходится, то не обязательно $\lim\limits_{x \to + \infty} f(x) = 0$
\end{Remark}

\begin{defin}{Абсолютная сходимость}
    $f \in C[a, b)$

    $\int\limits_a^b f$ абсолютно сходится, если $\int\limits_a^b |f| < + \infty$
\end{defin}

\begin{theo}{}
    Если $\int\limits_a^b f$ абсолютно сходится, то $\int\limits_a^b f$ сходится
\end{theo}

\textit{Доказательство:}

$0 \leq f_\pm \leq |f|$. Признак сравнения: $\int\limits_a^b |f|$ сходится $\Rightarrow \int\limits_a^b f_\pm$ сходится $\Rightarrow \int\limits_a^b f = \int\limits_a^b f_+ - \int\limits_a^b f_-$ сходится

\begin{Remark}{}
    Бывают интегралы, которые сходятся, но не абсолютно
\end{Remark}

\begin{Exercise}{}
    Что делать, если несколько точек, где нет непрерывности? Пусть отрезок $[a, b]$ нарезан на куски, т.е. $[a, b] = [a, c_1] \cup [c_1, c_2] \cup \ldots \cup [c_n, b]$ (где в $c_i$ нет непрерывности). Добавим в каждый полученный отрезок по точке $d_i$

    Итого подряд идут точки типа $a, d_1, c_1, d_2, c_2 \ldots d_n, c_n, d_{n + 1}, b$

    $\int\limits_a^b f$ сходится означает, что $\int\limits_a^{d_1}, \int\limits_{d_1}^{c_1}, \int\limits_{c_1}^{d_2}, \ldots, \int\limits_{d_{n + 1}}^b$ сходятся
\end{Exercise}

\begin{theo}{Признак Дирихле}
    $f, g \in C[a, + \infty)$

    \begin{enumerate}
        \item $f$ имеет ограниченную первообразную на $[a, + \infty)$ (т.е. $F(x) := \int\limits_a^x f$ ограничена)
        \item $g$ монотонна 
        \item $\lim\limits_{x \to + \infty} g(x) = 0$
    \end{enumerate}

    Тогда $\int\limits_a^{+ \infty} f(x)g(x)dx$ сходится
\end{theo}

\textit{Доказательство (для случая, когда $g \in C^1[a, + \infty)$):}

Хотим доказать, что $H(x) := \int\limits_a^x fg$ имеет конечный предел при $x \to + \infty$

$H(x) = \int\limits_a^x F'g = Fg|_a^x - \int\limits_a^b Fg'$. Проверим, что существует конечный предел

$\lim\limits_{x \to + \infty} F(x)g(x) = 0$, т.к. $F$ ограничена и $g$ -- бесконечно малая

Осталось доказать, что $\lim\limits_{x \to + \infty} \int\limits_a^x Fg'$ существует и конечен, т.е. что $\int\limits_a^{+ \infty} Fg'$ сходится

Проверим, что он абсолютно сходится, т.е. $\int\limits_a^{+ \infty} |Fg'| < + \infty$

$F$ -- ограничена $\Rightarrow \exists M : |F(x)| \leq M\ \forall x \Rightarrow |Fg'| \leq M|g'|$. По признаку сравнения достаточно проверить, что $\int\limits_a^{+ \infty} |g'|$ сходится. Из монотонности $g$ следует, что $g'$ фиксированного знака, поэтому надо проверить, что $\int\limits_a^{+ \infty} g'$ сходится

$\int\limits_a^{+ \infty} g' = g|_a^{+ \infty} = \lim\limits_{x \to + \infty} g(x) - g(a) = -g(a) < + \infty$

\begin{theo}{Признак Абеля}
    $f, g \in C[a, + \infty)$

    \begin{enumerate}
        \item $\int\limits_a^{+ \infty} f$ сходится
        \item $g$ монотонна 
        \item $g$ ограничена
    \end{enumerate}

    Тогда $\int\limits_a^{+ \infty} f(x)g(x)dx$ сходится
\end{theo}

\textit{Доказательство:}

Монотонная ограниченная функция имеет конечный предел $b := \lim\limits_{x \to + \infty} g(x)$

$\tilde{g}(x) := g(x) - b$ -- монотонная и $\lim\limits_{x \to + \infty} \tilde{g}(x) = 0$

$F(x) := \int\limits_a^x f$. По условию $\lim\limits_{x \to + \infty} F(x) = \int\limits_a^{+ \infty} f$ существует и конечен $\Rightarrow F$ локально ограничена, т.е. при $x \geq K\ |F(x)| \leq M$

Но на отрезке $[a, K]$ функция $F$ непрерывна $\Rightarrow$ ограничена

$\Rightarrow f$ и $\tilde{g}$ удовлетворяют условиям признака Дирихле $\Rightarrow \int\limits_a^{+ \infty} f\tilde{g}$ сходится

$\int\limits_a^{+ \infty} = \int\limits_a^{+ \infty} f(\tilde{g} + b) = \int\limits_a^{+ \infty} f\tilde{g} + b\int\limits_a^{+ \infty} f$ сходится как сумма двух сходящихся

\begin{theo}{Следствие}
    $f, g \in C[a, + \infty), f$ -- периодична с периодом $T$, $g$ -- монотонна и $\lim\limits_{x \to + \infty} g(x) = 0$

    \begin{enumerate}
        \item Если $\int\limits_a^{+ \infty} |g|$ сходится, то $\int\limits_a^{+ \infty} fg$ сходится абсолютно
        \item Если $\int\limits_a^{+ \infty} |g|$ расходится, то $\int\limits_a^{+ \infty} fg$ сходится $\Leftrightarrow \int\limits_a^{a + T} f = 0$
    \end{enumerate}
\end{theo}

\textit{Доказательство:}

\begin{enumerate}
    \item $f$ непрерывна на $[a, a + T] \Rightarrow$ ограничена на $[a, a + T] \Rightarrow$ ограничена, т.к. периодична

    $fg = O(g) \Rightarrow \int\limits_a^{+ \infty} fg$ абсолютно сходится по признаку сравнения

    \item 
    
    \begin{itemize}
        \item[$\Leftarrow$] $\int\limits_a^{a + T} f = 0$, тогда $F(x) := \int\limits_a^x f$ -- ограниченная функция
        
        Проверим, что $F$ периодична с периодом $T$

        $F(x + T) = \int\limits_a^{x + T} f = \int\limits_a^{a + T} f + \int\limits_{a + T}^{x + T} f = 0 + \int\limits_a^{x} f = F(x) \Rightarrow F$ ограничена $\Rightarrow$ принцип Дирихле

        \item[$\Rightarrow$] Пусть $\int\limits_a^{a + T} f = b \neq 0$. Тогда $\int\limits_a^{a + T} (f - \frac{b}{T}) = 0 \Rightarrow \int\limits_a^{+ \infty} (f - \frac{b}{T})g$ сходится 
        
        $\int\limits_a^{+ \infty} fg = \int\limits_a^{+ \infty} (f - \frac{b}{T})g + \frac{b}{T}\int\limits_a^{+ \infty} g$. Первое сходится, второе расходится $\Rightarrow \int\limits_a^{+ \infty} fg$ расходится
    \end{itemize}
\end{enumerate}

\begin{Example}{}
    $\int\limits_1^{+ \infty} \frac{\sin{x}}{x^p}dx$

    \begin{itemize}
        \item[$p > 1$] $|\frac{\sin{x}}{x^p}| \leq \frac{1}{x^p}$
        
        $\int\limits_1^{+ \infty} \frac{dx}{x^p}$ сходится $\Rightarrow \int\limits_1^{+ \infty} \frac{\sin{x}}{x^p}dx$ сходится абсолютно по признаку сравнения

        \item[$p > 0$] $\sin$ -- периодическая функция с периодом $2\pi$
        
        $\int\limits_0^{2\pi} \sin{x}dx = 0$

        $f(x) = \sin{x};\ g(x) = \frac{1}{x^p} \xrightarrow[x \to + \infty]{} 0$ монотонно $\Rightarrow \int\limits_1^{+ \infty} \frac{\sin{x}}{x^p}dx$ сходится 

        А что с абсолютной сходимостью? $\int\limits_0^{2\pi} |\sin{x}|dx > 0$

        $f(x) = |\sin{x}|;\ g(x) = \frac{1}{x^p} \xrightarrow[x \to + \infty]{} 0$ монотонно

        $\int\limits_1^{+ \infty} \frac{dx}{x^p}$ -- расходится $\Rightarrow \int\limits_1^{+ \infty} \frac{|\sin{x}|}{x^p}dx$ расходится $\Rightarrow \int\limits_1^{+ \infty} \frac{\sin{x}}{x^p}dx$ не имеет абсолютной сходимости

        \item[$p \leq 0$] Рассмотрим отрезок $[2\pi k + \frac{\pi}{6}, 2\pi k + \frac{5\pi}{6}]$. На нем $\sin{x} \geq \frac{1}{2}$
        
        $\int\limits_{2\pi k + \frac{\pi}{6}}^{2\pi k + \frac{5\pi}{6}} \frac{\sin{x}}{x^p}dx \geq \frac{1}{2} \int\limits_{2\pi k + \frac{\pi}{6}}^{2\pi k + \frac{5\pi}{6}} \frac{dx}{x^p} \geq \frac{1}{2} \cdot \frac{2\pi}{3} \cdot (2\pi k + \frac{\pi}{6})^{-p} \geq \frac{\pi}{3}$ -- противоречие с условие критерия Коши, значит $\int\limits_1^{+ \infty} \frac{\sin{x}}{x^p}dx$ расходится
    \end{itemize}
\end{Example}

\begin{Remark}{}
    В признаках Абеля и Дирихле нельзя отказаться от монотонности $g$

    $f(x) = \sin{x};\ g(x) = \frac{\sin{x}}{x}$

    $|\int\limits_1^x \sin{x}dx| = |\cos{1} - \cos{x}| \leq 2;\ \lim\limits_{x \to + \infty} g(x) = 0$

    $\int\limits_1^{+ \infty} fg = \int\limits_1^{+ \infty} \frac{\sin^2{x}}{x}dx$ 

    $\int\limits_0^{2\pi} \sin^2{x}dx = \pi > 0;\ \int\limits_1^{+ \infty} \frac{dx}{x}$ -- расходится

    $\Rightarrow \int\limits_1^{+ \infty} \frac{\sin^2{x}}{x}dx$ расходится
\end{Remark}

\section{Анализ в метрических пространствах}

\subsection{\S 1. Метрические пространства}

\begin{defin}{Метрика}
    $X$ -- множество. $\rho : X \times X \to \R$ -- метрика (расстояние)

    \begin{enumerate}
        \item $\rho(x, y) \geq 0$ и $\rho(x, y) = 0 \Leftrightarrow x = y$
        \item $\rho(x, y) = \rho(y, x)$
        \item Неравенство треугольника: $\rho(x, z) \leq \rho(x, y) + \rho(y, z)$
    \end{enumerate}
\end{defin}

\begin{defin}{Метрическое пространство}
    $(X, \rho)$ -- метрическое пространство
\end{defin}

\begin{Example}{}
    \begin{enumerate}
        \item Дискретная метрика (метрика лентяя) $\rho(x, y) = \begin{cases}
            0 & x = y \\
            1 & x \neq y
        \end{cases}$
        \item $X = \R;\ \rho(x, y) = |x - y|$
        \item $X = \R^2;\ \rho$ -- расстояние на плоскости
        \item $X = \R^d;\ p \geq 1$ и $\rho(x, y) = (|x_1 - y_1|^p + |x_2 - y_2|^p + \ldots + |x_d - y_d|^p)^{\frac{1}{p}}$
        
        Неравенство треугольника -- это неравенство Минковского

        \item Частный случай 4. $X = \R^2;\ \rho(x, y) = |x_1 - y_1| + |x_2 - y_2|$ -- Манхэттенское расстояние
        
        \item Французская железнодорожная метрика $X = \R^2$
        
        $\rho(A, B) =$ длина отрезка $AB$

        $\rho(C, D) = CP + PD$

        Тут кросивый рисуночек, типа чтоб проехать из города в другой оч часто надо заехать в $P$ -- пАрИж

        \item Метрика Хемминга $a_1, a_2 \ldots a_n$ слова из $n$ букв
        
        $\rho(A, B) =$ количество разрядов, в которых $A$ и $B$ различаются

        \item $X = C[a, b];\ \rho(f, g) := \max\limits_{x \in [a, b]} |f(x) - g(x)|$ -- равномерная метрика
        \item $X = C[a, b];\ \rho(f, g) := \int\limits_a^b |f(x) - g(x)|dx$ -- метрика в $L_1$
    \end{enumerate}
\end{Example}

\begin{defin}{Шар}
    $(X, \rho)$ -- метрическое пространство, $r > 0,\ a \in X$

    Открытый шар радиуса $r$ с центром в точке $a$

    $B_r(a) := \{x \in X : \rho(x, a) < r\}$

    Замкнутый шар $\overline{B_r}(a) := \{x \in X : \rho(x, a) \leq r\}$
\end{defin}

\begin{theo}{Свойства}
    \begin{enumerate}
        \item $B_{r_1}(a) \cap B_{r_2}(a) = B_{\min(r_1, r_2)}(a)$
        
        $B_{r_1}(a) \cup B_{r_2}(a) = B_{\max(r_1, r_2)}(a)$

        \item Если $a \neq b$, то найдется $r > 0 : \overline{B_r}(a) \cap \overline{B_r}(b) = \varnothing$
    \end{enumerate}
\end{theo}

\textit{Доказательство:}

\begin{enumerate}
    \item[2.] Возьмем $r = \frac{\rho(a, b)}{3} > 0$. Пусть $\overline{B_r}(a) \cap \overline{B_r}(b) \neq \varnothing$, т.е. найдется $x \in \overline{B_r}(a)$ и $x \in \overline{B_r}(b) \Rightarrow$
    
    $\Rightarrow \rho(a, b) \leq \rho(a, x) + \rho(x, b) \leq r + r = \frac{2}{3}\rho(a, b)$ -- противоречие
\end{enumerate}

\begin{defin}{Открытое множество}
    $U \subset X$ -- открытое множество, если $\forall a \in U$ найдется $B_r(a) \subset U$
\end{defin}

\begin{theo}{Свойства}
    \begin{enumerate}
        \item $\varnothing$ и $X$ -- открытые множества
        \item Объединение любого количества открытых множеств -- открытое множество
        \item Пересечение конечного количества открытых множеств -- открытое множество
        \item Открытый шар -- открытое множество
    \end{enumerate}
\end{theo}

\textit{Доказательство:}

\begin{enumerate}
    \item Очевидно
    \item Пусть $U_\alpha$ -- открытые при $\alpha \in I$. Докажем, что $U := \bigcup\limits_{\alpha \in I} U_\alpha$ -- открытое
    
    Возьмем $a \in U \Rightarrow a \in U_{\alpha_0}$ для некоторого $\alpha_0$, но $U_{\alpha_0}$ -- открытое $\Rightarrow \exists r > 0 : \\ B_r(a) \subset U_{\alpha_0} \subset U$

    \item Пусть $U_1, U_2 \ldots U_n$ открытые. Докажем, что $U := \bigcap\limits_{k = 1}^n U_k$ -- открытое
    
    Возьмем $a \in U \Rightarrow a \in U_k\ \forall k \in 1 \ldots n$. $U_k$ -- открытое $\Rightarrow \exists r_k > 0 : B_{r_k}(a) \subset U_k$

    Возьмем $r := \min(r_1, r_2 \ldots r_n) > 0 \Rightarrow B_r(a) \subset B_{r_k}(a) \subset U_k \Rightarrow B_r(a) \subset U = \bigcap\limits_{k = 1}^n U_k$

    \item $B_R(a)$ -- открытое множество
    
    Возьмем $x \in B_R(a) \Rightarrow \rho(x, a) < R$ и положим $r := R - \rho(x, a) > 0$

    Проверим, что $B_r(x) \subset B_R(a)$. Возьмем $y \in B_r(x)$ и проверим, что $y \in B_R(a)$

    $y \in B_r(x) \Rightarrow \rho(y, x) < r = R - \rho(x, a) \Rightarrow \rho(y, a) \leq \rho(y, x) + \rho(x, a) < r + \rho(x, a) = R$
\end{enumerate}

\begin{Remark}{}
    В 3 существенно, что множеств конечное число

    \begin{Example}{}
        $X = R;\ \rho(x, y) = |x - y|;\ U_n := (- \frac{1}{n}, \frac{1}{n})$ -- открытые множества

        $\bigcap\limits_{n = 1}^{+ \infty} U_n = \{0\}$ -- не открытое
    \end{Example}    
\end{Remark}

\begin{defin}{Внутренняя точка}
    $(X, \rho)$ -- метрическое пространство, $A \subset X,\ a \in A$

    $a$ -- внутренняя точка множества $A$, если $\exists r > 0 : B_r(a) \subset A$
\end{defin}

\begin{Remark}{}
    Открытое множество -- множество, все точки которого внутренние
\end{Remark}

\begin{defin}{Внутренность множества}
    $(X, \rho)$ -- метрическое пространство, $A \subset X$

    Внутренность множества -- все внутренние точки множества

    \textbf{Обозначение:} $\Int A$ (иногда $A^\circ$)
\end{defin}

\begin{theo}{Свойства}
    \begin{enumerate}
        \item $\Int A \subset A$
        \item $\Int A$ -- объединение всех открытых множеств, содержащихся в $A$
        \item $\Int A$ -- открытое множество
        \item $A$ -- открытое $\Leftrightarrow A = \Int A$
        \item $A \subset B \Rightarrow \Int A \subset \Int B$
        \item $\Int (\Int A) = \Int A$
        \item $\Int (A \cap B) = \Int A \cap \Int B$
    \end{enumerate}
\end{theo}

\textit{Доказательство:}

\begin{enumerate}
    \item Очев 
    \item Хотим $\Int A = \bigcup\limits_{G \subset A} G$, где $G$ -- открытое
    
    \begin{itemize}
        \item[$\subset$] Возьмем $a \in \Int A \Rightarrow a$ -- внутренняя точка $\Rightarrow \exists r > 0 : B_r(a) \subset A$, но $B_r(a)$ -- открытое множество $\Rightarrow B_r(a)$ присутствует среди множеств из объединения
        \item[$\supset$] Пусть $a \in U \Rightarrow \exists G$ -- откртытое $\subset A : a \in G \Rightarrow \exists r > 0 : B_r(a) \subset G \subset A \Rightarrow a$ -- внутренняя точка 
    \end{itemize}

    \item По пункту 2 $\Int A$ -- объединение открытых множеств $\Rightarrow \Int A$ -- открытое
    
    \item 
    
    \begin{itemize}
        \item[$\Leftarrow$] т.к. $\Int A$ -- открытое
        \item[$\Rightarrow$] $A$ -- открытое $\Rightarrow$ все точки внутренние $\Rightarrow$ все лежат в $\Int A \Rightarrow A = \Int A$
    \end{itemize}

    \item Если $a$ -- внутрення точка для $A$, то и для $B$ тоже, т.к. $A \subset B$
    \item $3 + 4 = 6$
\end{enumerate}

\begin{Exercise}{}
    \begin{enumerate}
        \item Доказать пункт 7
        \item Придумать пример, когда $\Int (A \cup B) \neq \Int A \cup \Int B$
    \end{enumerate}
\end{Exercise}

\begin{defin}{Замкнутое множество}
    Замкнутое множество -- множество, дополнение которого открыто 

    $A$ замкнуто $\Leftrightarrow X \setminus A$ открыто
\end{defin}

\begin{theo}{Свойства замкнутых множеств}
    \begin{enumerate}
        \item $\varnothing$ и $X$ -- замкнутые множества
        \item Пересечение любого количества замкнутых множеств -- замкнутое множество
        \item Объединение конечного количества замкнутых множеств -- замкнутое множество
        \item Замкнутый шар -- замкнутое множество
    \end{enumerate}
\end{theo}

\textit{Доказательство:}

\begin{enumerate}
    \item Очевидно, ничего писать не буду
    \item $F_\alpha$ -- замкнутые $\Rightarrow X \setminus F_\alpha$ -- открытые $\Rightarrow \bigcup X \setminus F_\alpha = X \setminus \bigcap F_\alpha$ -- открытое $\Rightarrow \bigcap F_\alpha$ -- замкнутое
    \item $F_1 \ldots F_n$ -- замкнутые $\Rightarrow X \setminus F_k$ -- открытые $\Rightarrow \bigcap X \setminus F_k = X \setminus \bigcup F_k$ -- открытое $\Rightarrow \bigcup F_k$ -- замкнутое
    \item $\overline{B}_R(a)$ -- замкнутое множество $\Leftrightarrow X \setminus \overline{B}_R(a) = \{x \in X : \rho(x, a) > R\}$ -- открытое
    
    Возьмем $r := \rho(x, a) - R$ и проверим, что $B_r(x) \subset X \setminus \overline{B}_R(a)$

    $x \notin \overline{B}_R(a)$, т.е. $B_r(x) \cap \overline{B}_R(a) = \varnothing$

    Предположим противное, тогда $\exists y \in B_r(x)$ и $y \in \overline{B}_R(a) \Rightarrow \rho(x, y) < r$ и $\rho(y, a) \leq R \Rightarrow \\ \Rightarrow \rho(x, a) \leq \rho(x, y) + \rho(y, a) < r + R = \rho(x, a)$ ?????
\end{enumerate}

\begin{defin}{Замыкание множества}
    Замыкание множества $A$ -- пересечение всех замкнутых множеств, содержащих $A$

    Обозначение $\Cl A$ или $\overline{A}$
\end{defin}

\begin{theo}{}
    $X \setminus \Cl A = \Int (X \setminus A)$ и $X \setminus \Int A = \Cl (X \setminus A)$
\end{theo}

\textit{Доказательство:}

Докажем, что $X \setminus \Cl A = \Int (X \setminus A)$

\begin{itemize}
    \item[$\subset$] Пусть $x \in X \setminus \Cl A \Rightarrow x \notin \Cl A \Rightarrow x \notin F$, где $F$ -- некоторое замкнутое множество, содержащее $A \Rightarrow x \in X \setminus F$ -- открытое и $X \setminus F \subset X \setminus A \Rightarrow x \in \Int (X \setminus A)$
    
    \item [$\supset$] Пусть $x \in \Int (X \setminus A) \Rightarrow x \in G$, где $G$ -- открытое, содержащееся в $X \setminus A \Rightarrow x \notin X \setminus G$ -- замкнутое и $X \setminus G \supset X \setminus (X \setminus A) = A \Rightarrow x \notin \Cl A \Rightarrow x \in X \setminus \Cl A$
\end{itemize}

\begin{theo}{Свойства замыкания}
    \begin{enumerate}
        \item $\Cl A \supset A$
        \item $\Cl A$ -- замкнутое множество
        \item $A$ -- замкнутое $\Leftrightarrow A = \Cl A$
        \item $A \subset B \Rightarrow \Cl A \subset \Cl B$
        \item $\Cl(\Cl A) = \Cl A$
        \item $\Cl(A \cup B) = \Cl A \cup \Cl B$
    \end{enumerate}
\end{theo}

\textit{Доказательство:}

\begin{enumerate}
    \item Очевидно
    \item Пересечение замкнутые -- замкнутое
    \item $A$ -- замкнутое $\Leftrightarrow X \setminus A$ -- открытое $\Leftrightarrow X \setminus A = \Int (X \setminus A) \Leftrightarrow A = \Cl A$
    \item $A \subset B \Rightarrow X \setminus A \supset X \setminus B \Rightarrow \Int (X \setminus A) \supset \Int (X \setminus B) \Rightarrow X \setminus \Int (X \setminus A) \subset X \setminus \Int (X \setminus B) \Rightarrow \Cl A \subset \Cl B$
    \item $2 + 3 = 5$
\end{enumerate}

\begin{Exercise}{}
    \begin{enumerate}
        \item Доказать пункт 6
        \item Придумать пример, когда $\Cl (A \cap B) \neq \Cl A \cap \Cl B$
        \item $A, \Int A, \Cl A, \Int \Cl A, \Cl \Int A, \Cl \Int \Cl A \ldots$
        
        Какое наибольше количество различных множеств может получиться?
    \end{enumerate}
\end{Exercise}

\begin{theo}{}
    $x \int \Cl A \Leftrightarrow \forall r > 0\ B_r(x) \cap A \neq \varnothing$
\end{theo}

\textit{Доказательство:}

$x \in Cl A = X \setminus \Int (X \setminus A) \Leftrightarrow x \notin \Int (X \setminus A) \Leftrightarrow \forall r > 0\ B_r(x) \not\subset X \setminus A \Leftrightarrow \forall r > 0\ B_r(x) \cap A \neq \varnothing$

\begin{defin}{Окрестность и проколотая окрестность}
    Окрестность точки $a$ -- шарик радиуса $r > 0$ с центром в точке $a$

    Обозначение $U_a$

    Проколотая окрестность -- $\mathring{U}_a = U_a \setminus \{a\}$
\end{defin}

\begin{defin}{Предельная точка}
    $x$ -- предельная точка множества $A$, если $\forall \mathring{U}_x$ содержится точки множества $A$

    Локальное обозначение (примерно на 1 лекцию): $A'$ -- множество предельных точек множества $A$
\end{defin}

\begin{theo}{Свойства}
    \begin{enumerate}
        \item $\Cl A = A \cup A'$
        \item $A \subset B \Rightarrow A' \subset B'$
        \item $A$ -- замкнутое $\Leftrightarrow A \supset A'$
        \item $(A \cup B)' = A' \cup B'$
    \end{enumerate}
\end{theo}

\textit{Доказательство:}

\begin{enumerate}
    \item $x \in \Cl A \Leftrightarrow \forall U_x\ U_x \cap A \neq \varnothing$

    Это так, если $x \in A$ или $x \in A'$ (тут $\mathring{U}_a \cap A \neq \varnothing$)

    \item очев 
    
    \item $A$ -- замкнутое $\Leftrightarrow A = \Cl A \Leftrightarrow A = A \cup A' \Leftrightarrow A \supset A'$
    
    \item 
    
    \begin{itemize}
        \item[$\supset$] $A \cup B \supset A \Rightarrow (A \cup B)' \supset A'$
        \item[$\subset$] Возьмем $x \in (A \cup B)'$ и предположим, что $x \notin A'$. Надо доказать, что $x \in B'$
        
        $x \in (A \cup B)' \Rightarrow \forall r > 0\ \mathring{B_r}(x) \cap (A \cup B) \neq \varnothing$

        $x \notin A' \Rightarrow \exists R > 0 : \mathring{B_R}(x) \cap A = \varnothing \Rightarrow \forall r < R\ \mathring{B_r}(x) \cap A = \varnothing$

        Тогда $\mathring{B_r}(x) \cap B \neq \varnothing \forall r < R \Rightarrow$ вообще любой $\mathring{B_r}(x)$ пересекается с $B \Rightarrow x \in B'$
    \end{itemize}
\end{enumerate}

\begin{theo}{}
    Следующие условия равносильны

    \begin{enumerate}
        \item $x \in A'$
        \item $\forall r > 0\ B_r(x)$ содержит бесконечное количество точек из $A$
        \item Существует последовательность различных точек $x_n \in A : \rho (x_n, x) \to 0$
        \item Существует последовательность $x_n \in A : \rho(x_n, x)$ строго убывает и $\to 0$
    \end{enumerate}
\end{theo}

\textit{Доказательство:}

\begin{enumerate}
    \item[$4 \Rightarrow 3$] очевидно
    \item[$3 \Rightarrow 2$] Если $\rho(x_n, x) \to 0$, то при больших и $\rho(x_n, x) < r$, т.е. $x_n \in B_r(x) \Rightarrow$ в $B_r(x)$ бесконечное количество точек из $A$
    \item[$2 \Rightarrow 1$] Если $B_r(x) \cap A$ состоит из бесконечного количество точек, то $\mathring{B_r}(x) \cap A \neq \varnothing$ (множества отличаются на одну точку или совпадают)
    \item[$1 \Rightarrow 4$] Рассмотрим $r = 1$. В открестности $\mathring{B_r}(x)$ есть точка из $A$, назовем ее $x_1 \neq x$
    
    $r = \frac{1}{2}\rho(x_1, x) < \frac{1}{2}$. В окрестности $\mathring{B_r}(x)$ есть точка из $A$, назовем ее $x_2 \neq x$

    $\rho(x_2, x) < r < \rho(x_1, x)$

    $r = \frac{1}{2}\rho(x_2, x) < \frac{1}{4}$. В окрестности $\mathring{B_r}(x)$ есть точка из $A$, назовем ее $x_3 \neq x$

    $\rho(x_3, x) < \rho(x_2, x)$ и так далее

    $x_1, x_2 \ldots \in A$, $\rho(x_1, x) > \rho(x_2, x) > \ldots$ и $\rho(x_n, x) < \frac{1}{2^n} \Rightarrow \rho(x_1, x) \to 0$
\end{enumerate}

\begin{theo}{Очевидное следствие}
    Конечное множество не имеет предельных точек
\end{theo}

\begin{defin}{Подпространство}
    $(X, \rho)$ -- метрическое пространство $Y \subset X$

    $(Y, \rho|_{Y \times Y})$ -- подпространство метрического пространства $(X, \rho)$
\end{defin}

\begin{theo}{Об открытых множества в пространстве и подпространстве}
    $(X, \rho)$ -- метрическое пространство, $Y \subset X$. Тогда

    \begin{enumerate}
        \item $A$ -- открыто в $(Y, \rho) \Leftrightarrow \exists G \subset X$, открытое в $(X, \rho) : A = G \cap Y$
        \item $A$ -- замкнуто в $(Y, \rho) \Leftrightarrow \exists F \subset X$, замкнутое в $(X, \rho) : A = F \cap Y$
    \end{enumerate}
\end{theo}

\textit{Доказательство:}

\begin{enumerate}
    \item 
    
    \begin{itemize}
        \item[$\Rightarrow$] $A$ открыто в $(Y, \rho) \Rightarrow A = \bigcup\limits_{x \in A} B^Y_{r_x}(x)$, где $r_x$ -- такой радиус, что $B^Y_{r_x}(x) \subset A$
        
        $B^Y_{r_x}(x) = \{z \in Y : \rho(z, x) < r_x\} = B^X_{r_x}(x) \cap Y$

        $\Rightarrow A = \bigcup\limits_{x \in A} (B^X_{r_x}(x) \cap Y) = Y \cap \bigcup\limits_{x \in A} B^X_{r_x}(x)$, где $G := \bigcup\limits_{x \in A} B^X_{r_x}(x)$ -- открытое в $(X, \rho)$ как объединение открытых

        \item[$\Leftarrow$] $A = G \cap Y$, $G$ -- открытое в $X$. Докажем, что $A$ открыто в $Y$
        
        Возьмем $a \in A \subset G \Rightarrow \exists r > 0 : B^X_r(a) \subset G \Rightarrow B^Y_r(a) = B^X_r(a) \cap Y \subset G \cap Y = A$
    \end{itemize}

    \item $A$ -- замкнуто в $(Y, \rho) \Leftrightarrow Y \setminus A$ -- открыто в $(Y, \rho) \Leftrightarrow \exists G \subset X$ открытое в \\ 
    $(X, \rho) : Y \setminus A = G \cap Y \Rightarrow Y \cap (X \setminus G) = A$, где $X \setminus G$ -- замкнутое $\Leftrightarrow G$ -- открытое 
\end{enumerate}

\begin{Example}{}
    $X = \R,\ Y = [0, 3)$

    $[0, 1)$ -- открыто в $Y$, т.к. $(-1, 1)$ -- открыто в $\R$, а $[0, 1) = (-1, 1) \cap Y$

    $[2, 3)$ -- замкнуто в $Y$, т.к. $[2, 4]$ -- замкнуто в $\R$, а $[2, 3) = [2, 4] \cap Y$
\end{Example}

\begin{defin}{Норма}
    $X$ -- векторное пространство. $\parallel \parallel$ -- норма на $X$, если 

    $\parallel \cdot \parallel : X \to R$

    \begin{enumerate}
        \item $\parallel x \parallel \forall x \in X$ и $\parallel x \parallel = 0 \Leftrightarrow x = \overrightarrow{0}$
        \item $\parl{\alpha x} = |\alpha| \cdot \parl{x}\ \forall x \in X\ \forall \alpha \in \R$
        \item $\parl{x + y} \leq \parl{x} + \parl{y}\ \forall x, y \in X$ (неравенство треугольника)
    \end{enumerate}
\end{defin}

\begin{Example}{}
    \begin{enumerate}
        \item $X = \R;\ |x|$ -- норма
        \item $X = \R^d;\ \parl{x}_p = (|x_1|^p + |x_2|^p + \ldots + |x_d|^p)^{\frac{1}{p}}$ -- норма, $p \geq 1$
        
        Неравенство треугольника -- это неравенство Минковского

        \item $X = C[a, b];\ \parl{f} := \max\limits_{x \in [a, b]} |f(x)|$
        \item $X = C[a, b];\ \parl{f} := \int\limits_a^b |f|$
    \end{enumerate}
\end{Example}

\begin{defin}{Скалярное произведение}
    $X$ -- векторное пространство, $\q{\cdot, \cdot} : X \times X \to \R$ -- скалярное произведение, если

    \begin{enumerate}
        \item $\q{x, x} \geq 0\ \forall x \in X$ и $\q{x, x} = 0 \Leftrightarrow x = \overrightarrow{0}$
        \item $\q{x, y} = \q{y, x}\ \forall x, y \in X$
        \item $\q{\alpha x, y} = \alpha \q{x, y}\ \forall x, y \in X\ \forall \alpha \in \R$
        \item $\q{x + y, z} = \q{x, z} + \q{y, z}\ \forall x, y, z \in X$
    \end{enumerate}
\end{defin}

\begin{Example}{}
    \begin{enumerate}
        \item $X = \R^d;\ \q{x, y} = x_1y_1 + x_2y_2 + \ldots + x_dy_d$
        
        $x = (x_1 \ldots x_d);\ y = (y_1 \ldots y_d)$

        \item $X = \R^d;\ w_1, w_2 \ldots w_d > 0$
        
        $\q{x, y} = w_1x_1y_1 + w_2x_2y_2 + \ldots + w_dx_dy_d$

        \item $X = C[a, b];\ \q{f, g} = \int\limits_a^b f(x)g(x)dx$
    \end{enumerate}
\end{Example}

\begin{theo}{Свойства}
    \begin{enumerate}
        \item Неравенство Коши-Буняковского
        
        $\q{x, y}^2 \leq \q{x, x} \cdot \q{y, y}$

        \item $\parl{x} := \sqrt{\q{x, x}}$ -- норма
        \item $\parl{\cdot}$ -- норма в $X \Rightarrow \rho(x, y) := \parl{x - y}$ -- метрика в $X$
        \item $\parl{x - y} \geq |\parl{x} - \parl{y}|$
    \end{enumerate}
\end{theo}

\newpage

\textit{Доказательство:}

\begin{enumerate}
    \item $f(t) := \q{x + ty, x + ty} \geq 0$
    
    $f(t) = \q{x, x + ty} + t\q{y, x + ty} = \q{x, x} + t\q{x, y} + t\q{y, x} + t^2\q{y, y} = t^2\q{y, y} + 2t\q{x, y} + \q{x, x}$ -- квадратный трехчлен

    $\Rightarrow f$ имеет не больше одного корня $\Rightarrow D \leq 0 \Leftrightarrow (2\q{x, y})^2 - 4\q{x, x}\q{y, y} \leq 0$

    Сократили на 4 и победили

    \item $\parl{\alpha x} = \sqrt{\q{\alpha x, \alpha x}} = \sqrt{\alpha^2\q{x, x}} = |\alpha|\sqrt{\q{x, x}} = |\alpha|\parl{x}$
    
    Неравенство треугольника: $\parl{x + y} \stackrel{?}{\leq} \parl{x} + \parl{y}$, т.е. $\sqrt{\q{x + y, x + y}} \leq \sqrt{\q{x, x}} + \\ + \sqrt{\q{y, y}}$

    Т.е. $\q{x + y, x + y} \leq \q{x, x} + \q{y, y} + 2\sqrt{\q{x,x}} \cdot \sqrt{\q{y, y}}$

    Но $\q{x + y, x + y} = \q{x, x} + 2\q{x, y} + \q{y, y}$. После сокращения осталось КБШ

    \item $\parl{x - y} \geq 0$ и $\parl{x - y} = 0 \Leftrightarrow x - y = \overrightarrow{0} \Leftrightarrow x = y$
    
    $\rho(y, x) = \parl{y - x} = \parl{(-1)(x - y)} = |-1| \cdot \parl{x - y} = \rho(x, y)$

    Неравенство треугольника: $\rho(x, y) + \rho(y, z) \geq \rho(x, z)$

    $\parl{x - y} + \parl{y - z} \geq \parl{x - y + y - z} = \parl{x - z}$

    \item Нужно доказать, что $-\parl{x - y} \leq \parl{x} - \parl{y} \leq \parl{x - y}$
    
    \begin{itemize}
        \item Правое:
        
        $\parl{x} - \parl{y} \leq \parl{x - y}$, т.е. $\parl{x - y} + \parl{y} \geq \parl{x - y + y} = \parl{x}$

        \item Левое:
        
        $\parl{x} - \parl{y} \geq -\parl{x - y}$, т.е. $\parl{x - y} + \parl{x} = \parl{y - x} + \parl{x} \geq \parl{y - x + x} = \parl{y}$
    \end{itemize}
\end{enumerate}

\begin{Exercise}{}
    Доказать, что норма $\parl{}$ в векторном пространстве $X$ порождается некоторым скалярным произведением $\Leftrightarrow \parl{x + y}^2 + \parl{x - y}^2 = 2\parl{x}^2 + 2\parl{y}^2$
\end{Exercise}

\end{document}