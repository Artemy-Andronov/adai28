\documentclass[12pt]{article}
\usepackage{config}
\usepackage{subfiles}

\def\multiset#1#2{\ensuremath{\left(\kern-.3em\left(\genfrac(){0pt}{}{#1}{#2}\right)\kern-.3em\right)}}
\def\divby{%
  \mathrel{\text{\vbox{\baselineskip.65ex\lineskiplimit0pt\hbox{.}\hbox{.}\hbox{.}}}}%
}
\newcommand{\q}[1]{\langle #1 \rangle}

\begin{document}

\tableofcontents
\newpage

\begin{flushright}
    Конспект Шорохова Сергея

    Если нашли опечатку/ошибку - пишите @le9endwp
\end{flushright}

\section{План на 3 модуль (или 2 сем...)}

\begin{enumerate}
    \item Множества
    \item ЧУМ
    \item Исчисление высказываний
    \item Исчисление предикатов
    \item Теория кодирования
\end{enumerate}

Почитать можно А. Х. Шеня

\section{Множества}

\begin{enumerate}
    \item $x \in A;\ y \not \in A$
    \item Арифметика множеств: $\bigcup, \bigcap, \backslash, \triangle$
    \item $\varnothing$
    \item $A = \{ a, b, c \};\ B = \{d\} \bigcup A$
    \item $A \subset B \Leftrightarrow \forall x \in A \Rightarrow x \in B$
\end{enumerate}

\begin{Remark}{}
    Чисто синтаксически вот такой бред: $\{ \varnothing, \{ \varnothing \}, \{ \varnothing, \{ \varnothing \} \} \}$ имеет смысл
\end{Remark}

$X$ -- множество: $X \neq \varnothing$. Рассмотрим $x \in X$

$Term(x)$ -- проблема, потому что мы не знаем, к каким характеристикам обращаемся и вообще не понятно, что мы выбрали

Спасают аксиомы ZFC

\begin{defin}{Равномощность}
    $A, B$ -- равномощны $\Leftrightarrow \exists f : A \rightarrow B$ -- биекция

    А что с бесконечностями? Давайте возьмем функцию $f : N \rightarrow 2N$

    Хотя множество четных чисел -- подмножество всех, но они равномощны, т.к. $f$ -- биекция
\end{defin}

\begin{defin}{Характеристическая функция}
    $X$ -- множество. Есть $\chi : X \rightarrow \{0, 1\}$, т.е. $\chi(x) = \begin{cases}
        1,\ x \in X \\
        0,\ x \not\in X
    \end{cases}$ -- характеристическая функция
\end{defin}

А пусть $X \subset Y$

\begin{itemize}
    \item произведение характеристических функций $X$ и $Y$ -- это характеристическая функция $X \bigcap Y$
    \item $1 - \chi(x)$ -- характеристическая функция дополнения $X$
    \item $max(\chi_X(x), \chi_Y(x))$ -- характеристическая функция $X \bigcup Y$
    \item $|X| = \sum\limits_{x \in Y} \chi_X(x)$
\end{itemize}

\begin{Example}{}
    Возьмем $2^N;\ B = \{ 0, 1 \}$ и $B^\infty$

    Равномощны ли они? Берем $x \in 2^N$, теперь $b_i = \begin{cases}
        1,\ i \in x \\
        0,\ i \not\in x
    \end{cases}$
\end{Example}

\begin{defin}{Счетное множество}
    $X$ -- счетное, если $X$ равномощно $N$
\end{defin}

\begin{Example}{}
    Например, множество целых чисел счетно, т.к. $x \in Z \Rightarrow \begin{cases}
        2x,\ x \geq 0 \\
        -2x + 1,\ x < 0
    \end{cases}$
\end{Example}

\begin{propos}{}
    \begin{enumerate}
        \item $X$ -- счетно и $Y \subset X \Rightarrow Y$ или счетно, или конечно
        \item $X$ -- бесконечно. Тогда $\exists Y$ -- счетное: $Y \subset X$
        \item $X_1, \ldots X_n \ldots$ -- конечные или счетные. Тогда $\bigcup X_i$ -- конечное или счетное
    \end{enumerate}
\end{propos}

\textit{Доказательство:}

\begin{enumerate}
    \item $X$ -- счетно, т.е. соответствует последовательности $\{x_1, \ldots x_n \ldots\} = \xi$
    
    Возьмем $\xi \cdot \chi(Y)$. Т.е. что-то типа $\{ 0, 0 \ldots x_{i_1}, 0 \ldots x_{i_2}, 0 \ldots \}$ который равносилен $y_1, y_2, \ldots y_n \ldots = Y$

    В свою очередь эта штука либо конечна, либо счетна, т.к. счетен $X$

    \item Просто выбираем по 1 элементу из $X$. Если они кончатся на каком-то шаге -- $X$ не бесконечно
    
    \item Рисуем табличку. Берем элемент (1, 1), потом (1, 2), потом (2, 1), потом (1, 3) и так далее. То есть по диагоналям. Так переберем вообще все элементы (если не понятно, погуглите метод Кантора)
\end{enumerate}

\begin{Exercise}{}
    В качестве следствия попробуйте построить явную биекцию между множеством рациональных чисел и натуральных
\end{Exercise}

\begin{theo}{}
    $A$ -- бесконечно, $B$ -- нбчс, т.е. $B$ -- конечно или счетно

    $A \bigcup B$ равномощно $A$
\end{theo}

\textit{Доказательство:}

$\exists Y \subset A$ -- счетное

$Y$ и $Y \bigcup B$ -- равномощны

$A \bigcup B = (A \backslash Y) \bigcup (Y \bigcup B)$

$A = Y \bigcup (A \backslash Y)$

Биекция между $Y$ и $Y \bigcup B$ сущесвтует, значит $A$ и $A \bigcup B$ равномощны

\begin{Example}{}
    $[0; 1]$ и $B^\infty$. Равномощны ли? Да. Последовательность единиц и нулей -- это бинпоиск числа

    Проблема: $0,(9) = 1,(0)$

    $b_1 \ldots b_k, 1, 1, 1, 1, (1)$

    $(b_1 \ldots b_k) + 1$

    $R \bigcup [0, 1] \sim B^{\infty}$ и $R \bigcup [0, 1] \sim [0, 1] \Rightarrow [0, 1] \sim B^{\infty}$
\end{Example}

\begin{Example}{}
    $[0, 1] \sim [0, 1] \times [0, 1]$

    $0, a_1a_2 \ldots a_k \ldots$

    $0, a_1a_3a_5 \ldots$ и $0, a_2a_4a_6 \ldots$

    \begin{Exercise}{}
        Проблема та же, что и в прошлом примере, но число уязвимых моментов кратно больше. Почините
    \end{Exercise}
\end{Example}

\begin{theo}{Кантор-Бернштейн}
    $A, B;\ A_1 \subset A;\ B_1 \subset B$

    $A_1 \sim B, B_1 \sim A \Rightarrow A \sim B$
\end{theo}

\textit{Доказательство:}

$A$ имеет мощность не больше $B$. Существует какое-то отображение. Нужна его биективность. А где-то по пути может докажем еще и полный порядок

$f: A \rightarrow B_1$ -- биекция

$g : B \rightarrow A_1$ -- еще одна биекция

Заметим, что $g(f(A)) = A_2$ -- биекция, более того этот процесс можно продолжить до бесконечности 

То есть имеем $A \supset A_1 \supset A_2 \ldots$ и $A \sim A_2 \sim A_4 \ldots$ и $A_1 \sim A_3 \sim A_5 \sim \ldots$

Возьмем просто много вложенных $C$-шек таких, что $C \rightarrow C_2 \rightarrow C_4 \ldots$ и $C_1 \rightarrow C_3 \ldots$ при какой-то биекции $h$

Как построить биекцию из $C_6$ в $C_7$? Положим $D_i = C_i \setminus C_{i + 1}$. Тогда $C_0 = D_0 \bigcup D_1 \bigcup D_2 \ldots$

При этом $C_1 = D_1 \bigcup D_2 \bigcup D_3 \ldots$

$D_2 = C_2 \setminus C_3;\ D_0 = C_0 \setminus C_1$. Ну тогда $C_2 = D_2 \bigcup C_3$ и $C_0 = D_0 \bigcup C_1$

При этом биекция $h$ все еще существует. Можем сопоставить $D_{2k} \rightarrow D_{2(k + 1)}$, а $D_{2k + 1} \rightarrow D_{2k + 1}$, т.е. построить биекцию между $C_0$ и $C_1$. Победа

Явная биекция: $q(x) = \begin{cases}
    x, x \in D_{2i + 1} \\
    h(x), x \in D_{2i}
\end{cases}$

\begin{theo}{Теорема Кантора}
    $B^{\inf}$ -- не счетно
\end{theo}

\textit{Доказательство:}

Построили последовательность типа 

\begin{enumerate}
    \item $a_1, a_2 \ldots$
    \item $b_1, b_2 \ldots$
    \item $c_1, c_2 \ldots$
\end{enumerate}

Ну возьмем еще одну последовательность $a_1, b_2, c_3 \ldots$ -- она будет отличаться от всех предыдущих как минимум в одном элементе. Значит $B^{\inf}$ не счетно

\begin{theo}{Обобщенная теорема Кантора}
    $\forall X,\ X \not\sim 2^X$
\end{theo}

\textit{Доказательство:}

Пусть $\exists \varphi : X \rightarrow 2^X$ -- биекция

$Z = \{ x | x \not\in \varphi(x) \}$

$Z \subset X$

$\not\exists z : \varphi(z) = Z \Rightarrow z \not\in Z \Rightarrow z \in Z$

\begin{theo}{Следствие}
    $|2^X| > |X|$

    $\N, 2^{\N}, 2^{2^{\N}}, \ldots$

    $\aleph_0, \aleph_1, \ldots$
\end{theo}

\begin{Remark}{}
    Почему не существует множества всех множеств?

    Пусть существует и называется $U$

    Посмотрим на $U$ и $2^U$

    По Кантору-Бернштейну $U \sim 2^U$, но по теореме Кантора $|U| < |2^U|$ ???? 
\end{Remark}

\begin{theo}{}
    $A$ и $B$ -- множества

    $|A \cup B| = |A| + |B| - |A \cap B|$ при $|A|, |B| < + \infty$

    Если же $|A| = + \infty$, а $|B| < + \infty$, то $|A \cup B| = |A|$

    Что если $|A| = + \infty$ и $|B| = + \infty$? Скажем, НУО $|A| \leq |B|$, тогда $|A \cup B| = |B|$

\begin{Remark}{}
    Вообще мы умеем еще и $|A \times B|$, но там разница будет только в конечных множествах
\end{Remark}

    Есть так же и возведение в степень. С нбчс работа очевидна, а вот с не нбчс уже не так просто

    Что такое $|A|^{|B|}$? Такое описать нормально не получится

    \begin{defin}{}
        Нечто абстрактное и <<умозрительное>> -- $\aleph$

        Так, например, $\aleph + n = \aleph$ и $\aleph \cdot n = \aleph$
    \end{defin}
\end{theo}

\begin{defin}{$\geq$}
    $X$ -- множество
    
    <<$\geq$>> $\subset X \times X$

    \begin{enumerate}
        \item $\forall x \in X \Rightarrow x \geq x$
        \item $\forall x, y, z : x \geq y,\ y \geq z \Rightarrow x \geq z$
        \item $\forall x, y : x \neq y,\ x \geq y \Rightarrow \overline{y \geq x}$
        \item[$\tilde{3}$] $\forall x, y \in X : x \geq y,\ y \geq x \Rightarrow x = y$
    \end{enumerate}
\end{defin}

\begin{theo}{Порядок}
    Заведем отношение $\geq$. Если оно существует для всех пар множества, то это порядок, иначе -- частичный порядок

    Заметим, что он нестрогий. Для строгого нужно добавить проверку на равенство
\end{theo}

\begin{defin}{Частично упорядоченное множество}
    $(X, \geq_X)$ -- ЧУМ
\end{defin}

\begin{Example}{}
    Взяли $\N$ и степенной порядок, т.е.

    $a, b \in \N;\ \exists x \in N\ (x > 1)\ : \begin{cases}
        a = x^k \\
        b = x^m
    \end{cases}$

    $a \geq b \Leftrightarrow k \geq m$
\end{Example}

\begin{defin}{Индуцированный порядок}
    Рассмотрим $Y \subset X$. Если пользоваться тем же отношением порядка на $Y \times Y$, то можно смотреть на $\geq_Y = (\geq_X) \cap (Y \times Y)$ -- индуцированный порядок
\end{defin}

\begin{Remark}{}
    Можно и на $X \times Y$ ввести $\geq_{X \times Y} : (x, y) \geq (a, b) \Leftrightarrow \left[ \begin{gathered}
        x \geq a \\
        \begin{cases}
            x < a \\
            y \geq b
        \end{cases}
    \end{gathered} \right.$

    Такой порядок называется лексикографическим (покоординатным), что в целом то же, что и $(X, \geq_X) + (Y, \geq_Y)$
\end{Remark}

\begin{defin}{Наибольший и максимальный элемент}
    $x \in X$ -- наибольший элемент $\Leftrightarrow \forall y \in X : y < x$

    $x \in X$. Если $\not\exists y \in X : y > x$, то $x$ -- максимальный элемент

    \begin{Remark}{}
        Наибольший элемент -- всегда максимальный, но не наоборот
    \end{Remark}
\end{defin}

\begin{defin}{Изоморфизм}
    $(X, \geq_X) \sim (Y, \geq_Y)$ -- изоморфизм, если $\exists f : X \rightarrow Y$ -- биекция, сохраняющая порядок
\end{defin}

Что можно сказать про $(\R, \geq_\R)$? Можно построить биекцию $x \mapsto x + 1$ -- это автоморфизм

А что с $\R_+, \geq_{\R_+}$? Тут уже не получится построить автоморфизм (т.к. из луча $(0, +\infty)$ уйдем в луч $(1, +\infty)$)

\begin{Remark}{}
    Из существования биекции не следует существование автоморфизма

    Берем $X, Y;\ h : X \to Y$ -- биекция

    И $\forall x, y \in X : x \geq y \Rightarrow h(x) \geq h(y)$

    Смотрим на $\Z, \Q$. Пусть $\exists h : \Z \to \Q$ 

    Рассмотрим двойку и тройку

    $\not\exists x \in \Z : 2 < x < 3$

    $h(2) = y_2;\ h(3) = y_3$

    $h^{-1}(\frac{y_2 + y_3}{2}) = x$

    Целого числа между 2 и 3 нет, но по биекции оно есть
\end{Remark}

\begin{defin}{Плотность}
    $x$ -- плотная точка, если

    $\begin{cases}
        \forall y < x\ \exists z : y < z < x \\
        \forall y > x\ \exists z : x < z < y
    \end{cases}$
\end{defin}

\begin{Example}{}
    Возьмем множество $\{0, \frac{1}{2}, \frac{1}{3}, \ldots \frac{1}{n} \ldots\}$. В нем плотная точка -- только 0
\end{Example}

\begin{theo}{}
    $X$ -- всюду плотное (нет соседних элементов), счетное, без наибольшего и наименьшего элемента

    Это значит, что $X \cong \Q$
\end{theo}

\textit{Доказательство:}

Возьмем $n$ точек из $X$ и $n$ точек из $\Q$. Построим между ними изоморфизм

Теперь нам нужен изоморфизм из $n + 1$ отрезков из $X$ в $n + 1$ отрезок множества $\Q$. Далее идем рекурсивно

Получим для точки что-то типа системы стягивающихся отрезков

\begin{Exercise}{}
    Попробуйте придумать явный изоморфизм между $\Q$ и $\Q \cap (0, 1)$
\end{Exercise}

\begin{Remark}{}
    $x \to x + 1$ -- автоморфизм $\Z$

    $h(\N) \neq \N$

    Пусть есть изоморфизм $g(\Z) \to \N$

    Применим прошлую функцию и получим $h(g(\Z)) \to h(\N)$

    Но $g(h(\Z)) = g(\Z) = \N$, а $h(\N) \neq \N$
\end{Remark}

\begin{nota}{}
    $\forall m < n;\ A(m)$ -- истина $\Rightarrow A(n)$ (если $A(0)$)
\end{nota}

\begin{theo}{}
    $X$ -- ЧУМ

    \begin{enumerate}
        \item $\forall Y \subset X;\ \exists \min Y$
        \item $\not\exists x_1, x_2 \ldots x_n \ldots : x_1 > x_2 > x_3 > \ldots > x_n \ldots$
        \item Для $X$ работает принцип индукции
    \end{enumerate}
\end{theo}

\begin{Remark}{}
    Переформулируем 3 пункт: $A$ -- какое-то произвольное свойство, тогда

    $(\forall x(\forall y(y < x) \Rightarrow A(y)) \Rightarrow A(x)) \Rightarrow \forall x\ A(x)$
\end{Remark}

\textit{Доказательство:}

\begin{itemize}
    \item[$2 \Rightarrow 1$] Пусть $\exists Y :$ в $Y$ нет минимального элемента
    
    Рассмотрим $X_1 \in Y \Rightarrow \exists x_2 < x_1 \Rightarrow \exists x_3 < x_2 \ldots$

    \item[$1 \Rightarrow 2$] очев 
    
    \item[$1 \Rightarrow 3$] Пусть $Y \neq \varnothing;\ \forall y \in Y\ \overline{A}(y)$
    
    $X \setminus Y = A(x)$

    $\exists y_0 = \min Y \Rightarrow \forall x < y_0\ A(x)$

    Тут что-то было

    \item[$3 \Rightarrow 1$] Пусть $\exists Y : \not\exists \min Y$
    
    $A(x) \sim x \notin Y$

    Смотрим на какой-то $x \in Y$. $\forall y < x\ A(y)$

    Дословно: если для какого-то $x$ выполнялось бы условие выше, то $x$ был минимальным, а минимального нет, значит $x$ нет
\end{itemize}

\end{document}